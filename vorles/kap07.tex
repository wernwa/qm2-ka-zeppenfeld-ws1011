\documentclass[10pt,a4paper,oneside,fleqn]{article}
\usepackage{geometry}
\geometry{a4paper,left=20mm,right=20mm,top=1cm,bottom=2cm}
\usepackage[utf8]{inputenc}
%\usepackage{ngerman}
\usepackage{amsmath}                % brauche ich um dir Formel zu umrahmen.
\usepackage{amsfonts}                % brauche ich für die Mengensymbole
\usepackage{graphicx}
\setlength{\parindent}{0px}
\setlength{\mathindent}{10mm}
\usepackage{bbold}                    %brauche ich für die doppel Zahlen Darstellung (Einheitsmatrix z.B)
\usepackage{dsfont}          %F�r den Einheitsoperator \mathds 1


\usepackage{color}
\usepackage{titlesec} %sudo apt-get install texlive-latex-extra

\definecolor{darkblue}{rgb}{0.1,0.1,0.55}
\definecolor{verydarkblue}{rgb}{0.1,0.1,0.35}
\definecolor{darkred}{rgb}{0.55,0.2,0.2}

%hyperref Link color
\usepackage[colorlinks=true,
        linkcolor=darkblue,
        citecolor=darkblue,
        filecolor=darkblue,
        pagecolor=darkblue,
        urlcolor=darkblue,
        bookmarks=true,
        bookmarksopen=true,
        bookmarksopenlevel=3,
        plainpages=false,
        pdfpagelabels=true]{hyperref}

\titleformat{\chapter}[display]{\color{darkred}\normalfont\huge\bfseries}{\chaptertitlename\
\thechapter}{20pt}{\Huge}

\titleformat{\section}{\color{darkblue}\normalfont\Large\bfseries}{\thesection}{1em}{}
\titleformat{\subsection}{\color{verydarkblue}\normalfont\large\bfseries}{\thesubsection}{1em}{}

% Notiz Box
\usepackage{fancybox}
\newcommand{\notiz}[1]{\vspace{5mm}\ovalbox{\begin{minipage}{1\textwidth}#1\end{minipage}}\vspace{5mm}}

\usepackage{cancel}
\setcounter{secnumdepth}{3}
\setcounter{tocdepth}{3}





%-------------------------------------------------------------------------------
%Diff-Makro:
%Das Diff-Makro stellt einen Differentialoperator da.
%
%Benutzung:
% \diff  ->  d
% \diff f  ->  df
% \diff^2 f  ->  d^2 f
% \diff_x  ->  d/dx
% \diff^2_x  ->  d^2/dx^2
% \diff f_x  ->  df/dx
% \diff^2 f_x  ->  d^2f/dx^2
% \diff^2{f(x^5)}_x  ->  d^2(f(x^5))/dx^2
%
%Ersetzt man \diff durch \pdiff, so wird der partieller
%Differentialoperator dargestellt.
%
\makeatletter
\def\diff@n^#1{\@ifnextchar{_}{\diff@n@d^#1}{\diff@n@fun^#1}}
\def\diff@n@d^#1_#2{\frac{\textrm{d}^#1}{\textrm{d}#2^#1}}
\def\diff@n@fun^#1#2{\@ifnextchar{_}{\diff@n@fun@d^#1#2}{\textrm{d}^#1#2}}
\def\diff@n@fun@d^#1#2_#3{\frac{\textrm{d}^#1 #2}{\textrm{d}#3^#1}}
\def\diff@one@d_#1{\frac{\textrm{d}}{\textrm{d}#1}}
\def\diff@one@fun#1{\@ifnextchar{_}{\diff@one@fun@d #1}{\textrm{d}#1}}
\def\diff@one@fun@d#1_#2{\frac{\textrm{d}#1}{\textrm{d}#2}}
\newcommand*{\diff}{\@ifnextchar{^}{\diff@n}
  {\@ifnextchar{_}{\diff@one@d}{\diff@one@fun}}}
%
%Partieller Diff-Operator.
\def\pdiff@n^#1{\@ifnextchar{_}{\pdiff@n@d^#1}{\pdiff@n@fun^#1}}
\def\pdiff@n@d^#1_#2{\frac{\partial^#1}{\partial#2^#1}}
\def\pdiff@n@fun^#1#2{\@ifnextchar{_}{\pdiff@n@fun@d^#1#2}{\partial^#1#2}}
\def\pdiff@n@fun@d^#1#2_#3{\frac{\partial^#1 #2}{\partial#3^#1}}
\def\pdiff@one@d_#1{\frac{\partial}{\partial #1}}
\def\pdiff@one@fun#1{\@ifnextchar{_}{\pdiff@one@fun@d #1}{\partial#1}}
\def\pdiff@one@fun@d#1_#2{\frac{\partial#1}{\partial#2}}
\newcommand*{\pdiff}{\@ifnextchar{^}{\pdiff@n}
  {\@ifnextchar{_}{\pdiff@one@d}{\pdiff@one@fun}}}
\makeatother
%
%Das gleich nur mit etwas andere Syntax. Die Potenz der Differentiation wird erst
%zum Schluss angegeben. Somit lautet die Syntax:
%
% \diff_x^2  ->  d^2/dx^2
% \diff f_x^2  ->  d^2f/dx^2
% \diff{f(x^5)}_x^2  ->  d^2(f(x^5))/dx^2
% Ansonsten wie Oben.
%
%Ersetzt man \diff durch \pdiff, so wird der partieller
%Differentialoperator dargestellt.
%
%\makeatletter
%\def\diff@#1{\@ifnextchar{_}{\diff@fun#1}{\textrm{d} #1}}
%\def\diff@one_#1{\@ifnextchar{^}{\diff@n{#1}}%
%  {\frac{\textrm d}{\textrm{d} #1}}}
%\def\diff@fun#1_#2{\@ifnextchar{^}{\diff@fun@n#1_#2}%
%  {\frac{\textrm d #1}{\textrm{d} #2}}}
%\def\diff@n#1^#2{\frac{\textrm d^#2}{\textrm{d}#1^#2}}
%\def\diff@fun@n#1_#2^#3{\frac{\textrm d^#3 #1}%
%  {\textrm{d}#2^#3}}
%\def\diff{\@ifnextchar{_}{\diff@one}{\diff@}}
%\newcommand*{\diff}{\@ifnextchar{_}{\diff@one}{\diff@}}
%
%Partieller Diff-Operator.
%\def\pdiff@#1{\@ifnextchar{_}{\pdiff@fun#1}{\partial #1}}
%\def\pdiff@one_#1{\@ifnextchar{^}{\pdiff@n{#1}}%
%  {\frac{\partial}{\partial #1}}}
%\def\pdiff@fun#1_#2{\@ifnextchar{^}{\pdiff@fun@n#1_#2}%
%  {\frac{\partial #1}{\partial #2}}}
%\def\pdiff@n#1^#2{\frac{\partial^#2}{\partial #1^#2}}
%\def\pdiff@fun@n#1_#2^#3{\frac{\partial^#3 #1}%
%  {\partial #2^#3}}
%\newcommand*{\pdiff}{\@ifnextchar{_}{\pdiff@one}{\pdiff@}}
%\makeatother

%-------------------------------------------------------------------------------
%%Nützliche Makros um in der Quantenmechanik Bras, Kets und das Skalarprodukt
%%zwischen den beiden darzustellen.
%%Benutzung:
%% \bra{x}  ->    < x |
%% \ket{x}  ->    | x >
%% \braket{x}{y} ->   < x | y >



\newcommand\bra[1]{\left\langle #1 \right|}
\newcommand\ket[1]{\left| #1 \right\rangle}
\newcommand\braket[2]{%
 \left\langle \vphantom{#2} #1%
   \middle|%
   \vphantom{#1} #2\right\rangle}%

%-------------------------------------------------------------------------------
%%Aus dem Buch:
%%Titel:  Latex in Naturwissenschaften und Mathematik
%%Autor:  Herbert Voß
%%Verlag: Franzis Verlag, 2006
%%ISBN:   3772374190, 9783772374197
%%
%%Hier werden drei Makros definiert:\mathllap, \mathclap und \mathrlap, welche
%%analog zu den aus Latex bekannten \rlap und \llap arbeiten, d.h. selbst
%%keinerlei horizontalen Platz benötigen, aber dennoch zentriert zum aktuellen
%%Punkt erscheinen.

\newcommand*\mathllap{\mathstrut\mathpalette\mathllapinternal}
\newcommand*\mathllapinternal[2]{\llap{$\mathsurround=0pt#1{#2}$}}
\newcommand*\clap[1]{\hbox to 0pt{\hss#1\hss}}
\newcommand*\mathclap{\mathpalette\mathclapinternal}
\newcommand*\mathclapinternal[2]{\clap{$\mathsurround=0pt#1{#2}$}}
\newcommand*\mathrlap{\mathpalette\mathrlapinternal}
\newcommand*\mathrlapinternal[2]{\rlap{$\mathsurround=0pt#1{#2}$}}

%%Das Gleiche nur mit \def statt \newcommand.
%\def\mathllap{\mathpalette\mathllapinternal}
%\def\mathllapinternal#1#2{%
%  \llap{$\mathsurround=0pt#1{#2}$}% $
%}
%\def\clap#1{\hbox to 0pt{\hss#1\hss}}
%\def\mathclap{\mathpalette\mathclapinternal}
%\def\mathclapinternal#1#2{%
%  \clap{$\mathsurround=0pt#1{#2}$}%
%}
%\def\mathrlap{\mathpalette\mathrlapinternal}
%\def\mathrlapinternal#1#2{%
%  \rlap{$\mathsurround=0pt#1{#2}$}% $
%}

%-------------------------------------------------------------------------------
%%Hier werden zwei neue Makros definiert \overbr und \underbr welche analog zu
%%\overbrace und \underbrace funktionieren jedoch die Gleichung nicht
%%'zerreißen'. Dies wird ermöglicht durch das \mathclap Makro.

\def\overbr#1^#2{\overbrace{#1}^{\mathclap{#2}}}
\def\underbr#1_#2{\underbrace{#1}_{\mathclap{#2}}}
%\includegraphics[width=0.75\textwidth]{thepic.png}

\begin{document}
\tableofcontents
\setcounter{chapter}{6}
\chapter{Quantisierung des Strahlungsfeldes}

Potentiale:

\[\vec B = \vec v \times \vec A\]

\[\vec E = -\vec \nabla \phi - \frac{\partial \vec A}{\partial t}\]

Eichfreiheit für \(\phi,\vec A\)

Maxwell \(\Rightarrow \)

\begin{align}
\vec \nabla \vec E &= - \vec \nabla^2\phi - \frac{\partial}{\partial t}\vec \nabla\cdot\vec A\\
&= \underbrace{(\frac{1}{c^2}\frac{\partial}{\partial t} - \nabla^2)}_{\square}\phi - \frac{\partial}{\partial t}(\nabla\cdot\vec A + \frac{1}{c^2}\frac{\partial\phi}{\partial t}) = \frac{\rho}{\epsilon_0}
\end{align}

\[\vec \nabla\times\vec B = \mu_0\vec j + \frac{1}{c^2}\frac{\partial\vec E}{\partial t} \Leftrightarrow  \]

\[\square \vec A + \vec\nabla(\vec\nabla\vec A + \frac{1}{c^2} \frac{\partial\phi}{\partial t}) = \mu_0 \vec j\]


Coulomb Eichung: \(\vec \nabla\cdot\vec A = 0\)

\[\Rightarrow \phi(\vec x,t) = \frac{1}{4\pi\epsilon_0}\int d^3\vec x' \frac{\rho(\vec x, t)}{|\vec x - \vec x'|}\]


Zunächst betrachte freier Fall \(\rho =0,\vec j = 0\Rightarrow \phi=0 \)

\[\Rightarrow  \boxed{\square \vec A = 0} \]

vergleiche Klein-Gorddon-Gleichung

Lösung: ebene Wellen

\[\vec A(\vec x,t) = \vec A_0 e^{i(\vec k\vec x - \omega t)}\]


\[\square \vec A = 0 \Leftrightarrow  -\frac{\omega^2}{c^2}+\vec k^2 = 0\Leftrightarrow \omega = c|\vec k| \]


Eichbed: \(\nabla\vec A = 0 \Leftrightarrow i\vec k\vec A_0 = 0 \Rightarrow \vec A_0 \bot \vec k \Rightarrow  \) transversale Wellen

Klassische Energie des Strahlungsfeldes

\[H_{rad} = \frac{1}{2}\int (\epsilon_0\vec E ^2 + \frac{1}{\mu_0}\vec B^2)d^2\vec x\]

in endlichen Kasten mit periodischen Randbedingungen

Forderung:

\[\vec A(x_1 = -\frac{L}{2},x_2,x_3,t) = \vec A(x_1 =+\frac{L}{2},x_2,x_3,t) \]

usw.

Lösungen sind

\[\vec A(\vec x,t) = \vec f_{r,\vec k}(\vec x)e^{-i\omega t} = (\frac{1}{\sqrt{V}}\vec \epsilon_r(k) e^{i\vec k\vec x} )e^{i\omega t} \]
\[\vec k = \frac{2\pi}{L}(n_1,n_2,n_3)\neq 0\qquad n_i\in \mathbb Z\]


Wähle \(\vec \epsilon_r(\vec k)\) so dass \(\vec \epsilon_1, \vec \epsilon_2,\vec \epsilon_3,\hat k = \frac{\vec k}{|\vec k|}\) ein rechtshändiges Orthonormalsystem bilden.

\[\vec \epsilon_r(\vec k)\cdot \vec \epsilon_s(\vec k) = \delta_{rs}\]

\[ \vec \epsilon_r(\vec k)\cdot \vec k = 0 \]

\[  \epsilon_1  \times \epsilon_2 = \hat k \qquad \text{und zyklisch}\]

\(\vec A(\vec x,t)\) darstellbar als Fourierreihe

\[\vec A(\vec x,t) = \sum_{r,\vec k}\sqrt{ \frac{\hbar}{2\epsilon_0 V\omega_k}}\left[ a_r(\vec k) e^{i(\vec k\vec x-\omega t} + a_r^*(\vec k) e^{-i(\vec k\vec x-\omega t}   \right]\vec \epsilon_r(\vec k)  \]

\(\sqrt{ \frac{\hbar}{2\epsilon_0 V\omega_k}}\) Faktor damit \(a_r(\vec k)\) dimensionslos sind. Einheiten: \([\vec A] = [\vec E\cdot\text{Zeit}] 0 \frac{N}{c}s\)

\[[\frac{\hbar}{\epsilon_0 V\omega}] = \frac{Nms}{(\frac{C^2}{Nm^2}m^3 \frac{1}{s}} = \frac{N^2s^2}{C^2}\qquad \checkmark\]


\[\vec A(\vec x,t ) = \sum_{r,\vec k} \sqrt{ \frac{\hbar}{2\epsilon_0 \omega_k}} \left[\vec f_{r,\vec k}(\vec x) a_r(\vec k,t)  + \vec f_{r,-\vec k}(\vec x)  a_r^*(\vec k,t)   \right] \]

mit \( a_r(\vec k,t) = a_r(\vec k)e^{-i\omega_k t}  \)

Im endlichen Kasten (\(V=L^3\))

\[\int_V d^3\vec x \vec f_{r,\vec k}(\vec x) \vec f^*_{r',\vec k'}(\vec x) = \delta_{\vec k,\vec k'}\delta_{rs}\]


\[ H_{rad} \approx \int (\vec E ^2 + c^2\vec B^2)d^2\vec x \]

mit \(\vec E = -\frac{\partial \vec A}{\partial t} = \sum_{r,\vec k}\sqrt{\frac{\hbar}{2\epsilon_0\omega_k}}(-i\omega_k)\left[a_r(\vec k,t)  - \delta_r a^*_r(-\vec k,t) \right]f_{r,\vec k}(\vec x)\)

mit \(\delta_r=\pm 1\)

\[\vec B = \vec \nabla\times\vec A = \sum_{r,\vec k}\sqrt{\frac{\hbar}{2\epsilon_0\omega_k}} \left[a_r(\vec k,t)  + \delta_r a^*_r(-\vec k,t) \right]i\vec k\times f_{r,\vec k}(\vec x)   \]


\begin{align}
\int \vec B^2 d^3\vec x &\approx \int(\underbrace{\vec k\times\vec f_{r,\vec k}(\vec x))(\vec k'\times \vec f^*_{r',\vec k'}(\vec x)}_{\vec k^2\vec f_{r,\vec k}\vec f^*_{r',\vec k'}-\vec k\vec f^*_{r',\vec k'}(\vec x)\vec k'\vec f_{r,\vec k}(x)})d^3\vec x\\
&= \delta_{\vec k,\vec k'}(\vec k^2\underbrace{\epsilon_r(\vec k)\vec \epsilon_{r'}(\vec k)  }_{\delta_{r,r'}}- \underbrace{\vec k\vec\epsilon_{r'}(\vec k)}_{0}\underbrace{\vec k\vec \epsilon_r(\vec k)}_{0}\\
&= |\vec k|^2\delta_{\vec k,\vec k'}\delta_{rr'}
\end{align}


Einsetzen in \(H_{rad}\)

\begin{align}
H_{rad} &= \frac{\epsilon_0}{2}\sum_{r,\vec k}\{ (a_r(\vec k,t) -delta_r a^*_rr(-\vec k,t))(\delta_r a_r(-\vec k,t)-a^*_r(\vec k,t))(-\omega^2_k)\\
&+(a_r(\vec k,t)+\delta_r a_r(-\vec k,t) + a^*_r(\vec k,t) )\underbrace{(c\vec k)^2}_{\omega^2_k}    \}  \\
&=\epsilon_0 \sum_{r,\vec k} \frac{\hbar}{2\epsilon_0\omega_k} \omega_k^2 (a_r(\vec k,t)a^*_rr(\vec k,t)+a^*_rr(+\vec k,t)a^*_rr(+\vec k,t))\\
&= \sum_{r,\vec k} \hbar\omega_k \frac{1}{2}(a_r(\vec k)a^*_r(\vec k) + a^*_r(\vec k)a_r(\vec k))
\end{align}

vergleiche mit Harmonischen Oszillator \(H = \hbar\omega \frac{1}{2}(a^\dagger a+ aa^\dagger)\equiv \hbar\omega(a^\dagger a+\frac{1}{2})\)



\subsection{Spektrum des quantisierten Strahlungsfeldes}

Für jeden Mode \(\vec k, r\)

1 Phonton \(E= \hbar\omega_k\)
2 Phonton \(E= 2\hbar\omega_k\)
\(n(\vec k,r)\) Phonton \(E= \hbar\omega_k n(\vec k,r)\)


Quantisierungsbedingung für das Strahlungsfeld, \(a_r(\vec k) , a^\dagger_r(\vec k)\) als Auf und Absteigeoperatoren:

\[[a_r(\vec k), \vec a^\dagger_s(\vec k')] = \delta_{rs}\delta_{\vec k,\vec k'}\]
\[[a_r(\vec k), \vec a_s(\vec k')] = 0\]

\[H_{rad} = \sum_{r,\vec k} \hbar\omega_k(a_r^\dagger (\underbrace{\vec k)a_r(\vec k)}_{N_r(\vec k)} +\frac{1}{2})\]

Grundzustand : \(|0\rangle \) mit \(a_r(\vec k) |0\rangle  = 0\) hat Energie

\[ E_0 = \frac{1}{2} \sum_{r,\vec k}\hbar\omega_k = \infty\]

Messe Energie relativ zum Grundzustand 

\[H_{rad} = H_{rad}-E_0\]

\[\boxed{H_{rad} = \sum_{r,\vec k} \hbar\omega_k(\underbrace{a_r^\dagger (\vec k)a_r(\vec k)}_{N_r(\vec k)}) } \]

\end{document}