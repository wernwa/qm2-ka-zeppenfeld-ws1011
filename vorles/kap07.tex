\documentclass[10pt,a4paper,oneside,fleqn]{article}
\usepackage{geometry}
\geometry{a4paper,left=20mm,right=20mm,top=1cm,bottom=2cm}
\usepackage[utf8]{inputenc}
%\usepackage{ngerman}
\usepackage{amsmath}                % brauche ich um dir Formel zu umrahmen.
\usepackage{amsfonts}                % brauche ich für die Mengensymbole
\usepackage{graphicx}
\setlength{\parindent}{0px}
\setlength{\mathindent}{10mm}
\usepackage{bbold}                    %brauche ich für die doppel Zahlen Darstellung (Einheitsmatrix z.B)
\usepackage{dsfont}          %F�r den Einheitsoperator \mathds 1


\usepackage{color}
\usepackage{titlesec} %sudo apt-get install texlive-latex-extra

\definecolor{darkblue}{rgb}{0.1,0.1,0.55}
\definecolor{verydarkblue}{rgb}{0.1,0.1,0.35}
\definecolor{darkred}{rgb}{0.55,0.2,0.2}

%hyperref Link color
\usepackage[colorlinks=true,
        linkcolor=darkblue,
        citecolor=darkblue,
        filecolor=darkblue,
        pagecolor=darkblue,
        urlcolor=darkblue,
        bookmarks=true,
        bookmarksopen=true,
        bookmarksopenlevel=3,
        plainpages=false,
        pdfpagelabels=true]{hyperref}

\titleformat{\chapter}[display]{\color{darkred}\normalfont\huge\bfseries}{\chaptertitlename\
\thechapter}{20pt}{\Huge}

\titleformat{\section}{\color{darkblue}\normalfont\Large\bfseries}{\thesection}{1em}{}
\titleformat{\subsection}{\color{verydarkblue}\normalfont\large\bfseries}{\thesubsection}{1em}{}

% Notiz Box
\usepackage{fancybox}
\newcommand{\notiz}[1]{\vspace{5mm}\ovalbox{\begin{minipage}{1\textwidth}#1\end{minipage}}\vspace{5mm}}

\usepackage{cancel}
\setcounter{secnumdepth}{3}
\setcounter{tocdepth}{3}





%-------------------------------------------------------------------------------
%Diff-Makro:
%Das Diff-Makro stellt einen Differentialoperator da.
%
%Benutzung:
% \diff  ->  d
% \diff f  ->  df
% \diff^2 f  ->  d^2 f
% \diff_x  ->  d/dx
% \diff^2_x  ->  d^2/dx^2
% \diff f_x  ->  df/dx
% \diff^2 f_x  ->  d^2f/dx^2
% \diff^2{f(x^5)}_x  ->  d^2(f(x^5))/dx^2
%
%Ersetzt man \diff durch \pdiff, so wird der partieller
%Differentialoperator dargestellt.
%
\makeatletter
\def\diff@n^#1{\@ifnextchar{_}{\diff@n@d^#1}{\diff@n@fun^#1}}
\def\diff@n@d^#1_#2{\frac{\textrm{d}^#1}{\textrm{d}#2^#1}}
\def\diff@n@fun^#1#2{\@ifnextchar{_}{\diff@n@fun@d^#1#2}{\textrm{d}^#1#2}}
\def\diff@n@fun@d^#1#2_#3{\frac{\textrm{d}^#1 #2}{\textrm{d}#3^#1}}
\def\diff@one@d_#1{\frac{\textrm{d}}{\textrm{d}#1}}
\def\diff@one@fun#1{\@ifnextchar{_}{\diff@one@fun@d #1}{\textrm{d}#1}}
\def\diff@one@fun@d#1_#2{\frac{\textrm{d}#1}{\textrm{d}#2}}
\newcommand*{\diff}{\@ifnextchar{^}{\diff@n}
  {\@ifnextchar{_}{\diff@one@d}{\diff@one@fun}}}
%
%Partieller Diff-Operator.
\def\pdiff@n^#1{\@ifnextchar{_}{\pdiff@n@d^#1}{\pdiff@n@fun^#1}}
\def\pdiff@n@d^#1_#2{\frac{\partial^#1}{\partial#2^#1}}
\def\pdiff@n@fun^#1#2{\@ifnextchar{_}{\pdiff@n@fun@d^#1#2}{\partial^#1#2}}
\def\pdiff@n@fun@d^#1#2_#3{\frac{\partial^#1 #2}{\partial#3^#1}}
\def\pdiff@one@d_#1{\frac{\partial}{\partial #1}}
\def\pdiff@one@fun#1{\@ifnextchar{_}{\pdiff@one@fun@d #1}{\partial#1}}
\def\pdiff@one@fun@d#1_#2{\frac{\partial#1}{\partial#2}}
\newcommand*{\pdiff}{\@ifnextchar{^}{\pdiff@n}
  {\@ifnextchar{_}{\pdiff@one@d}{\pdiff@one@fun}}}
\makeatother
%
%Das gleich nur mit etwas andere Syntax. Die Potenz der Differentiation wird erst
%zum Schluss angegeben. Somit lautet die Syntax:
%
% \diff_x^2  ->  d^2/dx^2
% \diff f_x^2  ->  d^2f/dx^2
% \diff{f(x^5)}_x^2  ->  d^2(f(x^5))/dx^2
% Ansonsten wie Oben.
%
%Ersetzt man \diff durch \pdiff, so wird der partieller
%Differentialoperator dargestellt.
%
%\makeatletter
%\def\diff@#1{\@ifnextchar{_}{\diff@fun#1}{\textrm{d} #1}}
%\def\diff@one_#1{\@ifnextchar{^}{\diff@n{#1}}%
%  {\frac{\textrm d}{\textrm{d} #1}}}
%\def\diff@fun#1_#2{\@ifnextchar{^}{\diff@fun@n#1_#2}%
%  {\frac{\textrm d #1}{\textrm{d} #2}}}
%\def\diff@n#1^#2{\frac{\textrm d^#2}{\textrm{d}#1^#2}}
%\def\diff@fun@n#1_#2^#3{\frac{\textrm d^#3 #1}%
%  {\textrm{d}#2^#3}}
%\def\diff{\@ifnextchar{_}{\diff@one}{\diff@}}
%\newcommand*{\diff}{\@ifnextchar{_}{\diff@one}{\diff@}}
%
%Partieller Diff-Operator.
%\def\pdiff@#1{\@ifnextchar{_}{\pdiff@fun#1}{\partial #1}}
%\def\pdiff@one_#1{\@ifnextchar{^}{\pdiff@n{#1}}%
%  {\frac{\partial}{\partial #1}}}
%\def\pdiff@fun#1_#2{\@ifnextchar{^}{\pdiff@fun@n#1_#2}%
%  {\frac{\partial #1}{\partial #2}}}
%\def\pdiff@n#1^#2{\frac{\partial^#2}{\partial #1^#2}}
%\def\pdiff@fun@n#1_#2^#3{\frac{\partial^#3 #1}%
%  {\partial #2^#3}}
%\newcommand*{\pdiff}{\@ifnextchar{_}{\pdiff@one}{\pdiff@}}
%\makeatother

%-------------------------------------------------------------------------------
%%Nützliche Makros um in der Quantenmechanik Bras, Kets und das Skalarprodukt
%%zwischen den beiden darzustellen.
%%Benutzung:
%% \bra{x}  ->    < x |
%% \ket{x}  ->    | x >
%% \braket{x}{y} ->   < x | y >



\newcommand\bra[1]{\left\langle #1 \right|}
\newcommand\ket[1]{\left| #1 \right\rangle}
\newcommand\braket[2]{%
 \left\langle \vphantom{#2} #1%
   \middle|%
   \vphantom{#1} #2\right\rangle}%

%-------------------------------------------------------------------------------
%%Aus dem Buch:
%%Titel:  Latex in Naturwissenschaften und Mathematik
%%Autor:  Herbert Voß
%%Verlag: Franzis Verlag, 2006
%%ISBN:   3772374190, 9783772374197
%%
%%Hier werden drei Makros definiert:\mathllap, \mathclap und \mathrlap, welche
%%analog zu den aus Latex bekannten \rlap und \llap arbeiten, d.h. selbst
%%keinerlei horizontalen Platz benötigen, aber dennoch zentriert zum aktuellen
%%Punkt erscheinen.

\newcommand*\mathllap{\mathstrut\mathpalette\mathllapinternal}
\newcommand*\mathllapinternal[2]{\llap{$\mathsurround=0pt#1{#2}$}}
\newcommand*\clap[1]{\hbox to 0pt{\hss#1\hss}}
\newcommand*\mathclap{\mathpalette\mathclapinternal}
\newcommand*\mathclapinternal[2]{\clap{$\mathsurround=0pt#1{#2}$}}
\newcommand*\mathrlap{\mathpalette\mathrlapinternal}
\newcommand*\mathrlapinternal[2]{\rlap{$\mathsurround=0pt#1{#2}$}}

%%Das Gleiche nur mit \def statt \newcommand.
%\def\mathllap{\mathpalette\mathllapinternal}
%\def\mathllapinternal#1#2{%
%  \llap{$\mathsurround=0pt#1{#2}$}% $
%}
%\def\clap#1{\hbox to 0pt{\hss#1\hss}}
%\def\mathclap{\mathpalette\mathclapinternal}
%\def\mathclapinternal#1#2{%
%  \clap{$\mathsurround=0pt#1{#2}$}%
%}
%\def\mathrlap{\mathpalette\mathrlapinternal}
%\def\mathrlapinternal#1#2{%
%  \rlap{$\mathsurround=0pt#1{#2}$}% $
%}

%-------------------------------------------------------------------------------
%%Hier werden zwei neue Makros definiert \overbr und \underbr welche analog zu
%%\overbrace und \underbrace funktionieren jedoch die Gleichung nicht
%%'zerreißen'. Dies wird ermöglicht durch das \mathclap Makro.

\def\overbr#1^#2{\overbrace{#1}^{\mathclap{#2}}}
\def\underbr#1_#2{\underbrace{#1}_{\mathclap{#2}}}
%\includegraphics[width=0.75\textwidth]{thepic.png}

\begin{document}
\tableofcontents
\setcounter{chapter}{6}
\chapter{Quantisierung des Strahlungsfeldes}

Wiederholung (siehe  http://de.wikipedia.org/wiki/Maxwell-Gleichungen) :\\
\\
\begin{tabular}{llp{7cm}}
  \textbf{Name}&\textbf{SI}&\textbf{Physikalischer Bedeutung}\\
\\
Gaußsches Gesetz & \( \vec \nabla\cdot\vec E = \frac{\rho}{\epsilon_0}  \)& Elektrische Feldlinien divergieren voneinander unter Anwesenheit elektrischer Ladung, die Ladung ist Quelle des elektrischen Feldes.\\
\\
Gaußsches Gesetz für Magnetfelder & \( \vec \nabla\cdot\vec B = 0  \) &Magnetische Feldlinien divergieren nicht, das magnetische Feld ist quellenfrei; es gibt keine magnetischen Monopole.\\
\\
Induktionsgesetz von Faraday& \(\vec\nabla\times\vec E = - \frac{\partial \vec B}{\partial t} \)&Änderungen der magnetischen Flussdichte führt zu einem elektrischen Wirbelfeld.\\
\\
Erweitertes ampèresches Gesetz& \( \vec\nabla\times\vec B = \mu_0\vec j + \mu_0\epsilon_0 \frac{\partial \vec E}{\partial t}  \)&Elektrische Ströme – einschließlich des Verschiebungsstroms – führen zu einem magnetischen Wirbelfeld.
\end{tabular}\\
\\
Potentiale:

\[\vec B = \vec \nabla \times \vec A\]

\[\vec E = -\vec \nabla \phi - \frac{\partial \vec A}{\partial t}\]

Eichfreiheit für \(\phi,\vec A\)

Maxwell \(\Rightarrow \)

\begin{align}
\vec \nabla \vec E &= - \vec \nabla^2\phi - \frac{\partial}{\partial t}\vec \nabla\cdot\vec A\\
&= \underbrace{(\frac{1}{c^2}\frac{\partial^2}{\partial t^2} - \nabla^2)}_{\square}\phi - \frac{\partial}{\partial t}(\nabla\cdot\vec A + \frac{1}{c^2}\frac{\partial\phi}{\partial t}) = \frac{\rho}{\epsilon_0}
\end{align}

\[\vec \nabla\times\vec B = \mu_0\vec j + \frac{1}{c^2}\frac{\partial\vec E}{\partial t} \Leftrightarrow  \]

\[\square \vec A + \vec\nabla(\vec\nabla\vec A + \frac{1}{c^2} \frac{\partial\phi}{\partial t}) = \mu_0 \vec j\]


Coulomb Eichung: \(\vec \nabla\cdot\vec A = 0\)

\[\Rightarrow \phi(\vec x,t) = \frac{1}{4\pi\epsilon_0}\int d^3\vec x' \frac{\rho(\vec x, t)}{|\vec x - \vec x'|}\]


Zunächst betrachte freier Fall \(\rho =0,\vec j = 0\Rightarrow \phi=0 \)

\[\Rightarrow  \boxed{\square \vec A = 0} \]

vergleiche Klein-Gorddon-Gleichung \( \left(\square +\left(\frac{mc}{\hbar}\right)^2\right)\psi = 0 \)

Lösung: ebene Wellen

\[\vec A(\vec x,t) = \vec A_0 e^{i(\vec k\vec x - \omega t)}\]


\[\square \vec A = 0 \Leftrightarrow  -\frac{\omega^2}{c^2}+\vec k^2 = 0\Leftrightarrow \omega = c|\vec k| \]


Eichbedingung: \(\nabla\vec A = 0 \Leftrightarrow i\vec k\vec A_0 = 0 \Rightarrow \vec A_0 \bot \vec k \Rightarrow  \) transversale Wellen

Klassische Energie des Strahlungsfeldes

\[H_{rad} = \frac{1}{2}\int (\epsilon_0\vec E ^2 + \frac{1}{\mu_0}\vec B^2)d^3\vec x\]

in einem endlichen Kasten mit periodischen Randbedingungen

Forderung:

\[\vec A(x_1 = -\frac{L}{2},x_2,x_3,t) = \vec A(x_1 =+\frac{L}{2},x_2,x_3,t) \]

usw.

Lösungen sind

\[\vec A(\vec x,t) = \vec f_{r,\vec k}(\vec x)e^{-i\omega t} = \underbrace{\frac{1}{\sqrt{V}}\vec \epsilon_r(k) e^{i\vec k\vec x} }_{\vec f_{r,\vec k}(\vec x) }e^{-i\omega t} \]
\[\vec k = \frac{2\pi}{L}(n_1,n_2,n_3)\neq 0\qquad n_i\in \mathbb Z\]


Wähle \(\vec \epsilon_r(\vec k)\) so dass \(\vec \epsilon_1, \vec \epsilon_2,\vec \epsilon_3,\hat k = \frac{\vec k}{|\vec k|}\) ein rechtshändiges Orthonormalsystem bilden.

\[\vec \epsilon_r(\vec k)\cdot \vec \epsilon_s(\vec k) = \delta_{rs}\]

\[ \vec \epsilon_r(\vec k)\cdot \vec k = 0 \]

\[  \epsilon_1  \times \epsilon_2 = \hat k \qquad \text{und zyklisch}\]

\(\vec A(\vec x,t)\) darstellbar als Fourierreihe

\[\vec A(\vec x,t) = \sum_{r,\vec k}\sqrt{ \frac{\hbar}{2\epsilon_0 V\omega_k}}\left[ a_r(\vec k) e^{i(\vec k\vec x-\omega t)} + a_r^*(\vec k) e^{-i(\vec k\vec x-\omega t)}   \right]\vec \epsilon_r(\vec k)  \]

\(\sqrt{ \frac{\hbar}{2\epsilon_0 V\omega_k}}\) Faktor damit \(a_r(\vec k)\) dimensionslos sind. Einheiten: \([\vec A] = [\vec E\cdot t] = \frac{N}{C}s\)

\[[\frac{\hbar}{\epsilon_0 V\omega}] = \frac{Nms}{\frac{C^2}{Nm^2}m^3 \frac{1}{s}} = \frac{N^2s^2}{C^2}\qquad \checkmark\]


\[\vec A(\vec x,t ) = \sum_{r,\vec k} \sqrt{ \frac{\hbar}{2\epsilon_0 \omega_k}} \left[\vec f_{r,\vec k}(\vec x) a_r(\vec k,t)  + \vec f_{r,-\vec k}(\vec x)  a_r^*(\vec k,t)   \right] \]

mit \( a_r(\vec k,t) = a_r(\vec k)e^{-i\omega_k t}  \)

Im endlichen Kasten (\(V=L^3\))

\[\int_V d^3\vec x \vec f_{r,\vec k}(\vec x) \vec f^*_{r',\vec k'}(\vec x) = \delta_{\vec k,\vec k'}\delta_{rs}\]


\[ H_{rad} \approx \int (\vec E ^2 + c^2\vec B^2)d^2\vec x \]

mit \(\vec E = -\frac{\partial \vec A}{\partial t} = \sum_{r,\vec k}\sqrt{\frac{\hbar}{2\epsilon_0\omega_k}}(-i\omega_k)\left[a_r(\vec k,t)  - \delta_r a^*_r(-\vec k,t) \right]f_{r,\vec k}(\vec x)\)

mit \(\delta_r=\pm 1\)

\[\vec B = \vec \nabla\times\vec A = \sum_{r,\vec k}\sqrt{\frac{\hbar}{2\epsilon_0\omega_k}} \left[a_r(\vec k,t)  + \delta_r a^*_r(-\vec k,t) \right]i\vec k\times f_{r,\vec k}(\vec x)   \]


\begin{align}
\int \vec B^2 d^3\vec x &\approx \int(\underbrace{\vec k\times\vec f_{r,\vec k}(\vec x))(\vec k'\times \vec f^*_{r',\vec k'}(\vec x)}_{\vec k^2\vec f_{r,\vec k}\vec f^*_{r',\vec k'}-\vec k\vec f^*_{r',\vec k'}(\vec x)\vec k'\vec f_{r,\vec k}(x)})d^3\vec x\\
&= \delta_{\vec k,\vec k'}(\vec k^2\underbrace{\epsilon_r(\vec k)\vec \epsilon_{r'}(\vec k)  }_{\delta_{r,r'}}- \underbrace{\vec k\vec\epsilon_{r'}(\vec k)}_{0}\underbrace{\vec k\vec \epsilon_r(\vec k)}_{=0 \quad \text{da}\quad \vec k\bot\vec\epsilon_r}\\
&= |\vec k|^2\delta_{\vec k,\vec k'}\delta_{rr'}
\end{align}



Einsetzen in \(H_{rad}\)

\begin{align}
H_{rad} &= \frac{\epsilon_0}{2}\sum_{r,\vec k}\{ (a_r(\vec k,t) -\delta_r a^*_rr(-\vec k,t))(\delta_r a_r(-\vec k,t)-a^*_r(\vec k,t))(-\omega^2_k)\\
&+(a_r(\vec k,t)+\delta_r a_r(-\vec k,t) + a^*_r(\vec k,t) )\underbrace{(c\vec k)^2}_{\omega^2_k}    \}  \\
&=\epsilon_0 \sum_{r,\vec k} \frac{\hbar}{2\epsilon_0\omega_k} \omega_k^2 (a_r(\vec k,t)a^*_rr(\vec k,t)+a^*_rr(+\vec k,t)a^*_rr(+\vec k,t))\\
&= \sum_{r,\vec k} \hbar\omega_k \frac{1}{2}(a_r(\vec k)a^*_r(\vec k) + a^*_r(\vec k)a_r(\vec k))
\end{align}

vergleiche mit Harmonischen Oszillator \(H = \hbar\omega \frac{1}{2}(a^\dagger a+ aa^\dagger)\equiv \hbar\omega(a^\dagger a+\frac{1}{2})\)



\subsection{Spektrum des quantisierten Strahlungsfeldes}

Für jeden Mode \(\vec k, r\)\\
\\
1 Phonton \(E= \hbar\omega_k\)\\
2 Phonton \(E= 2\hbar\omega_k\)\\
\(n(\vec k,r)\) Phonton \(E= \hbar\omega_k n(\vec k,r)\)\\
\\

Quantisierungsbedingung für das Strahlungsfeld, \(a_r(\vec k) , a^\dagger_r(\vec k)\) als Auf und Absteigeoperatoren:

\[[a_r(\vec k), \vec a^\dagger_s(\vec k')] = \delta_{rs}\delta_{\vec k,\vec k'}\]
\[[a_r(\vec k), \vec a_s(\vec k')] = 0\]

\[H_{rad} = \sum_{r,\vec k} \hbar\omega_k(\underbrace{a_r^\dagger (\vec k)a_r(\vec k)}_{N_r(\vec k)} +\frac{1}{2})\]

Grundzustand : \(|0\rangle \) mit \(a_r(\vec k) |0\rangle  = 0\) hat Energie

\[ E_0 = \frac{1}{2} \sum_{r,\vec k}\hbar\omega_k = \infty\]

Messe Energie relativ zum Grundzustand 

\[H_{rad} = H_{rad}-E_0\]

\[\boxed{H_{rad} = \sum_{r,\vec k} \hbar\omega_k(\underbrace{a_r^\dagger (\vec k)a_r(\vec k)}_{N_r(\vec k)}) } \]

Harmonischer Oszillator für jeden Mode \((\vec k,r)\)

\[\vec p_{rad} = \sum_{\vec k,r} \hbar\vec k N_r(\vec k)\]

mit Teilchenzahloperator \(N(\vec k) = a^\dagger_r(\vec k)a_r(\vec k)\) mit Mode \((\vec k,r)\) Eigenzustände von \(N_r(\vec k)\) (und \(H_{rad},\vec p_{rad}\))

\[|n_r(\vec k)=n\rangle  = \frac{(a^\dagger_r(\vec k))^n}{\sqrt{n!}}|0\rangle \]


Allgemeiner Zustand: Produkt über Moden (\underline{Fock Raum} für Photonen):

\[|... n_r(\vec k)...\rangle  = \prod_{\vec k,r}|n_r(\vec k)\rangle \in \mathcal H_{rad} = \otimes_{\vec k,r}\mathcal H_{\vec k,r}^{H0}\]

Grundzustand: \(|0\rangle \) def. durch \(a_r(\vec k)|0\rangle =0\)

\begin{itemize}
\item 1 Photon Zustand: \(a^\dagger_r(\vec k) |0\rangle = |\vec k,r\rangle  \)
\item 2 Photon Zustand: \(a^\dagger_{r1} (\vec k_1) a^\dagger_{r2} (\vec k_2)   |0\rangle = |(\vec k_1,r_1),(\vec k_2,r_2)\rangle   \)
\end{itemize}

sind Eigenzustände zu \(H_{rad},\vec P_{rad}\)

\[ H_{rad}a^\dagger_r(\vec k) |E,\vec P\rangle  =  [H_{rad},a^\dagger_r]+a^\dagger_r(\vec k)\underbrace{H_{rad}}_{E}|E,\vec p\rangle \]


mit \([H_{rad},a^\dagger_r] = \hbar\omega_k[a^\dagger_r(\vec k) a_r(\vec k),a^\dagger_r(\vec k)] =\hbar\omega_k a^\dagger_r(\vec k)  \)

\[ H_{rad}a^\dagger_r(\vec k) |E,\vec P\rangle  = (\hbar\omega_k+E) a^\dagger_r(\vec k)|E,\vec P\rangle  \]

\( a^\dagger_r(\vec k) \) erzeugt extra Photon im Zustand mit Energie \(\hbar\omega_k\) und Impuls \(\hbar \vec k\)

\(\Rightarrow  a^\dagger_r(\vec k) \) ist Erzeugungsoperator für 1 Photon

\(\Rightarrow  a_r(\vec k) \) ist Vernichtungsoperator für 1 Photon\\
\\

Allgemeiner 1 Photon zustand:

\[|1\rangle  = \sum_{\vec k,r}\psi(\vec k,r)a^\dagger_r(\vec k)|0\rangle \]

Bose Symmetrie (Fall n=2, gilt für alle n)


\[|(\vec k_1,r_1),(\vec k_2,r_2)\rangle = a^\dagger_{r1}(\vec k_1) a^\dagger_{r2} |0\rangle =  a^\dagger_{r2}(\vec k_2) a^\dagger_{r1}(\vec k_1) |0\rangle   \]

\[= |(\vec k_2,r_2),(\vec k_1,r_1)\rangle    \]

ist automatisch Bose-symmetrisch!


Vektor potential des quantisierten Feldes:

\[\vec A(\vec x,t) = \vec A^+(\vec x,t) +\vec A^-(\vec x,t)  \]


ist linear in erzeugern und Vernichtern:

\[\vec A^-(\vec x,t) =\vec \sum_{\vec k,r} \sqrt{\frac{\hbar}{2\epsilon_0 V\omega_k}}\vec \epsilon_r(\vec k)a^\dagger_r(\vec k)e^{-i(\vec k\vec x-\omega_k t)}\]
\[\vec A^+(\vec x,t) =\vec \sum_{\vec k,r} \sqrt{\frac{\hbar}{2\epsilon_0 V\omega_k}}\vec \epsilon_r(\vec k)a_r(\vec k)e^{+i(\vec k\vec x-\omega_k t)}\]

ist linear in erzeugern und Vernichtern:

\[\Rightarrow \vec E = -\frac{\partial \vec A}{\partial t};\quad \vec B = \vec\nabla\times\vec A\]

\(\vec E\) und \(\vec B\) sind Operatoren auf dem Fock Raum.

\(\vec x,t\) sind Parameter der Felder \(\vec E(\vec x,t),\vec A(\vec x,t)...\) (sind keine Operatoren)


Klassischer Limes

\[\vec E_{\text{Klassisch}}(\vec x,t) = \langle A|\vec E(\vec x,t)|A\rangle \qquad |A\rangle \in \mathcal H_{rad}\]

\(A\) darf keine feste Anzahl (=n) Photoonen haben. 

\[|B\rangle  = |n \text{ Photonen}\rangle  \Rightarrow \underbrace{\vec A(\vec x,t)}_{\vec A^++\vec A^-}|n \text{ Photonen}\rangle = \#|n-1\rangle +  \#|n+1\rangle  \]


\[\langle n| \vec A(\vec x,t)|n\rangle \]


Zustände mit vorgegebener fester Photonenzahl \(\cancel \Leftrightarrow \) Klassischer Zustände. Klassischer Limes entspricht den Kohärenten Zuständen. 


\section{Wechselwirkung Strahlung in Materie}


Materie: \(N\) Punktteilchen, Masse \(m_i\), Ladung \(e_i\). 

\[\tilde H_m = \sum_i^N \frac{1}{2}m_i (\frac{d\vec r_i}{dt})^2 + H_C\]

mit Coulomb WW.

\[H_C = \frac{1}{2}\sum_{i\neq j} \frac{e_ie_j}{4\pi\epsilon_0|\vec r_i-\vec r_j|}\]

mit der Beziehung: \(\sum_{i,j;i\neq j} = 2\sum_{i,j;i<j}\)

Strahlungsfeld in Coulomb Eichung \(\vec \nabla\cdot\vec A = 0\)

aber:

\[ \vec \nabla\cdot\vec E = \frac{\rho (\vec x,t)}{\epsilon} = \frac{1}{\epsilon_0}\sum_{i=1}^N e_i\delta^3(\vec x-\vec r_i(t))\neq 0\]

\(\Rightarrow \vec E\) hat longitudinale Komponente \(\vec E_L\)

\[\vec E = \vec E_L + \vec E_T = -\frac{\partial \vec A}{\partial t}-\vec \nabla\phi\qquad\text{mit }\vec\nabla\times\vec E_L = 0,\quad\vec\nabla\cdot\vec E_T = 0\]


Es gilt \(\boxed{\vec E_L = -\vec \nabla\phi,\quad \vec E_T = -\frac{\partial \vec A}{\partial t}} \) wegen \(\vec \nabla\cdot\vec E-T = -\frac{\partial }{\partial t}\vec \nabla\cdot\vec A \)

Energie des e.m.Feldes

\begin{align}
H_{e.m.} &= \frac{\epsilon_0}{2}\int d^3\vec x(\vec E^2+c^2\vec B^2)\\
&=\underbrace{ \frac{\epsilon_0}{2}\int d^3\vec x(\vec E^2_T+c^2\vec B^2 }_{H_{rad}} + \vec E_L^2+\cancel{2\vec E_L\cdot\vec E_T})
\end{align}

NR: \(\vec E_L\cdot\vec E_T = \vec\nabla\phi\frac{\partial\vec A}{\partial t} = \vec\nabla\cdot(\phi\frac{\partial\vec A}{\partial t}) - \phi\frac{\partial}{\partial t}\underbrace{\vec \nabla\vec A}_{=0} \rightarrow  \) Oberflächenterm \(\rightarrow 0\)
\begin{align} 
\frac{\epsilon_0}{2} \int \underbrace{ \vec E^2_L}_{(-\vec\nabla\phi)^2}d^3\vec x &= -\frac{\epsilon_0}{2} \int\phi\underbrace{\nabla^2\phi}_{-\frac{\rho(\vec x,t)}{\epsilon_0}}d^3\vec x\\
&= \frac{1}{2}\int d^3\vec x \rho(\vec x,t) \frac{\rho(\vec x',t)}{4\pi\epsilon_0|\vec x-\vec x'|} \\
&=  \frac{1}{2} \sum_{i,j=1}^N \frac{e_ie_j}{4\pi\epsilon_0|\vec r_i(t)-\vec r_j(t)|}\\
&= H_C+E_S
\end{align}


(\(E_S\) = Selbstenergie der N Teilchen)

\(H_C\) schon in \(\tilde H_m\) enthalten

Korrekter Kanonischer Impuls für \(\tilde H_m\)

\[\vec p_i = m_i\frac{d\vec r_i}{dt}+e_i\vec A(\vec r_i(t),t)\]

\[\Rightarrow \tilde H_m = \sum_i\frac{1}{2m_i}(\vec p_i-e_i\underbrace{\vec A(\vec r_i)}_{\vec A_i})^2+H_C\]


\[\tilde H_m = \underbrace{\sum_i \frac{\vec p_i^2}{2m_i}+H_C}_{H_m} + \underbrace{\sum_i \frac{1}{2m_i}(-e_i\vec p_i A_i-e_i\vec A_ip_i+e^2_iA^2_i)}_{H_I}\]

in Coulomb Eichung gilt: \(\vec p_i\cdot\vec A_i = \vec A_i\cdot\vec p_i\) wegen \(\vec\nabla\vec A = 0\)

\[H_I = \sum_{i=1}^N \{ i\frac{e_i\hbar}{m_i}\vec A(\vec r_i,t)\cdot\vec\nabla_i + \frac{ e^2_i}{2m_i}\vec A^2(\vec r_i,t) \}\]

\(\Rightarrow \) Störungentwicklung für

\[H = \underbrace{H_{rad}+H_m}_{H_0} + H_I\]


\section{Materie + Strahlung}

\[H = \underbrace{H_m+H_{rad}}_{H_0}+H_I\]

\[H_I = \sum_{i=1}^{N}(-\frac{e_i}{m_i}\vec A(\vec r_i,t)\vec p_i + \frac{e_i^2}{2m}\vec A^2)\]

\[\vec A(\vec r,t) =\sum_{\vec k,r}\sqrt{\frac{\hbar}{2\epsilon_0V\omega_k}}\vec\epsilon_r(\vec k)[a^\dagger_r(\vec k) e^{-i(\vec k\vec x-\omega t)}+a_r(\vec k) e^{i(\vec k\vec x-\omega t)} ]\]

ungestörter Zustand

\[|A;...n_r(\vec k) ...\rangle  = |A\rangle |...n_r(\vec k)...\rangle \]


\(|A\rangle \) Eigenzustand von \(H_m\)

\[H_m|A\rangle  = E_A|A\rangle \]

\underline{Emission} von einem Photon:

\[A\rightarrow B+\gamma\]

Vergleiche mit Fermi-Goldener-Regel:

\[ w_{i \to n} = \frac{2\pi}{\hbar}|V_{ni}|^2\delta(E_n-E_i)  \]

Übergangswahrscheinlichkeit/Zeit 

\[p_r(\vec k) = \frac{2\pi}{\hbar}|\langle B;...n_r(\vec k)+1...|H_I|A;...n_r(\vec k)...\rangle |^2\cdot\delta(E_A-E_B-\hbar\omega_k)\]

Zerfallswarcheinlichkeit/Zeit

\[w_r = \sum_{\vec k}p_r(\vec k) = \frac{V}{(2\pi)^3}\sum_{\vec k}\underbrace{(\frac{2\pi}{L})^3}_{\Delta^3k}p_r(\vec k)\rightarrow \frac{V}{(2\pi)^3}\int d^3\vec k p_r(\vec k)\]

Matrixelement:

\[\langle B;n_r(\vec k) +1 |H_I|A;n_r(\vec k)\rangle = -\sum_i \frac{e_i}{m_i}\sqrt{\frac{\hbar}{2\epsilon_o V\omega_k}}\underbrace{\langle n_r(\vec k) +1 |a^\dagger_r(\vec k)|n_r(\vec k)\rangle }_{\sqrt{n_r(\vec k)+1}} \langle B|e^{-i\vec k\cdot\vec r_i}\epsilon_r(\vec k)\cdot \vec p_i|A\rangle \]


\begin{itemize}
\item \(V_{ni}\propto \sqrt{n+1} \rightarrow w_r\propto (n+1)\)
\(\rightarrow \) Stimulierte Emission für große n : Laser

\item Emission auch für \(n=0\). Aber klassisch \(\vec E = 0=\vec B \Rightarrow \vec A=0\Rightarrow H_I=0\). spontane Emission ist rein QM.

\end{itemize}


Dipolapproximation für \(\langle B|e^{-ikr}...|A\rangle \):

\[e^{-i\vec k\vec r_i}\approx 1-i\vec k\vec r_i\approx 1\]

und den Impulsoperator können wir schreiben als:

\[\vec p_i = \frac{m_i}{i\hbar} [\vec x_i,\frac{\vec p_i^2}{2m_i}] = \frac{m_i}{i\hbar} [\vec x_i,H_m]  \]


\begin{align}
\Rightarrow &\langle B|\sum_i \underbrace{e^{-i\vec k\vec r_i}}_{\approx 1} \frac{e_i}{m_i}\vec p_i|A\rangle \\
&=\frac{1}{i\hbar} \langle B|[\sum_i e_i \vec x_i,H_m] |A\rangle \\
&=\frac{1}{i\hbar} (E_A-E_B) \langle B|\sum_i e_i\vec x_i|A\rangle\\
&=\frac{1}{i\hbar} (E_A-E_B) e\vec X_{AB}\\
&= (-im\omega \vec X_{AB})\frac{e}{m}
\end{align}


mit \(\omega = \omega_k = \frac{E_A-E_B}{\hbar}\) und \(m\)=Elektronenmasse


\(\Rightarrow \) Zerfallswarhscheinlichkeit/Zeit


\begin{align}
\langle B;n_r(\vec k) +1 |H_I|A;n_r(\vec k)\rangle &= \sqrt{\frac{\hbar}{\epsilon_0}}\sqrt{\frac{(2\pi)^3}{V}}\frac{1}{\sqrt{(2\pi)^32\omega_k}}\sqrt{n_r(k)+1}\frac{e}{m}(-im\omega\vec X_{AB}\vec \epsilon_r(\vec k) \\
\end{align}


\begin{align}
\Rightarrow w_r &= \sum_{\vec k}\frac{(2\pi)^3}{V} \frac{\hbar}{\epsilon_0}\frac{1}{(2\pi)^32\omega_k}(n_r(\vec k)+1)\frac{2\pi}{\hbar}\delta(\hbar\omega-E_A+E_B)\frac{e}{m^2} |-im\omega\vec\epsilon_r(\vec k)\vec X_{AB}|^2\\
&=\int d^3\vec k\frac{e^2}{4\pi\epsilon_0\hbar c}\frac{\hbar c}{\pi m^32\omega_k}(n_r(\vec k)+1)|-im\omega\vec\epsilon_r(\vec k)\vec X_{AB}|^2 \delta(\hbar\omega-E_A+E_B)
\end{align}

NR: \(d^3\vec k = d\Omega \frac{(ck)^2 d(ck)}{c^3} = d\Omega \frac{\omega^2}{\hbar^3}d(\hbar\omega)\)



\[w_r = \int d\Omega \alpha \frac{\omega}{m^2c^2 2\pi}(n_r(k) +1) m^2\omega^2|\epsilon_r(\vec k)\vec X_{AB}|^2\]

\[\boxed{w_r = \int d\Omega \alpha \frac{\omega^3}{2\pi c^2} |\epsilon_r(\vec k)\vec X_{AB}|^2 (n_r(k) +1)}\]

Spontane Emission \(n_r(k) = 0\)


Keine Polarisation : \(\sum_{r=1,2}\)

\begin{align}
&\int d\Omega \sum_r|\vec X\vec \epsilon|^2\\
&=\int d\Omega \vec X^*(\underbrace{\vec \epsilon_1\vec \epsilon_1\vec X+\vec \epsilon_2\vec \epsilon_2\vec X +\hat K\hat K\vec X}_{\vec X}-\hat K\hat K\vec X)\\
&=\int d\Omega (|\vec X|^2-\underbrace{|\vec X\hat K|^2}_{|\vec X|cos^2\theta} = |\vec X|^2 \underbrace{\int_0^{2\pi}d\phi}_{4\pi}\underbrace{2\int_0^1dcos\phi(1-cos^2\theta)}_{\frac{2}{3}}\\
&=\frac{2\pi}{3}|\vec X|^2
\end{align}

\[w_{A\to B}= \alpha \frac{\omega^3}{2\pi c^2}\frac{8\pi}{3}|\vec X_{AB}|^2\]

\[\boxed{w_{A\to B}= \alpha\frac{4}{3} \frac{\omega^3}{c^2}|\vec X_{AB}|^2}\]

Lebensdauer des Zustands \(|A\rangle \)

\[\boxed{\frac{1}{\tau_A} = \sum_B w_{A\to B}}\]


\section{Planck'sche Strahlungsformel für Schwarzkörperstrahlung}


Übergang \(A \rightarrow  B+\gamma\) im thermischen Gleichgewicht. 

Anzahl der Atome in Level A \(N(A)\propto e^{-E_A/kT}\)

Anzahl der Atome in Level B \(N(B)\propto e^{-E_B/kT}\)

\[\frac{N(B)}{N(A)} = \frac{e^{-E_B/kT}}{e^{-E_A/kT}} = e^{\frac{\hbar \omega}{kT}}\]

Gleichgewichtsbedingung:

\[N(B) w_{abs.} = N(A)w_{emmiss.}\]


\[w_{emmis.} = \alpha \frac{\omega}{2\pi m^2c^2}|\langle B|\sum e^{-i\vec K\vec r_i}\vec \epsilon_r\vec p_i|A\rangle |^2(n_r(\vec k)+1)\]

\[w_{abs} = \alpha \frac{\omega}{2\pi m^2c^2}|\langle A|\sum e^{i\vec K\vec r_i}\vec \epsilon_r\vec p_i|B\rangle |^2n_r(\vec k)\]

Das Matrixelement von \(w_{abs}\) ist genau das komlex konjugierte von \(w_{emmis.}\)

\[\Rightarrow \frac{w_{emiss}}{w_{abs}} = \frac{n_r(k)+1}{n_r(k)} = 1+\frac{1}{n_r(\vec k)}\]

\[\frac{N(B)}{N(A)} = e^{\frac{\hbar\omega}{kT}}\]



\[\Rightarrow \boxed{n_r(\vec k) = \frac{1}{e^{\frac{\hbar\omega}{kT}}-1} }\qquad \text{vergleiche Einstein-Bose-Statistik} \]


Energie im Strahlungsfeld pro Volumeneinheit

\begin{align}
\int U(\omega)d\omega &= \frac{1}{V} \sum_{r,\vec k}n_r(\vec k) \hbar \omega_k = \underbrace{\frac{(2\pi)^3}{V}\sum_{r,\vec k}}_{\int d^3 k\cdot 2}  n_r(\vec k)\hbar\omega \frac{1}{(2\pi)^3}\\
&=8\pi \int \frac{k^2 dck}{c^3}\frac{\hbar\omega}{(2\pi)^3}\frac{1}{e^{\hbar\omega/kT}-1}\\
&= \frac{8\pi\hbar}{c^3}\int d\omega (\frac{\omega}{2\pi})^3 \frac{1}{e^{\hbar\omega/kT}-1}
\end{align}


\[U(\omega) = \frac{8\pi\hbar}{c^3} (\frac{\omega}{2\pi})^3 \frac{1}{e^{\hbar\omega/kT}-1} = u(\nu) \frac{d\omega}{d\nu} = \frac{1}{2\pi}u(\nu) \]

mit \(\frac{d\omega}{d\nu} = \frac{1}{2\pi}\)

\[\Rightarrow u(\nu) = 2\pi U(\nu) = \frac{8\pi h}{c^3} r^3 \frac{1}{e^{\hbar\nu/kT}-1} \]

\[\boxed{ u(\nu)= \frac{8\pi h}{c^3} r^3 \frac{1}{e^{\hbar\nu/kT}-1}  }  \]



\end{document}

