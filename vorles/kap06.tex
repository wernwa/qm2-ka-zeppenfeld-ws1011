\documentclass[10pt,a4paper,oneside,fleqn]{article}
\usepackage{geometry}
\geometry{a4paper,left=20mm,right=20mm,top=1cm,bottom=2cm}
\usepackage[utf8]{inputenc}
%\usepackage{ngerman}
\usepackage{amsmath}                % brauche ich um dir Formel zu umrahmen.
\usepackage{amsfonts}                % brauche ich für die Mengensymbole
\usepackage{graphicx}
\setlength{\parindent}{0px}
\setlength{\mathindent}{10mm}
\usepackage{bbold}                    %brauche ich für die doppel Zahlen Darstellung (Einheitsmatrix z.B)
\usepackage{dsfont}          %F�r den Einheitsoperator \mathds 1


\usepackage{color}
\usepackage{titlesec} %sudo apt-get install texlive-latex-extra

\definecolor{darkblue}{rgb}{0.1,0.1,0.55}
\definecolor{verydarkblue}{rgb}{0.1,0.1,0.35}
\definecolor{darkred}{rgb}{0.55,0.2,0.2}

%hyperref Link color
\usepackage[colorlinks=true,
        linkcolor=darkblue,
        citecolor=darkblue,
        filecolor=darkblue,
        pagecolor=darkblue,
        urlcolor=darkblue,
        bookmarks=true,
        bookmarksopen=true,
        bookmarksopenlevel=3,
        plainpages=false,
        pdfpagelabels=true]{hyperref}

\titleformat{\chapter}[display]{\color{darkred}\normalfont\huge\bfseries}{\chaptertitlename\
\thechapter}{20pt}{\Huge}

\titleformat{\section}{\color{darkblue}\normalfont\Large\bfseries}{\thesection}{1em}{}
\titleformat{\subsection}{\color{verydarkblue}\normalfont\large\bfseries}{\thesubsection}{1em}{}

% Notiz Box
\usepackage{fancybox}
\newcommand{\notiz}[1]{\vspace{5mm}\ovalbox{\begin{minipage}{1\textwidth}#1\end{minipage}}\vspace{5mm}}

\usepackage{cancel}
\setcounter{secnumdepth}{3}
\setcounter{tocdepth}{3}





%-------------------------------------------------------------------------------
%Diff-Makro:
%Das Diff-Makro stellt einen Differentialoperator da.
%
%Benutzung:
% \diff  ->  d
% \diff f  ->  df
% \diff^2 f  ->  d^2 f
% \diff_x  ->  d/dx
% \diff^2_x  ->  d^2/dx^2
% \diff f_x  ->  df/dx
% \diff^2 f_x  ->  d^2f/dx^2
% \diff^2{f(x^5)}_x  ->  d^2(f(x^5))/dx^2
%
%Ersetzt man \diff durch \pdiff, so wird der partieller
%Differentialoperator dargestellt.
%
\makeatletter
\def\diff@n^#1{\@ifnextchar{_}{\diff@n@d^#1}{\diff@n@fun^#1}}
\def\diff@n@d^#1_#2{\frac{\textrm{d}^#1}{\textrm{d}#2^#1}}
\def\diff@n@fun^#1#2{\@ifnextchar{_}{\diff@n@fun@d^#1#2}{\textrm{d}^#1#2}}
\def\diff@n@fun@d^#1#2_#3{\frac{\textrm{d}^#1 #2}{\textrm{d}#3^#1}}
\def\diff@one@d_#1{\frac{\textrm{d}}{\textrm{d}#1}}
\def\diff@one@fun#1{\@ifnextchar{_}{\diff@one@fun@d #1}{\textrm{d}#1}}
\def\diff@one@fun@d#1_#2{\frac{\textrm{d}#1}{\textrm{d}#2}}
\newcommand*{\diff}{\@ifnextchar{^}{\diff@n}
  {\@ifnextchar{_}{\diff@one@d}{\diff@one@fun}}}
%
%Partieller Diff-Operator.
\def\pdiff@n^#1{\@ifnextchar{_}{\pdiff@n@d^#1}{\pdiff@n@fun^#1}}
\def\pdiff@n@d^#1_#2{\frac{\partial^#1}{\partial#2^#1}}
\def\pdiff@n@fun^#1#2{\@ifnextchar{_}{\pdiff@n@fun@d^#1#2}{\partial^#1#2}}
\def\pdiff@n@fun@d^#1#2_#3{\frac{\partial^#1 #2}{\partial#3^#1}}
\def\pdiff@one@d_#1{\frac{\partial}{\partial #1}}
\def\pdiff@one@fun#1{\@ifnextchar{_}{\pdiff@one@fun@d #1}{\partial#1}}
\def\pdiff@one@fun@d#1_#2{\frac{\partial#1}{\partial#2}}
\newcommand*{\pdiff}{\@ifnextchar{^}{\pdiff@n}
  {\@ifnextchar{_}{\pdiff@one@d}{\pdiff@one@fun}}}
\makeatother
%
%Das gleich nur mit etwas andere Syntax. Die Potenz der Differentiation wird erst
%zum Schluss angegeben. Somit lautet die Syntax:
%
% \diff_x^2  ->  d^2/dx^2
% \diff f_x^2  ->  d^2f/dx^2
% \diff{f(x^5)}_x^2  ->  d^2(f(x^5))/dx^2
% Ansonsten wie Oben.
%
%Ersetzt man \diff durch \pdiff, so wird der partieller
%Differentialoperator dargestellt.
%
%\makeatletter
%\def\diff@#1{\@ifnextchar{_}{\diff@fun#1}{\textrm{d} #1}}
%\def\diff@one_#1{\@ifnextchar{^}{\diff@n{#1}}%
%  {\frac{\textrm d}{\textrm{d} #1}}}
%\def\diff@fun#1_#2{\@ifnextchar{^}{\diff@fun@n#1_#2}%
%  {\frac{\textrm d #1}{\textrm{d} #2}}}
%\def\diff@n#1^#2{\frac{\textrm d^#2}{\textrm{d}#1^#2}}
%\def\diff@fun@n#1_#2^#3{\frac{\textrm d^#3 #1}%
%  {\textrm{d}#2^#3}}
%\def\diff{\@ifnextchar{_}{\diff@one}{\diff@}}
%\newcommand*{\diff}{\@ifnextchar{_}{\diff@one}{\diff@}}
%
%Partieller Diff-Operator.
%\def\pdiff@#1{\@ifnextchar{_}{\pdiff@fun#1}{\partial #1}}
%\def\pdiff@one_#1{\@ifnextchar{^}{\pdiff@n{#1}}%
%  {\frac{\partial}{\partial #1}}}
%\def\pdiff@fun#1_#2{\@ifnextchar{^}{\pdiff@fun@n#1_#2}%
%  {\frac{\partial #1}{\partial #2}}}
%\def\pdiff@n#1^#2{\frac{\partial^#2}{\partial #1^#2}}
%\def\pdiff@fun@n#1_#2^#3{\frac{\partial^#3 #1}%
%  {\partial #2^#3}}
%\newcommand*{\pdiff}{\@ifnextchar{_}{\pdiff@one}{\pdiff@}}
%\makeatother

%-------------------------------------------------------------------------------
%%Nützliche Makros um in der Quantenmechanik Bras, Kets und das Skalarprodukt
%%zwischen den beiden darzustellen.
%%Benutzung:
%% \bra{x}  ->    < x |
%% \ket{x}  ->    | x >
%% \braket{x}{y} ->   < x | y >



\newcommand\bra[1]{\left\langle #1 \right|}
\newcommand\ket[1]{\left| #1 \right\rangle}
\newcommand\braket[2]{%
 \left\langle \vphantom{#2} #1%
   \middle|%
   \vphantom{#1} #2\right\rangle}%

%-------------------------------------------------------------------------------
%%Aus dem Buch:
%%Titel:  Latex in Naturwissenschaften und Mathematik
%%Autor:  Herbert Voß
%%Verlag: Franzis Verlag, 2006
%%ISBN:   3772374190, 9783772374197
%%
%%Hier werden drei Makros definiert:\mathllap, \mathclap und \mathrlap, welche
%%analog zu den aus Latex bekannten \rlap und \llap arbeiten, d.h. selbst
%%keinerlei horizontalen Platz benötigen, aber dennoch zentriert zum aktuellen
%%Punkt erscheinen.

\newcommand*\mathllap{\mathstrut\mathpalette\mathllapinternal}
\newcommand*\mathllapinternal[2]{\llap{$\mathsurround=0pt#1{#2}$}}
\newcommand*\clap[1]{\hbox to 0pt{\hss#1\hss}}
\newcommand*\mathclap{\mathpalette\mathclapinternal}
\newcommand*\mathclapinternal[2]{\clap{$\mathsurround=0pt#1{#2}$}}
\newcommand*\mathrlap{\mathpalette\mathrlapinternal}
\newcommand*\mathrlapinternal[2]{\rlap{$\mathsurround=0pt#1{#2}$}}

%%Das Gleiche nur mit \def statt \newcommand.
%\def\mathllap{\mathpalette\mathllapinternal}
%\def\mathllapinternal#1#2{%
%  \llap{$\mathsurround=0pt#1{#2}$}% $
%}
%\def\clap#1{\hbox to 0pt{\hss#1\hss}}
%\def\mathclap{\mathpalette\mathclapinternal}
%\def\mathclapinternal#1#2{%
%  \clap{$\mathsurround=0pt#1{#2}$}%
%}
%\def\mathrlap{\mathpalette\mathrlapinternal}
%\def\mathrlapinternal#1#2{%
%  \rlap{$\mathsurround=0pt#1{#2}$}% $
%}

%-------------------------------------------------------------------------------
%%Hier werden zwei neue Makros definiert \overbr und \underbr welche analog zu
%%\overbrace und \underbrace funktionieren jedoch die Gleichung nicht
%%'zerreißen'. Dies wird ermöglicht durch das \mathclap Makro.

\def\overbr#1^#2{\overbrace{#1}^{\mathclap{#2}}}
\def\underbr#1_#2{\underbrace{#1}_{\mathclap{#2}}}


\begin{document}
\tableofcontents
\setcounter{chapter}{5}
\chapter{Kap 6. Relativistische QM}
%\setcounter{chapter}{6}


Notation: Vierer-Vektoren

\[x^{\mu} = ct,x,y,z) = (x^0,x^1,x^2,x^3) = (ct,\vec r)\]

invariante Lönge \(\sqrt{x^2}\)

\[x^2=x\cdot x = x^{\mu}x_\mu = x^\mu g_{\mu \nu}x^\nu\]



Einsteinsche Summenkonvention: \(\sum_{\mu=0}^3\) für jedes Paar von oberen und unteren Index

Metrischer Tensor

\[ g_{\mu\nu} = \begin{bmatrix}
 1 & 0 & 0 & 0\\
 0 & -1 & 0 & 0\\
 0 & 0 & -1 & 0\\
 0 & 0 & 0 & -1\\
\end{bmatrix} \]

\[ x_\mu = g_{\mu\nu}x^\nu = (ct,-\vec r)\]

\[x^\mu = g^{\mu\nu}x_\nu = g^{\mu\nu}x^\nu = g^\nu_\nu x^\nu\] 

\[g^\nu_\nu = \delta^\nu_\nu =\begin{cases}
  1,  & \mu = \nu\\
  0 & \text{sonst}
\end{cases} \]

\[= g^{\mu\rho}g_{\rho\nu} \rightarrow  g^{\mu\nu} = [g_{\mu\nu}]^{-1} = \begin{bmatrix}
 1 & 0 & 0 & 0\\
 0 & -1 & 0 & 0\\
 0 & 0 & -1 & 0\\
 0 & 0 & 0 & -1\\
\end{bmatrix} \]

Vierer-Impuls: \(p^\mu=(\frac{E}{c},\vec p)\) \(E= \sqrt{(mc^2)^2+(\vec p c)^2}\)

\[p^2 = p_\mu p^\mu = \frac{E^2}{c^2}-\vec p^2 = \frac{m^2c^4+\vec p^2 c^2}{c^2} - \vec p^2 = m^2c^2\]

Vierer-Potential:  LT \(x^{'\mu} = \Lambda^{\mu}_\nu x^\nu\)

\[A^\mu = (\frac{\phi}{c},\vec A \qquad \rightarrow  A^{'\mu}(x') = \Lambda^\mu_\nu A^\nu(x)\]

Strom: \(j^\mu = (c\rho,\vec j)\) in E und M

Skalarprodukt für \(a^\mu,b^\mu: \qquad a\cdot b = a^\mu b_\mu = a^\mu g_{\mu\nu}b^\nu = a^0 b^0 - \vec a\cdot \vec b\)

Ableitung nach \(x^\nu\)

\[\partial_\mu = \frac{\partial}{\partial x^\mu} = (\frac{1}{c} \frac{\partial}{\partial t},\vec \nabla )\]

ist kovarianter Vektor unter Index wg: \(\partial \mu a\cdot x = \frac{\partial}{\partial x^\mu}(a_\nu x^\nu)= a_\mu\)

Entsprechend \(\partial^\mu = g^{\mu\nu}\partial_\nu = (\frac{1}{c}\frac{\partial}{\partial t},-\vec\nabla)\)


d'Alebert Operator

\[\square = \partial_\mu\partial^\mu = \frac{1}{c^2} \frac{\partial^2}{\partial t^2} - \vec \nabla^2\]

\subsection{QM eines freien Teilchens}

\[E\rightarrow i\hbar\frac{\partial}{\partial t}, \quad \vec p = \frac{\hbar}{i}\vec \nabla\]

\[ p^\mu = (\frac{E}{c},\vec p) \rightarrow (i\hbar \frac{1}{c}\frac{\partial}{\partial t},-i\hbar \vec \nabla) = i\hbar\partial^\mu\]

Schrödinger Gl. für NR freies Teilchens

\[E=\frac{\vec p^2}{2m} \rightarrow  i\hbar \frac{\partial }{\partial t}\psi = -\frac{\hbar^2\nabla^2}{2m}\psi(\vec x,t) \]


Relativistischer Fall

\begin{enumerate}
\item[1)] \(E = \sqrt{m^2c^4+\vec p^2c^2} \rightarrow \) nichtlokalen Operator
\item[2)] \(\frac{E^2}{c^2} = m^2c^2 + \vec p^2 \rightarrow - \frac{\hbar^2}{c^2}\frac{\partial^2 }{\partial t^2}\psi = m^2c^2\psi -\hbar^2\vec \nabla^2\psi \)
\end{enumerate}

\[\Leftrightarrow  0 = m^2c^2\psi + \hbar^2(\frac{1}{c^2}\frac{\partial^2}{\partial t^2 - \nabla^2)\psi = m^2c^2\psi+\hbar^2 \square \psi})\]


Klein Gordon Gl.

\[(\square + (\frac{mc}{\hbar})^2)\psi(x) = 0\]

Anwendbar auf skalare Teilichen (Spin 0) wie \(\pi^+,\pi^-,\pi^0,K,H\)

Lösungen der KG-Gl. durch ebene Wellen

\[\psi_p(x) = N e^{-ip\cdot x/\hbar} = Ne^{-iEt/\hbar}e^{+i\vec p\cdot\vec x/\hbar}\]

mit \(p\cdot x = p^\mu x_\mu = Et-\vec p\cdot \vec x\)

\[\square \psi_p = (x) = \frac{\partial}{\partial x^\mu} \frac{\partial}{\partial x_\mu} \psi_p(x) = N(-\frac{i}{\hbar}p_\mu)(-\frac{i}{\hbar}p^\mu) e^{-ip\cdot x/\hbar} = -\frac{p^2}{\hbar^2}\psi_p \]


KG:

\[\Rightarrow (-\frac{p^2}{\hbar^2} + \frac{m^2c^2}{\hbar^2})\psi_p(x) = 0 \Leftrightarrow p^2 = m^2 c^2; \frac{E^2}{c^2}-\vec p^2\]

\[\rightarrow E = \pm c \sqrt{m^2c^2+\vec p^2}\]

Lösungen mit Negativer Energie und das Energiespektrum ist nach unten nicht beschränkt. 

\subsection{Wahrscheinlichkeitserhaltung}

Kontin.Gl \(\frac{\partial\rho}{\partial t}+\vec \nabla\cdot \vec J = 0 \Leftrightarrow \partial_\mu j^\mu = 0\) mit \(j^\mu = (\rho c,\vec j)\). 

Gibt es einen erhaltenen 4-Strom für die lösung der KG-Gleichung?

\[\psi^*(\square + (\frac{mc}{\hbar})^2)\psi(x) - \psi(\square + (\frac{mc}{\hbar})^2)\psi^*(x) = 0\]


\[\psi^*(\partial_\mu\partial^\mu \psi) - \psi(\partial_\mu\partial^\mu \psi^*) = 0 \]

\[\partial_\mu(\underbrace{\psi^*\partial^\mu\psi - \psi \partial^\mu \psi^*}_{\propto j^\mu}) = 0\]

\[ j^\mu \propto (\psi^*\frac{i}{c}\frac{\propto}{\propto t}\psi - \psi\frac{i}{c}\frac{\propto}{\propto t}\psi^* ,-(\psi^*\vec \nabla \psi - \psi\vec \nabla\psi^*)) \]

Kandidat für Wahrscheinlichkeits Strom \(\frac{2im}{\hbar}\vec j\) in Schrödinger Gl

\[j^\mu = \frac{i\hbar}{2m} (\psi^*\partial^\mu \psi - \psi \partial^\mu \psi^*)\]

\[\rightarrow j^0 = \rho c =  \frac{i\hbar}{2mc}(\psi^*\frac{\partial\psi}{\partial t} - \psi\frac{\partial\psi^*}{\partial t}) \]

Anwendung auf stationäre Lösung: \(\psi_E(x) = e^{-iEt/\hbar}\psi_E(\vec x)\)

\[\frac{\partial \psi_E}{\partial t} = -\frac{iE}{\hbar}\psi_E,\frac{\partial \psi_E^*}{\partial t} = -\frac{iE}{\hbar}\psi_E^* \Rightarrow \rho = \frac{i\hbar}{2mc^2}|\psi_E(\vec x)|^2\frac{-2iE}{\hbar} = \frac{E}{mc^2}|\psi_E(x)|^2 \]

\(\rho < 0\) für Zustände mit \(E<0\)

\(\Rightarrow  \) Keine mögliche Wahrscheinlichkeitsdichte. (Ok für Zustände mit positiver Energie)

Interpretation: Zustände mit \(E>0\Leftrightarrow \) z.B. \(\pi^+\) und  \(E<0\Leftrightarrow \) z.B. \(\pi^-\)(Antiteilchen zum  \(\pi^+\))

\(\rho > 0\): \(\pi^+\) dominieren
\(\rho < 0\): \(\pi^-\) dominieren

\(\rho \propto\) elektromagn. Ladungsdichte

\[ j^\mu = |e|\frac{i\hbar}{2mc} (\psi^*\partial^\mu\psi -\psi\partial^\mu\psi^* )\]


Elektronen: Spin

\(\rightarrow \) Wellenfunktion \(\psi(x)\) hat \(\geq 2\) Komponenten

\[\psi(x) =\begin{pmatrix} \psi_1(x)\\ ...\\ \psi_N(x) \end{pmatrix} \]

Möglichkeit: Matrixstruktur für \(\hat H\)

\[i\hbar \frac{\partial}{\partial t}\psi(x) = \hat H \psi(x)\]




Ansatz: \(i\hbar \frac{\partial }{\partial t} \psi = \hat H \psi\) mit \(\psi(x) =\begin{pmatrix} \psi_1(x)\\ ...\\ \psi_N(x) \end{pmatrix} \)

und Wahrscheinlichkeitsdichte \(\rho = \sum_{i=1}^N |\psi_i|^2\)

\[\Rightarrow \hat H \propto \frac{\partial}{\partial x^i}\propto \hat p_i\]

Ansatz für \(\hat H\)

\[\hat H = c(\alpha_x\hat p_x+\alpha_y\hat p_y) +\beta mc^2 = c\sum_{i=1}^3 \alpha_i\hat p_i+ \beta mc^2\]

Ebene Wellenlösung für freie Teilchen

\[\psi(x) = e^{-px/\hbar}\psi(p)\]

mit \(p^2 = m^2c^2\)

\[\Rightarrow  E\psi(p) =[c \sum_{i=1}^3 \alpha_ip_i+\beta mc^2]\psi(p)\]

\[E^2\psi(p) = (m^2c^4+\vec p^2 c^2)\psi(p)\]
\[E c(\vec\alpha\vec p + \beta mc)\psi(p) = c^2(\vec \alpha\vec p + \beta mc)^2\psi(p)\]

\[= c^2(\sum_{i,j=1}^3\alpha_i\alpha_j p_ip_j+\sum_{i=1}^3(\alpha_i\beta+\beta\alpha_i)p_i mc +  \beta^2m^2c^2)\psi(p)\]

Koeffizienfenvergleich: \(\beta^2=1\); Antikommutator: \[\{\alpha_i,\beta\}=0\]
\begin{itemize}
\item  \(\boxed{\beta^2=1}\)
\item Antikommutator: \(\boxed{\{\alpha_i,\beta\}=0}\)
\item \(i\neq j\): z.B: \(p_xp_y\{\alpha_x\alpha_y+\alpha_y\alpha_x\}\); \(\{\alpha_i,\alpha_j\}=0\)
\item \(i=j\): \(\alpha_x^2p_x^2+\alpha_y^2p_y^2+\alpha_z^2p_z^2=\vec p^2 \Rightarrow \alpha_i^2 = 1\)
\[\Rightarrow \boxed{\{\alpha_i,\alpha_j\}=2\delta_{ij}}\]
\end{itemize}


\begin{enumerate}
\item[1)] \(\hat p_i,\hat H\) hermitesch \(\Rightarrow \vec\alpha,\beta\) hermitesch
\item[2)] \(\alpha_i^2=1,\beta^2=1 \Rightarrow \) Eigenwerte von \(\alpha_i,\beta\)
\item[3)] \(\alpha_i\beta + \beta\alpha_i=0\qquad |\cdot \beta\)
\[\Rightarrow \alpha_i=-\beta\alpha_i\beta \Rightarrow Tr[\alpha_i] = -Tr[\beta\alpha_i\beta]=-Tr[\alpha_i\beta^2]=-Tr[\alpha_i]\]
\end{enumerate}

\# - Anzahl; N - Dimension der Matrix

\# EW +1 = \# EW -1

\(\Rightarrow N\) gerade (\(N=2,4,...\))

\(N=2\Rightarrow 3\) Pauli Matrizen als Kandidaten benötigt: 4 Matrizen
\(\Rightarrow N\geq 4: N=4\) funktioniert

\(N=4:\) Dirac Basis: \(\beta\) diagonal

\[\beta=\begin{pmatrix}1&0&0&0\\ 0&1&0&0\\ 0&0&-1&0\\0&0&0&-1\end{pmatrix}= \begin{pmatrix}\mathbb 1&0\\ 0&\mathbb 1\end{pmatrix}\]

\(\alpha_i\) hermitesch + \(\{\alpha_i,\beta\}=0\)

\[\alpha =\begin{pmatrix}A&B\\ C&D\end{pmatrix}\qquad \begin{pmatrix}A&-B\\ C&-D\end{pmatrix} \]

\(A=D=0\), \(C=B^\dagger\)

\[\beta\alpha = \begin{pmatrix}A&B\\ -C&-D\end{pmatrix}\]

\[\Rightarrow \alpha_i =\begin{pmatrix}0&\tau_i\\ \tau_i^\dagger&0\end{pmatrix} \]

\[\{\alpha_i,\alpha_j\}=2\delta_{ij} \Leftrightarrow \tau_i\tau_j^\dagger+\tau_j\tau_i^\dagger = 2\delta_{ij}\]

Lösung \(\tau_i=\sigma_i=\) Pauli Matrizen

\[\Rightarrow \boxed{ \beta= \begin{pmatrix}\mathbb 1&0\\ 0&-\mathbb 1\end{pmatrix};\qquad \alpha_i=\begin{pmatrix} 0&\sigma_i\\ \sigma_i&0\end{pmatrix} }\]

\section{Dirac Gleichung}

\[i\hbar \frac{\partial}{\partial t}\psi(x) = c(\vec \alpha\cdot\frac{\hbar}{i}\vec \nabla + \beta mc)\psi(x) \qquad |\cdot \frac{\beta}{\hbar c}\]

Alternativ: kovariante Form

\[\Rightarrow i\beta \underbrace{\frac{i}{c}\frac{\partial}{\partial t}}_{\frac{\partial}{\partial x^0}}\psi+i\underbrace{\beta\vec \alpha_i}_{\gamma^i}\cdot\underbrace{\vec\nabla_i}_{\frac{\partial}{\partial x^i} }\psi-\frac{mc}{\hbar}\psi=0\]

\[\Rightarrow (i\gamma^\mu\frac{\partial}{\partial x^\mu} - \frac{mc}{\hbar})\psi = 0 \]

\(\gamma^0 = \beta\); \(\gamma^i = \beta\alpha_i\)

\[\boxed{\left(i\gamma^\mu\partial_\mu - \frac{mc}{\hbar}\right)\psi=0}\]

Kovariante Form der Dirac Gleichung mit \(\boxed{\{\gamma^\mu,\gamma^\nu\}=2g^{\mu\nu}}=2g^{\mu\nu}\mathbb 1_4\)

z.B. \(\{\gamma^i,\gamma^j\}=\beta\underbrace{\alpha_I\beta}_{-\beta\alpha_i}\alpha_j + \beta\underbrace{\alpha_j\beta}_{-\beta\alpha_j}\alpha_i = -\{\alpha_i,\alpha_j\}=-2\delta_{ij}\)

\subsection{Wahrscheinlichkeitsstrom}

\[i\hbar \frac{\partial \psi}{\partial t} = \frac{\hbar c}{i}\vec \alpha\cdot\vec\nabla\psi+\beta mc^2\psi\]

adjungierte Dirac Gleichung:

\[-i\hbar \frac{\partial \psi^\dagger}{\partial t} = \frac{\hbar c}{i}\vec(\vec\nabla\psi^\dagger)\vec \alpha+\beta mc^2\psi^\dagger \qquad |\cdot \psi\]

Differenz der beiden Gleichungen:

\[ i\hbar\frac{\partial}{\partial t} (\psi^\dagger\psi) = \frac{\hbar c}{i}(\psi^\dagger \vec \alpha\cdot\vec\nabla\psi+(\vec\nabla\psi^\dagger)\vec\alpha \psi) \]

\[\Rightarrow \frac{\partial}{\partial t}(\psi^\dagger\psi) = -c\vec\nabla(\psi^\dagger\vec\alpha\psi)\]

\[ \frac{\partial}{\partial t}\underbrace{(\psi^\dagger\psi) }_{\rho}+\vec\nabla\cdot(\underbrace{c\psi^\dagger\vec\alpha\psi }_{\vec j}) \]


\[\rho =\psi^\dagger\psi = \sum_i|\psi_i|^2\geq 0 \]

\(\rho\) ist positiv definierte Warscheinlichkeitsdichte

Kovariante Form des W-Stroms

\begin{align}
j^\mu &= (c\psi^\dagger\psi ,c\psi^\dagger\vec\alpha\psi )\\
&= (c\psi^\dagger\beta\gamma^0\psi ,c\psi^\dagger\beta\vec\gamma\psi )\\
&= c\psi^\dagger \beta\gamma^\mu \psi = c\overline \psi\gamma^\mu \psi
\end{align}

wobei \(\overline \psi = \psi^\dagger\beta=\psi^\dagger\gamma^0\) der Pauli adungierte Spinor ist.


\subsection{Elektromagnetische Wechselwirkung}

externe \(\vec E,\vec B\) Fleder \(\vec B = \vec\nabla\times\vec A\), \(\vec E = -\vec\nabla\phi-\frac{\partial\vec A}{\partial t}\)

\[\rightarrow A^\mu = (\frac{\phi}{c},\vec A)\]

minimale Subsittution:

\[p^\mu\rightarrow p^\mu-eA^\mu \quad QM\rightarrow i\hbar\partial^\mu-eA^\mu = i\hbar(\partial^\mu+\frac{ie}{\hbar}A^\mu)=i\hbar D^\mu\]

Komponenten der Kovarianten Ableitung \(D^\mu\)

\[i\hbar D^\mu = (i\hbar \frac{1}{c} \frac{\partial}{\partial t} - \frac{e}{c}\phi,\frac{\hbar}{i}\vec\nabla-e\vec A)\]

\[=(\frac{i}{c}(c\hbar\frac{\partial}{\partial t}-e\phi),\frac{\hbar}{i}\vec\nabla-e\vec A) \]

jErsetze in freier Dirac-Gl \(\partial \)


\[\boxed{i\hbar \frac{\partial}{\partial t}\psi(x) = c\vec \alpha(\frac{\hbar}{i}\vec\nabla-e\vec A)\psi+\beta m c^2\psi+e\phi\psi}\]

oder

\[\boxed{(i\gamma^\mu D_\mu - \frac{mc}{\hbar})\psi = 0}\]

beschreibt WW eines Elektrons der Ladung e mit dem elektromagnetischen Feld.


\end{document}
