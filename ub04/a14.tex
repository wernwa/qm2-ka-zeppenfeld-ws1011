\documentclass[10pt,a4paper,oneside,fleqn]{article}
\usepackage{geometry}
\geometry{a4paper,left=20mm,right=20mm,top=1cm,bottom=2cm}
\usepackage[utf8]{inputenc}
%\usepackage{ngerman}
\usepackage{amsmath}                % brauche ich um dir Formel zu umrahmen.
\usepackage{amsfonts}                % brauche ich für die Mengensymbole
\usepackage{graphicx}
\setlength{\parindent}{0px}
\setlength{\mathindent}{10mm}
\usepackage{bbold}                    %brauche ich für die doppel Zahlen Darstellung (Einheitsmatrix z.B)
\usepackage{dsfont}          %F�r den Einheitsoperator \mathds 1


\usepackage{color}
\usepackage{titlesec} %sudo apt-get install texlive-latex-extra

\definecolor{darkblue}{rgb}{0.1,0.1,0.55}
\definecolor{verydarkblue}{rgb}{0.1,0.1,0.35}
\definecolor{darkred}{rgb}{0.55,0.2,0.2}

%hyperref Link color
\usepackage[colorlinks=true,
        linkcolor=darkblue,
        citecolor=darkblue,
        filecolor=darkblue,
        pagecolor=darkblue,
        urlcolor=darkblue,
        bookmarks=true,
        bookmarksopen=true,
        bookmarksopenlevel=3,
        plainpages=false,
        pdfpagelabels=true]{hyperref}

\titleformat{\chapter}[display]{\color{darkred}\normalfont\huge\bfseries}{\chaptertitlename\
\thechapter}{20pt}{\Huge}

\titleformat{\section}{\color{darkblue}\normalfont\Large\bfseries}{\thesection}{1em}{}
\titleformat{\subsection}{\color{verydarkblue}\normalfont\large\bfseries}{\thesubsection}{1em}{}

% Notiz Box
\usepackage{fancybox}
\newcommand{\notiz}[1]{\vspace{5mm}\ovalbox{\begin{minipage}{1\textwidth}#1\end{minipage}}\vspace{5mm}}

\usepackage{cancel}
\setcounter{secnumdepth}{3}
\setcounter{tocdepth}{3}





%-------------------------------------------------------------------------------
%Diff-Makro:
%Das Diff-Makro stellt einen Differentialoperator da.
%
%Benutzung:
% \diff  ->  d
% \diff f  ->  df
% \diff^2 f  ->  d^2 f
% \diff_x  ->  d/dx
% \diff^2_x  ->  d^2/dx^2
% \diff f_x  ->  df/dx
% \diff^2 f_x  ->  d^2f/dx^2
% \diff^2{f(x^5)}_x  ->  d^2(f(x^5))/dx^2
%
%Ersetzt man \diff durch \pdiff, so wird der partieller
%Differentialoperator dargestellt.
%
\makeatletter
\def\diff@n^#1{\@ifnextchar{_}{\diff@n@d^#1}{\diff@n@fun^#1}}
\def\diff@n@d^#1_#2{\frac{\textrm{d}^#1}{\textrm{d}#2^#1}}
\def\diff@n@fun^#1#2{\@ifnextchar{_}{\diff@n@fun@d^#1#2}{\textrm{d}^#1#2}}
\def\diff@n@fun@d^#1#2_#3{\frac{\textrm{d}^#1 #2}{\textrm{d}#3^#1}}
\def\diff@one@d_#1{\frac{\textrm{d}}{\textrm{d}#1}}
\def\diff@one@fun#1{\@ifnextchar{_}{\diff@one@fun@d #1}{\textrm{d}#1}}
\def\diff@one@fun@d#1_#2{\frac{\textrm{d}#1}{\textrm{d}#2}}
\newcommand*{\diff}{\@ifnextchar{^}{\diff@n}
  {\@ifnextchar{_}{\diff@one@d}{\diff@one@fun}}}
%
%Partieller Diff-Operator.
\def\pdiff@n^#1{\@ifnextchar{_}{\pdiff@n@d^#1}{\pdiff@n@fun^#1}}
\def\pdiff@n@d^#1_#2{\frac{\partial^#1}{\partial#2^#1}}
\def\pdiff@n@fun^#1#2{\@ifnextchar{_}{\pdiff@n@fun@d^#1#2}{\partial^#1#2}}
\def\pdiff@n@fun@d^#1#2_#3{\frac{\partial^#1 #2}{\partial#3^#1}}
\def\pdiff@one@d_#1{\frac{\partial}{\partial #1}}
\def\pdiff@one@fun#1{\@ifnextchar{_}{\pdiff@one@fun@d #1}{\partial#1}}
\def\pdiff@one@fun@d#1_#2{\frac{\partial#1}{\partial#2}}
\newcommand*{\pdiff}{\@ifnextchar{^}{\pdiff@n}
  {\@ifnextchar{_}{\pdiff@one@d}{\pdiff@one@fun}}}
\makeatother
%
%Das gleich nur mit etwas andere Syntax. Die Potenz der Differentiation wird erst
%zum Schluss angegeben. Somit lautet die Syntax:
%
% \diff_x^2  ->  d^2/dx^2
% \diff f_x^2  ->  d^2f/dx^2
% \diff{f(x^5)}_x^2  ->  d^2(f(x^5))/dx^2
% Ansonsten wie Oben.
%
%Ersetzt man \diff durch \pdiff, so wird der partieller
%Differentialoperator dargestellt.
%
%\makeatletter
%\def\diff@#1{\@ifnextchar{_}{\diff@fun#1}{\textrm{d} #1}}
%\def\diff@one_#1{\@ifnextchar{^}{\diff@n{#1}}%
%  {\frac{\textrm d}{\textrm{d} #1}}}
%\def\diff@fun#1_#2{\@ifnextchar{^}{\diff@fun@n#1_#2}%
%  {\frac{\textrm d #1}{\textrm{d} #2}}}
%\def\diff@n#1^#2{\frac{\textrm d^#2}{\textrm{d}#1^#2}}
%\def\diff@fun@n#1_#2^#3{\frac{\textrm d^#3 #1}%
%  {\textrm{d}#2^#3}}
%\def\diff{\@ifnextchar{_}{\diff@one}{\diff@}}
%\newcommand*{\diff}{\@ifnextchar{_}{\diff@one}{\diff@}}
%
%Partieller Diff-Operator.
%\def\pdiff@#1{\@ifnextchar{_}{\pdiff@fun#1}{\partial #1}}
%\def\pdiff@one_#1{\@ifnextchar{^}{\pdiff@n{#1}}%
%  {\frac{\partial}{\partial #1}}}
%\def\pdiff@fun#1_#2{\@ifnextchar{^}{\pdiff@fun@n#1_#2}%
%  {\frac{\partial #1}{\partial #2}}}
%\def\pdiff@n#1^#2{\frac{\partial^#2}{\partial #1^#2}}
%\def\pdiff@fun@n#1_#2^#3{\frac{\partial^#3 #1}%
%  {\partial #2^#3}}
%\newcommand*{\pdiff}{\@ifnextchar{_}{\pdiff@one}{\pdiff@}}
%\makeatother

%-------------------------------------------------------------------------------
%%Nützliche Makros um in der Quantenmechanik Bras, Kets und das Skalarprodukt
%%zwischen den beiden darzustellen.
%%Benutzung:
%% \bra{x}  ->    < x |
%% \ket{x}  ->    | x >
%% \braket{x}{y} ->   < x | y >



\newcommand\bra[1]{\left\langle #1 \right|}
\newcommand\ket[1]{\left| #1 \right\rangle}
\newcommand\braket[2]{%
 \left\langle \vphantom{#2} #1%
   \middle|%
   \vphantom{#1} #2\right\rangle}%

%-------------------------------------------------------------------------------
%%Aus dem Buch:
%%Titel:  Latex in Naturwissenschaften und Mathematik
%%Autor:  Herbert Voß
%%Verlag: Franzis Verlag, 2006
%%ISBN:   3772374190, 9783772374197
%%
%%Hier werden drei Makros definiert:\mathllap, \mathclap und \mathrlap, welche
%%analog zu den aus Latex bekannten \rlap und \llap arbeiten, d.h. selbst
%%keinerlei horizontalen Platz benötigen, aber dennoch zentriert zum aktuellen
%%Punkt erscheinen.

\newcommand*\mathllap{\mathstrut\mathpalette\mathllapinternal}
\newcommand*\mathllapinternal[2]{\llap{$\mathsurround=0pt#1{#2}$}}
\newcommand*\clap[1]{\hbox to 0pt{\hss#1\hss}}
\newcommand*\mathclap{\mathpalette\mathclapinternal}
\newcommand*\mathclapinternal[2]{\clap{$\mathsurround=0pt#1{#2}$}}
\newcommand*\mathrlap{\mathpalette\mathrlapinternal}
\newcommand*\mathrlapinternal[2]{\rlap{$\mathsurround=0pt#1{#2}$}}

%%Das Gleiche nur mit \def statt \newcommand.
%\def\mathllap{\mathpalette\mathllapinternal}
%\def\mathllapinternal#1#2{%
%  \llap{$\mathsurround=0pt#1{#2}$}% $
%}
%\def\clap#1{\hbox to 0pt{\hss#1\hss}}
%\def\mathclap{\mathpalette\mathclapinternal}
%\def\mathclapinternal#1#2{%
%  \clap{$\mathsurround=0pt#1{#2}$}%
%}
%\def\mathrlap{\mathpalette\mathrlapinternal}
%\def\mathrlapinternal#1#2{%
%  \rlap{$\mathsurround=0pt#1{#2}$}% $
%}

%-------------------------------------------------------------------------------
%%Hier werden zwei neue Makros definiert \overbr und \underbr welche analog zu
%%\overbrace und \underbrace funktionieren jedoch die Gleichung nicht
%%'zerreißen'. Dies wird ermöglicht durch das \mathclap Makro.

\def\overbr#1^#2{\overbrace{#1}^{\mathclap{#2}}}
\def\underbr#1_#2{\underbrace{#1}_{\mathclap{#2}}}

%couchdb db=physik
%couchdb id=qm2uba14_SU(2)_und_SO(3)
%couchdb tags=qm2ub
%couchdb pdflink=http://wernwa-physik-ka.googlecode.com/svn/qm2ub/ub04/a14.pdf

\begin{document}
\section*{Aufgabe 14: SU(2) und SO(3)}

\begin{enumerate}
\item[\textbf{a})] Zeigen Sie, dass jede hermitesche, spurlose \(2\times 2\)-Matrix \(P\) als \(P=\vec p \cdot \vec \sigma\) geschrieben werden kann. Darin sind \( \sigma_i\) die Pauli-Matrizen und \(\vec p \in \mathbb R^3\)

\subsection*{LSG}

\(P\) sei hermitesche \(2\times 2\) Matrix mit \(Spur(P)=0\) und \(P^\dagger=P\);

\(\Rightarrow P=\begin{pmatrix} a&b\\c&d\end{pmatrix}\) mit \(a,b,c,d\in\mathbb C\) und \(a=-d\) und \(b^* = c\)

Pauli Matritzen: \(\sigma_1=\begin{pmatrix} 0&1\\1&0\end{pmatrix}\),\(\sigma_2=\begin{pmatrix} 0&-i\\i&0\end{pmatrix}\),\(\sigma_3=\begin{pmatrix} 1&0\\0&-1\end{pmatrix}\)

\(\Rightarrow P=\begin{pmatrix} a&b\\b^*&-a\end{pmatrix} =\begin{pmatrix} a&x+iy\\x-iy&-a\end{pmatrix} \); mit \(b = x+iy\)


\(\Rightarrow P=a\cdot\sigma_3 + \begin{pmatrix} 0&x+iy\\x-iy&0\end{pmatrix}= a\cdot \sigma_3 +x\cdot\sigma_1-y\cdot\sigma_2 = a\cdot \sigma_3 +x\cdot\sigma_1+y'\cdot\sigma_2 \)

setze \(a=z\)  \(P=\begin{pmatrix} z&x+iy\\x-iy&-z\end{pmatrix}\) mit \(x,y,z \in \mathbb R\)

Da \(P\) hermitesch ist \(a=z\in \mathbb R\)



\item[\textbf{b})] Solch eine Matrix werde nun mit einer unitären Matrix \(U\in SU(2)\) trasformiert, \(P' = U^{-1}PU\). Zeigen Sie, dass \(P'=(\vec p)'\cdot \vec \sigma\) mit \((\vec p)' \in \mathbb R^3\) und berechnen Sie \(det P\) sowie \(det P'\). Begründen Sie aus Ihren Ergebnissen, dass sich \(\vec p\) wie ein dreidimensionaler Vektor unter Drehungen transformiert.

\subsection*{LSG}

zu Zeigen: det P = det P' weil U unitär ist (\(detU=\frac{1}{detU^{-1}})\), weiterhin
Sp(P)=Sp(P')=0 wegen a)

\(U =\begin{pmatrix}a&b\\-b^*&a^* \end{pmatrix} \Rightarrow U^\dagger = \begin{pmatrix}a^*&-b\\b^*&a \end{pmatrix}  \)

\[ P' = U^\dagger P U = U^\dagger \vec p \vec \sigma U\]

\[Sp(P') = \sum_k U^\dagger_{ki} p_n\sigma_{ij} U_{jk} = \sum_j \underbrace{U_{jk}U^\dagger_{ki}}_{=1} p_n\sigma_{ij} = Sp(P)=0\]

\begin{align}
 P'&=U^\dagger PU =\begin{pmatrix}a^*&-b\\b^*&a\end{pmatrix}\begin{pmatrix} z&x+iy\\x-iy&-z\end{pmatrix}\begin{pmatrix}a&b\\-b^*&a^*\end{pmatrix}\\
&=\begin{pmatrix}a^*&-b\\b^*&a\end{pmatrix}\begin{pmatrix} az-b^*(x+iy)&bz+a^*(x+iy)\\ a(x-iy)+b^*z&b(x-iy)-a^*z \end{pmatrix}\\
&=\begin{pmatrix}a^*az-a^*b^*(x+iy)-ba(x-iy)-bb^*z&a^*bz+a^*a^*(x+iy)-b^2(x-iy)+ba^*z\\b^*az-b^*b^*(x+iy)+a^2(x-iy)+ab^*z & b^*bz+b^*a^*(x+iy)+ab(x-iy)-aa^*z\end{pmatrix} \\
&\equiv\begin{pmatrix}p_1&p_2\\p_2^*&-p_1\end{pmatrix} = p_1\sigma_3+Re\{p_2\}\sigma_1+Im\{p_2\}\sigma_2
\end{align}


\(det P =  -z^2-((x+iy)(x-iy)) = -(x^2+y^2+z^2)\equiv -||\vec p||^2\)

\begin{align}
det P' &= -||(\vec p)'||^2 \\
&= det(U^\dagger P U) \\
&= det(U^\dagger)\cdot det(P)\cdot det(U) \\
&= det(U^\dagger)\cdot det(U) \cdot det(P) \\
&= det(U^\dagger U) \cdot det(P)\\
&= \underbrace{det(U^\dagger U)}_{=1}\cdot det(P) \\
&= det(P) = -||\vec p||^2
\end{align}


Also kann die Unitäre Transformation in SU(2) eine Drehung in SO(3) sein, weil sich die Länge des Vektors nicht verändert.

\[\Rightarrow ||\vec p||^2 = ||(\vec p)'||^2 \]


Es ist möglich jedes Element aus SO(3) mit einem Element aus SU(2) eindeutig zuzuordnen (surjektiv???). Wenn wir \(U\) durch \(-U\in SU(2)\) ersetzen, folg:

\[P'' = (-U)^\dagger P(-U) = U^\dagger P U = P'\]

Somit wird \(-U\in SU(2)\) genau das gleiche \(R\in SO(3)\) wi auch \(U\in SU(2)\) zugeordnet.





\item[\textbf{c})] Wenn nun also \(U\) für \(\vec p\) eine 'gewöhnliche' Drehung \(R\) induziert, also \(p'_i = R_{ij}p_j\), dann sollte es einen Zusammenhang zwischen \(R\) und \(U\) geben. Drücken Sie die Matrixelemente \(R_{ij}\) mit Hilfe der Pauli-Matrizen durch \(U\) aus. Ist diese Zuordnung eindeutig?

\subsection*{LSG}

first try:
Mit der Darstellung \(P=\vec p \vec \sigma\) wird mit der U aus SU(2) Trasformiert und danach als eine 3D Rotationsmatrix dargestellt wie folgt: \(\vec p = 3DMatrix \cdot (\vec p)'\) 

schreibe \(\vec p =\begin{pmatrix}x\\y\\z\end{pmatrix}\) und \((\vec p)'=\begin{pmatrix}Re\{p_2\}\\Im\{p_2\}\\p1\end{pmatrix}\); TODO \((\vec p)' = R\cdot \vec p\)

\(p_1= a^*az-a^*b^*(x+iy)-ba(x-iy)-bb^*z\),
\(p_2 = a^*bz+(a^*)^2(x+iy)-b^2(x-iy)+ba^*z\)

\begin{align}
Re\{p_2\} &= (a^*bz+(a^*)^2(x+iy)-b^2(x-iy)+ba^*z)(a^*bz+(a^*)^2(x+iy)-b^2(x-iy)+ba^*z)^*\\
&=(a^*bz+(a^*)^2(x+iy)-b^2(x-iy)+ba^*z)(ab^*z+(a)^2(x-iy)-(b^*)^2(x+iy)+b^*az)\\
&=(aa^*bb^*z^2...
\end{align}



\begin{align}
\begin{pmatrix}Re\{p_2\}\\Im\{p_2\}\\p1\end{pmatrix} &= R\cdot \begin{pmatrix}x\\y\\z\end{pmatrix} \\
\end{align} 

\begin{align}
\begin{pmatrix}Re\{p_2\}\\Im\{p_2\}\\p1\end{pmatrix} &=
\begin{pmatrix} & & \\ & & \\(a^*)^2-b^2 &i(a^*)^2+ib^2 & a^*b+ba^*\end{pmatrix}
\begin{pmatrix}x\\y\\z\end{pmatrix} \\
\end{align}

\((x+iy)(x+iy) = x^2-y^2+2ixy\) 


second try:


\begin{align}
Up_j\sigma_j U^\dagger &= p'_k\sigma_k \\
U\sigma_j U^\dagger p_j &= p'_k\sigma_k\qquad |\cdot \sigma_i \\
U\sigma_j U^\dagger p_j \cdot \sigma_i &= p'_k\sigma_k\cdot \sigma_i
\end{align}

mit \([\sigma_i,\sigma_j]=2i\epsilon_{ijk}\sigma_k\);  \(\sigma_i\sigma_j=\sigma_j\sigma_i+2i\epsilon_{ijk}\sigma_k\); \(\sigma_j\sigma_i = \mathbb 1\cdot \delta_{ji}+2i\epsilon_{ijk}\sigma_k\)

\begin{align}
U\sigma_j U^\dagger p_j \cdot \sigma_i &= p'_k\mathbb 1 \delta_{ki}+2\epsilon_{kip}\sigma_p\\
U\sigma_j U^\dagger p_j \cdot \sigma_i &= p'_k\mathbb 1 \delta_{ki}+p'_k2\epsilon_{kip}\sigma_p\\
U\sigma_j U^\dagger p_j \cdot \sigma_i &= p'_i\mathbb 1 + p'_k2\epsilon_{kip}\sigma_p\\
Sp(U\sigma_j U^\dagger p_j \cdot \sigma_i) &= Sp(p'_i\mathbb 1 + p'_k2\epsilon_{kip}\sigma_p)\\
Sp(U\sigma_j U^\dagger\sigma_i)p_j &= p'_i \underbrace{Sp(\mathbb 1)}_{=2} + p'_k2\epsilon_{kip}\underbrace{Sp(\sigma_p)}_{=0}\\
\end{align}

\[ \Leftrightarrow p'_i = \underbrace{\frac{Sp(U\sigma_j U^\dagger\sigma_i)}{2}}_{R_{ij}}p_j\]




\end{enumerate}







-------------------------
Spur: \(Sp(A) = \sum_n \langle n |A|n\rangle\); 

\begin{align}
Sp(AB) = \sum_n \langle n |A\cdot B|n\rangle &= \sum_n \langle n |A\cdot \mathbb 1 \cdot B|n\rangle \\
&= \sum_n \langle n |A|m\rangle \langle m |B|n\rangle \\
&\equiv \sum_m \langle m |B|n\rangle \langle n |A|m\rangle \\
= Sp(BA)
\end{align}

\begin{align}
Sp(ABC) = \sum_k \langle k |A\cdot B\cdot C|k\rangle &= \sum_k \langle k |A\cdot \mathbb 1\cdot C\cdot \mathbb 1 \cdot B|k\rangle \\
&= \sum_k \langle k |A| i\rangle \langle i|C| j\rangle \langle j |B|k\rangle \\
&= \sum_j \langle j |B|k\rangle \langle k |A| i\rangle \langle i|C| j\rangle \\
&= Sp(CAB)\\
&= Sp(BCA)
\end{align}



mit \(Sp(\sigma_i)=0 \rightarrow Sp(P)=\sum_jp_i\sigma_{jj}=p_i\sum_j\sigma_{jj}=0\)




\end{document}
