\documentclass[10pt,a4paper,oneside,fleqn]{article}
\usepackage{geometry}
\geometry{a4paper,left=20mm,right=20mm,top=1cm,bottom=2cm}
\usepackage[utf8]{inputenc}
%\usepackage{ngerman}
\usepackage{amsmath}                % brauche ich um dir Formel zu umrahmen.
\usepackage{amsfonts}                % brauche ich für die Mengensymbole
\usepackage{graphicx}
\setlength{\parindent}{0px}
\setlength{\mathindent}{10mm}
\usepackage{bbold}                    %brauche ich für die doppel Zahlen Darstellung (Einheitsmatrix z.B)
\usepackage{dsfont}          %F�r den Einheitsoperator \mathds 1


\usepackage{color}
\usepackage{titlesec} %sudo apt-get install texlive-latex-extra

\definecolor{darkblue}{rgb}{0.1,0.1,0.55}
\definecolor{verydarkblue}{rgb}{0.1,0.1,0.35}
\definecolor{darkred}{rgb}{0.55,0.2,0.2}

%hyperref Link color
\usepackage[colorlinks=true,
        linkcolor=darkblue,
        citecolor=darkblue,
        filecolor=darkblue,
        pagecolor=darkblue,
        urlcolor=darkblue,
        bookmarks=true,
        bookmarksopen=true,
        bookmarksopenlevel=3,
        plainpages=false,
        pdfpagelabels=true]{hyperref}

\titleformat{\chapter}[display]{\color{darkred}\normalfont\huge\bfseries}{\chaptertitlename\
\thechapter}{20pt}{\Huge}

\titleformat{\section}{\color{darkblue}\normalfont\Large\bfseries}{\thesection}{1em}{}
\titleformat{\subsection}{\color{verydarkblue}\normalfont\large\bfseries}{\thesubsection}{1em}{}

% Notiz Box
\usepackage{fancybox}
\newcommand{\notiz}[1]{\vspace{5mm}\ovalbox{\begin{minipage}{1\textwidth}#1\end{minipage}}\vspace{5mm}}

\usepackage{cancel}
\setcounter{secnumdepth}{3}
\setcounter{tocdepth}{3}





%-------------------------------------------------------------------------------
%Diff-Makro:
%Das Diff-Makro stellt einen Differentialoperator da.
%
%Benutzung:
% \diff  ->  d
% \diff f  ->  df
% \diff^2 f  ->  d^2 f
% \diff_x  ->  d/dx
% \diff^2_x  ->  d^2/dx^2
% \diff f_x  ->  df/dx
% \diff^2 f_x  ->  d^2f/dx^2
% \diff^2{f(x^5)}_x  ->  d^2(f(x^5))/dx^2
%
%Ersetzt man \diff durch \pdiff, so wird der partieller
%Differentialoperator dargestellt.
%
\makeatletter
\def\diff@n^#1{\@ifnextchar{_}{\diff@n@d^#1}{\diff@n@fun^#1}}
\def\diff@n@d^#1_#2{\frac{\textrm{d}^#1}{\textrm{d}#2^#1}}
\def\diff@n@fun^#1#2{\@ifnextchar{_}{\diff@n@fun@d^#1#2}{\textrm{d}^#1#2}}
\def\diff@n@fun@d^#1#2_#3{\frac{\textrm{d}^#1 #2}{\textrm{d}#3^#1}}
\def\diff@one@d_#1{\frac{\textrm{d}}{\textrm{d}#1}}
\def\diff@one@fun#1{\@ifnextchar{_}{\diff@one@fun@d #1}{\textrm{d}#1}}
\def\diff@one@fun@d#1_#2{\frac{\textrm{d}#1}{\textrm{d}#2}}
\newcommand*{\diff}{\@ifnextchar{^}{\diff@n}
  {\@ifnextchar{_}{\diff@one@d}{\diff@one@fun}}}
%
%Partieller Diff-Operator.
\def\pdiff@n^#1{\@ifnextchar{_}{\pdiff@n@d^#1}{\pdiff@n@fun^#1}}
\def\pdiff@n@d^#1_#2{\frac{\partial^#1}{\partial#2^#1}}
\def\pdiff@n@fun^#1#2{\@ifnextchar{_}{\pdiff@n@fun@d^#1#2}{\partial^#1#2}}
\def\pdiff@n@fun@d^#1#2_#3{\frac{\partial^#1 #2}{\partial#3^#1}}
\def\pdiff@one@d_#1{\frac{\partial}{\partial #1}}
\def\pdiff@one@fun#1{\@ifnextchar{_}{\pdiff@one@fun@d #1}{\partial#1}}
\def\pdiff@one@fun@d#1_#2{\frac{\partial#1}{\partial#2}}
\newcommand*{\pdiff}{\@ifnextchar{^}{\pdiff@n}
  {\@ifnextchar{_}{\pdiff@one@d}{\pdiff@one@fun}}}
\makeatother
%
%Das gleich nur mit etwas andere Syntax. Die Potenz der Differentiation wird erst
%zum Schluss angegeben. Somit lautet die Syntax:
%
% \diff_x^2  ->  d^2/dx^2
% \diff f_x^2  ->  d^2f/dx^2
% \diff{f(x^5)}_x^2  ->  d^2(f(x^5))/dx^2
% Ansonsten wie Oben.
%
%Ersetzt man \diff durch \pdiff, so wird der partieller
%Differentialoperator dargestellt.
%
%\makeatletter
%\def\diff@#1{\@ifnextchar{_}{\diff@fun#1}{\textrm{d} #1}}
%\def\diff@one_#1{\@ifnextchar{^}{\diff@n{#1}}%
%  {\frac{\textrm d}{\textrm{d} #1}}}
%\def\diff@fun#1_#2{\@ifnextchar{^}{\diff@fun@n#1_#2}%
%  {\frac{\textrm d #1}{\textrm{d} #2}}}
%\def\diff@n#1^#2{\frac{\textrm d^#2}{\textrm{d}#1^#2}}
%\def\diff@fun@n#1_#2^#3{\frac{\textrm d^#3 #1}%
%  {\textrm{d}#2^#3}}
%\def\diff{\@ifnextchar{_}{\diff@one}{\diff@}}
%\newcommand*{\diff}{\@ifnextchar{_}{\diff@one}{\diff@}}
%
%Partieller Diff-Operator.
%\def\pdiff@#1{\@ifnextchar{_}{\pdiff@fun#1}{\partial #1}}
%\def\pdiff@one_#1{\@ifnextchar{^}{\pdiff@n{#1}}%
%  {\frac{\partial}{\partial #1}}}
%\def\pdiff@fun#1_#2{\@ifnextchar{^}{\pdiff@fun@n#1_#2}%
%  {\frac{\partial #1}{\partial #2}}}
%\def\pdiff@n#1^#2{\frac{\partial^#2}{\partial #1^#2}}
%\def\pdiff@fun@n#1_#2^#3{\frac{\partial^#3 #1}%
%  {\partial #2^#3}}
%\newcommand*{\pdiff}{\@ifnextchar{_}{\pdiff@one}{\pdiff@}}
%\makeatother

%-------------------------------------------------------------------------------
%%Nützliche Makros um in der Quantenmechanik Bras, Kets und das Skalarprodukt
%%zwischen den beiden darzustellen.
%%Benutzung:
%% \bra{x}  ->    < x |
%% \ket{x}  ->    | x >
%% \braket{x}{y} ->   < x | y >



\newcommand\bra[1]{\left\langle #1 \right|}
\newcommand\ket[1]{\left| #1 \right\rangle}
\newcommand\braket[2]{%
 \left\langle \vphantom{#2} #1%
   \middle|%
   \vphantom{#1} #2\right\rangle}%

%-------------------------------------------------------------------------------
%%Aus dem Buch:
%%Titel:  Latex in Naturwissenschaften und Mathematik
%%Autor:  Herbert Voß
%%Verlag: Franzis Verlag, 2006
%%ISBN:   3772374190, 9783772374197
%%
%%Hier werden drei Makros definiert:\mathllap, \mathclap und \mathrlap, welche
%%analog zu den aus Latex bekannten \rlap und \llap arbeiten, d.h. selbst
%%keinerlei horizontalen Platz benötigen, aber dennoch zentriert zum aktuellen
%%Punkt erscheinen.

\newcommand*\mathllap{\mathstrut\mathpalette\mathllapinternal}
\newcommand*\mathllapinternal[2]{\llap{$\mathsurround=0pt#1{#2}$}}
\newcommand*\clap[1]{\hbox to 0pt{\hss#1\hss}}
\newcommand*\mathclap{\mathpalette\mathclapinternal}
\newcommand*\mathclapinternal[2]{\clap{$\mathsurround=0pt#1{#2}$}}
\newcommand*\mathrlap{\mathpalette\mathrlapinternal}
\newcommand*\mathrlapinternal[2]{\rlap{$\mathsurround=0pt#1{#2}$}}

%%Das Gleiche nur mit \def statt \newcommand.
%\def\mathllap{\mathpalette\mathllapinternal}
%\def\mathllapinternal#1#2{%
%  \llap{$\mathsurround=0pt#1{#2}$}% $
%}
%\def\clap#1{\hbox to 0pt{\hss#1\hss}}
%\def\mathclap{\mathpalette\mathclapinternal}
%\def\mathclapinternal#1#2{%
%  \clap{$\mathsurround=0pt#1{#2}$}%
%}
%\def\mathrlap{\mathpalette\mathrlapinternal}
%\def\mathrlapinternal#1#2{%
%  \rlap{$\mathsurround=0pt#1{#2}$}% $
%}

%-------------------------------------------------------------------------------
%%Hier werden zwei neue Makros definiert \overbr und \underbr welche analog zu
%%\overbrace und \underbrace funktionieren jedoch die Gleichung nicht
%%'zerreißen'. Dies wird ermöglicht durch das \mathclap Makro.

\def\overbr#1^#2{\overbrace{#1}^{\mathclap{#2}}}
\def\underbr#1_#2{\underbrace{#1}_{\mathclap{#2}}}

%couchdb db=physik
%couchdb id=qm2uba15_clebsch-gordan-koeffizienten
%couchdb tags=qm2ub
%couchdb pdflink=http://wernwa-physik-ka.googlecode.com/svn/qm2ub/ub04/a15.pdf

\begin{document}
\subsection*{Aufgabe 15: Clebsch-Gordan Koeffizienten}

ein System sei aus zwei Spin-1 Systemen zusammengesetzt. Wie lauten die Eigenzustände des Gesamtdrehimpulses, \(|jm\rangle\), ausgedrückt als Linearkombinationen der gekoppelten \(|1m_1\rangle\) \(|1m_2\rangle\)-Zustände? Mit anderen Worten: bestimmen Sie die Clebsch-Gordan Koeffizienten für \(1\otimes 1=0\oplus 1\oplus 2\). Beachten Sie die Condon-Shortley Konvention. Sie können die Koeffizienten für \(|j-m\rangle\) aus denen für \(|jm\rangle\) durch bekannte Symmetrien bestimmen. Als letzen zustand werden Sie vermutlich \(|00\rangle\) berechnen. \(J_-\) auf deisen angewandt sollte 0 ergeben. Üerprüfen Sie das. (B ei der Gelegenheit lohnt es sich herauszufinden, was Alfred Clebsch mit dem Polytechnikum Karlsruhe zu tun hatte.)

\subsection*{LSG}


Die CGKs bilden eine vollständige orthonormale Basis für die gilt: \(\sum_{m_1,m_2}|j_1j_2;m_1m_2\rangle\langle j_1j_2;m_1m_2|=\mathbb 1\)
\begin{align}
|jm\rangle &= \left(\sum_{m_1,m_2}|j_1j_2;m_1m_2\rangle\langle j_1j_2;m_1m_2|\right)|jm\rangle \\
&= \sum_{m_1,m_2} \underbrace{\langle j_1j_2;m_1m_2|jm\rangle}_{CGK}|j_1j_2;m_1m_2 \rangle
\end{align}


Aufgrund der Beziegung \(m=m_1+m_2 \rightarrow \langle j_1j_2;m_1m_2|j,m=m_1+m_2\rangle \neq 0 \) lassen sich zunächst die Gesamtbasis \(|jm\rangle\) in der anderen Basis \(|j_1j_2;m_1m_2 \rangle\) ausdrücken:

\(j=0,1,2\), \(m=-2,-1,0,1,2\), \(m_1,m_2=\pm 1,0\)

\begin{itemize}
\item \(|00\rangle = \langle 11;-11|00\rangle|11;-11\rangle + \langle 11;1-1|00\rangle|11;1-1\rangle + \langle 11;00|00\rangle|11;00\rangle\)
\item \(|1-1\rangle =\langle 11;-10|1-1\rangle| 11;-10\rangle +\langle 11;0-1 |1-1\rangle |11;0-1 \rangle\)
\item \(|10\rangle = \langle 11;-11|10\rangle|11;-11\rangle +  \langle 11;1-1|10\rangle| 11;1-1\rangle+ \langle 11;00|10\rangle|11;00\rangle\)
\item \(|11\rangle = \langle 11;10 |11\rangle|11;10\rangle + \langle 11;01 |11\rangle|11;01\rangle\)
\item \(|2-2\rangle = \langle 11;-1-1|2-2\rangle|11;-1-1\rangle\)
\item \(|2 -1\rangle = \langle 11;-1 0 |2-1\rangle|11;-1 0\rangle + \langle 11;0-1|2-1\rangle|11;0-1\rangle\)
\item \(|2 0\rangle = \langle 11;-1 1 |20\rangle|11;-1 1\rangle+\langle11; 1-1 |20\rangle|11;1-1\rangle+\langle 11;00 |20\rangle|11;00\rangle\)
\item \(|21\rangle = \langle 11;10 |21\rangle|11;10\rangle+\langle 11; 01 |21\rangle|11;01\rangle\)
\item \(|22\rangle = \langle 11;11|22\rangle|11;111\rangle\)
\end{itemize}

\(\langle j'm'|jm\rangle = \delta_{j'j'}\delta_{m'm}\) \( \rightarrow\) \(\langle jm|jm\rangle = 1\)


\begin{align}
\langle jm |jm\rangle &= \langle jm|\left( \sum_{m_1m_2} |j_1j_2;m_1m_2\rangle \langle j_1j_2;m_1m_2|\right) |jm\rangle \\
 &= \sum_{m_1m_2}\langle jm|j_1j_2;m_1m_2\rangle \langle j_1j_2;m_1m_2|jm\rangle \\
&= \sum_{m_1m_2}\langle jm |j_1j_2;m_1m_2\rangle^2 \\
&=1
\end{align}

(1) \(\Rightarrow \sum_{m_1m_2} \langle jm |j_1j_2;m_1m_2\rangle^2 = 1\)

die Condon-Shortley Konvention (2): \(\langle jj|j_1j_1;m_1=j_2,m_2=j-j_1\rangle \equiv \text{positiv}\) (3) \(\langle j_1j_2;m_1m_2|jm\rangle = (-1)^{j-j_1-j_2}\langle j_2 j_1; m_2 m_1| j m\rangle\)
 (4) \[\langle j_1j_2;m_1m_2|jm\rangle = (-1)^{j-j_1-j_2}\langle j_2 j_1; -m_1 -m_2| j -m\rangle = \langle j_2 j_1; -m_2 -m_1| j -m\rangle \]

mit (1),(2) und (3) lassen sich die einfachen und zweifachen CGKs leicht finden:

\begin{itemize}
\item \(|00\rangle = \underbrace{\langle 11;-11|00\rangle}_{\equiv \text{positiv}}|11;-11\rangle + \underbrace{\langle 11;1-1|00\rangle}_{\equiv \text{positiv}}|11;1-1\rangle + \langle 11;00|00\rangle|11;00\rangle\)
\item \(|1-1\rangle =\underbrace{\langle 11;-10|1-1\rangle}_{\equiv -\langle 11;0-1 |1-1\rangle}| 11;-10\rangle +\langle 11;0-1 |1-1\rangle |11;0-1 \rangle\)
\item \(|10\rangle = \underbrace{\langle 11;-11|10\rangle}_{\equiv -\langle 11;1-1|10\rangle}|11;-11\rangle +  \langle 11;1-1|10\rangle| 11;1-1\rangle+ \langle 11;00|10\rangle|11;00\rangle\)
\item \(|11\rangle = \underbrace{\langle 11;10 |11\rangle}_{\equiv \text{aus (2) positiv} \rightarrow\frac{1}{\sqrt 2}} |11;10\rangle + \underbrace{\langle 11;01 |11\rangle}_{\equiv \text{aus (3) negativ} \rightarrow -\frac{1}{\sqrt 2}} |11;01\rangle\)
\item \(|2-2\rangle =\underbrace{ \langle 11;-1-1|2-2\rangle}_{\equiv 1\qquad \text{wegen (1)}}|11;-1-1\rangle\)
\item \(|2 -1\rangle = \underbrace{\langle 11;-1 0|2-1\rangle}_{\frac{1}{\sqrt 2}} |11;-1 0\rangle + \underbrace{\langle 11;0-1|2-1\rangle}_{\frac{1}{\sqrt 2}} |11;0-1\rangle\)
\item \(|2 0\rangle = \underbrace{\langle 11;-1 1 |20\rangle}_{\equiv \langle11; 1-1 |20\rangle}|11;-1 1\rangle+\langle11; 1-1 |20\rangle|11;1-1\rangle+\langle 11;00 |20\rangle|11;00\rangle\)
\item \(|21\rangle = \underbrace{\langle 11;10|21\rangle}_{\frac{1}{\sqrt 2}} |11;10\rangle+\underbrace{\langle 11; 01 |21\rangle}_{\frac{1}{\sqrt 2}}|11;01\rangle\)
\item \(|22\rangle = \underbrace{\langle 11;11|22\rangle}_{\equiv 1\quad \text{wegen (1) oder (2)}}|11;11\rangle\)
\end{itemize}


Aufgrund der Beziehung \(\langle j_1,j_2;m_1,m_2|J_{\pm}|jm\rangle=\langle j_1,j_2;m_1,m_2|J_{1\pm}+J_{2\pm}|jm\rangle\) und sich daraus ergebenden Gleichung:

(5)
\begin{align}
\sqrt{(j\mp m)(j\pm m+1)}&\langle j_1,j_2; m_1,m_2 | j,m\pm 1\rangle =\\
&=\sqrt{(j_1\pm m_1)(j_1\mp m_1 +1)}\langle j_1,j_2; m_1 \mp 1,m_2|jm\rangle \\
&+ \sqrt{(j_2\pm m_2)(j_2\mp m_2 +1)}\langle j_1,j_2; m_1 ,m_2 \mp 1|jm\rangle
\end{align}

Kommt man auf die restlichen CGKs. 

mit \(j=0,m=0; m_1=1; m_2=0\) eingesetzt in (5) ergibt \(- \langle 11;-11|00\rangle = \langle 11;00|00\rangle \) 
aus (1) und dem schon bekeannten Teilergebniss für \(|00\rangle\) folgt: \( \frac{1}{\sqrt 3} = \langle 11;-11|00\rangle = |11;1-1\rangle= -\langle 11;00|00\rangle\)

\(\Rightarrow |00\rangle =  \frac{1}{\sqrt 3}[|11;-11\rangle + |11;1-1\rangle - |00\rangle|11;00\rangle ]\)

mit \(j=1;m=-1; m_1=m_2=0\) in (5)

aus (3) und (4) \(\langle 11;10|11\rangle\equiv \text{positiv} \equiv \langle 11;0-1|1-1\rangle\)


\(\Rightarrow |1 -1\rangle = \frac{1}{\sqrt 2}[ |11;0-1\rangle - |11;-10\rangle]\)

mit \(j=2,m=1,m_1=-1, m_2=1\) in (5) \(\sqrt{3\cdot 2}\langle 11;-11|20\rangle = \sqrt{2} \underbrace{\langle 11;01|21\rangle}_{\frac 1 {\sqrt{2}}}\)
\(\langle 11;-11|20\rangle = \langle 11;1-1|20\rangle= \frac 1 {\sqrt 6}\)

aus (1) \(\frac{1}{6}+\frac{1}{6}+ \langle 11;00|20\rangle^2 = 1\) \(\rightarrow \langle 11;00|20\rangle=\pm 2 \frac{1}{\sqrt 6}\)

mit \(j=2,m=-1; m_1=m_2=0\) in (5) \(\sqrt{3\cdot 2}\langle 11;00|20\rangle = \sqrt{2} \underbrace{\langle 11;-10|2-1\rangle}_{\frac 1 {\sqrt{2}}}+\sqrt{2}\underbrace{\langle 11;0-1|2-1\rangle}_{\frac 1 {\sqrt{2}}}\)
\(\rightarrow \langle 11;00|20\rangle= 2 \frac{1}{\sqrt 6}\equiv \text{positiv}\)

\(\Rightarrow |2 0\rangle = \frac{1}{\sqrt 6}[|11;-1 1\rangle+|11;1-1\rangle+2|11;00\rangle]\)

Zusammenfassung der Ergebnisse:

\begin{itemize}
\item \(|00\rangle =  \frac{1}{\sqrt 3}[|11;-11\rangle + |11;1-1\rangle - |00\rangle|11;00\rangle ]\)
\item \(|1 -1\rangle = \frac{1}{\sqrt 2}[ |11;0-1\rangle - |11;-10\rangle]\)
\item \(|10\rangle = \frac{1}{\sqrt 2}[|11;-11\rangle + | 11;1-1\rangle]\) Rechnung TODO
\item \(|11\rangle = \frac{1}{\sqrt 2}[ |11;10\rangle - |11;01\rangle]\)
\item \(|2-2\rangle =|11;-1-1\rangle\)
\item \(|2 -1\rangle = \frac{1}{\sqrt 2}[|11;-1 0\rangle + |11;0-1\rangle]\)
\item \(|2 0\rangle = \frac{1}{\sqrt 6}[|11;-1 1\rangle+|11;1-1\rangle+2|11;00\rangle]\)
\item \(|21\rangle =\frac{1}{\sqrt 2}[|11;10\rangle+|11;01\rangle]\)
\item \(|22\rangle = |11;11\rangle\)
\end{itemize}

TODO Alernativ \(J_-\) auf den höchsten ket zum niedrigsten anwenden und somit ebenfalls die Werte erhalten.

\end{document}
