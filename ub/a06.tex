\documentclass[12pt,a4paper,titlepage,oneside]{report}
\usepackage[utf8]{inputenc}
%\usepackage{ngerman}
\usepackage{amsmath}                % brauche ich um dir Formel zu umrahmen.
\usepackage{amsfonts}                % brauche ich für die Mengensymbole
\usepackage{graphicx}

\begin{document}
\section{Aufgabe 6: Translationsoperator}

Untersuchen Sie den Translationsoperator

$$ \mathcal T(\vec l) = exp\left( -\frac i \hbar \vec p\cdot \vec l
\right) $$

(a) Zeigen Sie für beliebige Funktionen $F$ und $G$, die durch eine
Potenzreihe darstellbar sind, aufgrund der fundamentalen
Vertauschungsrelation zwischen Ort und Impuls die beiden Relationen

$$ [x_i, G(\vec p)]=i\hbar \frac{\partial G(\vec p)}{\partial p_i}, [p_i,F(\vec x)] = -i\hbar \frac{\partial F(\vec
  x)}{\partial x_i} $$

\underline{(LSG)}

Die Funktion $G(\vec p)$ lässt sich, da $\vec p$ ein Vektor ist und
dieser Funktion theoretisch alle mögliche mit dem Vektor passieren
kann (in Sinne einer Potenzreihe),  als
eine Summe von Kombination von Potenzen aus den Komponenten $p_i$ mit
beliebigen konstanten vorfaktoren $c$ darstellen:

$$  G(\vec p) = \sum_{\alpha,\beta,\gamma}
c_{\alpha\beta\gamma}p^\alpha_1 p^\beta_2 p^\gamma_3$$
Nun setzen wir die Potenzreihe in den Kommutator und versuchen die
Relatio in der Aufgabe zu bekommen. Wir kennen den Kommutator
zwischen Ort und Impuls $[x_i,p_i]=i\hbar\delta_{ij}$ und weitere
Beziehung erstmal ohne Beweiß $[x_i,p_i^k]=i\hbar k p_i^{(k-1)}$ ist für
die Aufgabe nützlich:

$$ [x_i,\sum_{\alpha,\beta,\gamma}
c_{\alpha\beta\gamma}p^\alpha_i p^\beta_j p^\gamma_k] = \sum_{\alpha,\beta,\gamma}
c_{\alpha\beta\gamma}\underbrace{[x_i,p^\alpha_i]}_{i\hbar \alpha
  p^{\alpha-1}_i} p^\beta_j p^\gamma_k$$
$$=  \sum_{\alpha,\beta,\gamma}
c_{\alpha\beta\gamma}i\hbar \alpha p^{\alpha-1}_i p^\beta_j
p^\gamma_k$$
Das sieht genau nach einer Ableitung von $p_i$ nach $\alpha$ aus

$$\Rightarrow \sum_{\alpha,\beta,\gamma}
c_{\alpha\beta\gamma}i\hbar \alpha p^{\alpha-1}_i p^\beta_j
p^\gamma_k =i\hbar \frac{\partial G(\vec p)}{\partial p_i} =  [x_i,
G(\vec p)]$$

Beweiß der Beziehung $[x_i,p_i^k]=i\hbar k p_i^{(k-1)}$:

1. Beweiß durch Induktion
Erster Schritt wähle $k=1$, Einsetzen ergibt die bekannte Orts-Impuls Relation:
$[x_i,p_i^k]=i\hbar k$;  Zweiter Schritt:

\begin{align} [x_i,p_i^{k+1}] &= x_ip_i^{k+1}-p_i^{k+1}x_i \\
&= x_ip_i^{k+1}-p_i\overbrace{(p_i^kx_i)}^{x_ip_i^k-i\hbar k p_i^{k-1}}\\
&=x_ip_i^{k+1}-p_ix_ip_i^k+i\hbar k p_i^{k}\\
&=x_ip_ip_i^{k}-p_ix_ip_i^k+i\hbar k p_i^{k}\\
&=(\overbrace{x_ip_i-p_ix_i}^{=[x_i,p_i]=i\hbar}+i\hbar k )p_i^{k}\\
&=i\hbar (k+1) p_i^{k} 
\end{align}


2. Beweiß durch die Kommutatorbeziehung 

\begin{align} [A,BC]&=ABC-BCA-BAC+BAC\\
&=(AB-BA)C+B(AC-CA)\\
&=[A,B]C+B[A,C]
\end{align}


Hier sei zur Vereinfachung $x_i=x$ und $p_i=p$ angenommen:
\begin{align} [x,p^k]&=[x,p\cdot p^{k-1}]\\
&=[x,p]p^{k-1}+p[x,p^{k-1}]\\
&=[x,p]p^{k-1}+p[x,p\cdot p^{k-2}]\\
&=[x,p]p^{k-1}+p([x,p]p^{k-2}+p[x,p^{k-2}]) \\
&=[x,p]p^{k-1}+[x,p]p^{k-1}+p^2[x,p^{k-2}]) \\
&=\underbrace{[x,p]p^{k-1}+[x,p]p^{k-1}+...+[x,p]p^{k-1}}_{=\times
  k}+\underbrace{p^k[x,p^{0}]}_{[x,1]=0}\\
&=k[x,p]p^{k-1} \\
&=i\hbar kp^{k-1}
\end{align}








\end{document}
