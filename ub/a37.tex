\documentclass[10pt,a4paper,oneside,fleqn]{article}
\usepackage{geometry}
\geometry{a4paper,left=20mm,right=20mm,top=1cm,bottom=2cm}
\usepackage[utf8]{inputenc}
%\usepackage{ngerman}
\usepackage{amsmath}                % brauche ich um dir Formel zu umrahmen.
\usepackage{amsfonts}                % brauche ich für die Mengensymbole
\usepackage{graphicx}
\setlength{\parindent}{0px}
\setlength{\mathindent}{10mm}
\usepackage{bbold}                    %brauche ich für die doppel Zahlen Darstellung (Einheitsmatrix z.B)
\usepackage{dsfont}          %F�r den Einheitsoperator \mathds 1


\usepackage{color}
\usepackage{titlesec} %sudo apt-get install texlive-latex-extra

\definecolor{darkblue}{rgb}{0.1,0.1,0.55}
\definecolor{verydarkblue}{rgb}{0.1,0.1,0.35}
\definecolor{darkred}{rgb}{0.55,0.2,0.2}

%hyperref Link color
\usepackage[colorlinks=true,
        linkcolor=darkblue,
        citecolor=darkblue,
        filecolor=darkblue,
        pagecolor=darkblue,
        urlcolor=darkblue,
        bookmarks=true,
        bookmarksopen=true,
        bookmarksopenlevel=3,
        plainpages=false,
        pdfpagelabels=true]{hyperref}

\titleformat{\chapter}[display]{\color{darkred}\normalfont\huge\bfseries}{\chaptertitlename\
\thechapter}{20pt}{\Huge}

\titleformat{\section}{\color{darkblue}\normalfont\Large\bfseries}{\thesection}{1em}{}
\titleformat{\subsection}{\color{verydarkblue}\normalfont\large\bfseries}{\thesubsection}{1em}{}

% Notiz Box
\usepackage{fancybox}
\newcommand{\notiz}[1]{\vspace{5mm}\ovalbox{\begin{minipage}{1\textwidth}#1\end{minipage}}\vspace{5mm}}

\usepackage{cancel}
\setcounter{secnumdepth}{3}
\setcounter{tocdepth}{3}





%-------------------------------------------------------------------------------
%Diff-Makro:
%Das Diff-Makro stellt einen Differentialoperator da.
%
%Benutzung:
% \diff  ->  d
% \diff f  ->  df
% \diff^2 f  ->  d^2 f
% \diff_x  ->  d/dx
% \diff^2_x  ->  d^2/dx^2
% \diff f_x  ->  df/dx
% \diff^2 f_x  ->  d^2f/dx^2
% \diff^2{f(x^5)}_x  ->  d^2(f(x^5))/dx^2
%
%Ersetzt man \diff durch \pdiff, so wird der partieller
%Differentialoperator dargestellt.
%
\makeatletter
\def\diff@n^#1{\@ifnextchar{_}{\diff@n@d^#1}{\diff@n@fun^#1}}
\def\diff@n@d^#1_#2{\frac{\textrm{d}^#1}{\textrm{d}#2^#1}}
\def\diff@n@fun^#1#2{\@ifnextchar{_}{\diff@n@fun@d^#1#2}{\textrm{d}^#1#2}}
\def\diff@n@fun@d^#1#2_#3{\frac{\textrm{d}^#1 #2}{\textrm{d}#3^#1}}
\def\diff@one@d_#1{\frac{\textrm{d}}{\textrm{d}#1}}
\def\diff@one@fun#1{\@ifnextchar{_}{\diff@one@fun@d #1}{\textrm{d}#1}}
\def\diff@one@fun@d#1_#2{\frac{\textrm{d}#1}{\textrm{d}#2}}
\newcommand*{\diff}{\@ifnextchar{^}{\diff@n}
  {\@ifnextchar{_}{\diff@one@d}{\diff@one@fun}}}
%
%Partieller Diff-Operator.
\def\pdiff@n^#1{\@ifnextchar{_}{\pdiff@n@d^#1}{\pdiff@n@fun^#1}}
\def\pdiff@n@d^#1_#2{\frac{\partial^#1}{\partial#2^#1}}
\def\pdiff@n@fun^#1#2{\@ifnextchar{_}{\pdiff@n@fun@d^#1#2}{\partial^#1#2}}
\def\pdiff@n@fun@d^#1#2_#3{\frac{\partial^#1 #2}{\partial#3^#1}}
\def\pdiff@one@d_#1{\frac{\partial}{\partial #1}}
\def\pdiff@one@fun#1{\@ifnextchar{_}{\pdiff@one@fun@d #1}{\partial#1}}
\def\pdiff@one@fun@d#1_#2{\frac{\partial#1}{\partial#2}}
\newcommand*{\pdiff}{\@ifnextchar{^}{\pdiff@n}
  {\@ifnextchar{_}{\pdiff@one@d}{\pdiff@one@fun}}}
\makeatother
%
%Das gleich nur mit etwas andere Syntax. Die Potenz der Differentiation wird erst
%zum Schluss angegeben. Somit lautet die Syntax:
%
% \diff_x^2  ->  d^2/dx^2
% \diff f_x^2  ->  d^2f/dx^2
% \diff{f(x^5)}_x^2  ->  d^2(f(x^5))/dx^2
% Ansonsten wie Oben.
%
%Ersetzt man \diff durch \pdiff, so wird der partieller
%Differentialoperator dargestellt.
%
%\makeatletter
%\def\diff@#1{\@ifnextchar{_}{\diff@fun#1}{\textrm{d} #1}}
%\def\diff@one_#1{\@ifnextchar{^}{\diff@n{#1}}%
%  {\frac{\textrm d}{\textrm{d} #1}}}
%\def\diff@fun#1_#2{\@ifnextchar{^}{\diff@fun@n#1_#2}%
%  {\frac{\textrm d #1}{\textrm{d} #2}}}
%\def\diff@n#1^#2{\frac{\textrm d^#2}{\textrm{d}#1^#2}}
%\def\diff@fun@n#1_#2^#3{\frac{\textrm d^#3 #1}%
%  {\textrm{d}#2^#3}}
%\def\diff{\@ifnextchar{_}{\diff@one}{\diff@}}
%\newcommand*{\diff}{\@ifnextchar{_}{\diff@one}{\diff@}}
%
%Partieller Diff-Operator.
%\def\pdiff@#1{\@ifnextchar{_}{\pdiff@fun#1}{\partial #1}}
%\def\pdiff@one_#1{\@ifnextchar{^}{\pdiff@n{#1}}%
%  {\frac{\partial}{\partial #1}}}
%\def\pdiff@fun#1_#2{\@ifnextchar{^}{\pdiff@fun@n#1_#2}%
%  {\frac{\partial #1}{\partial #2}}}
%\def\pdiff@n#1^#2{\frac{\partial^#2}{\partial #1^#2}}
%\def\pdiff@fun@n#1_#2^#3{\frac{\partial^#3 #1}%
%  {\partial #2^#3}}
%\newcommand*{\pdiff}{\@ifnextchar{_}{\pdiff@one}{\pdiff@}}
%\makeatother

%-------------------------------------------------------------------------------
%%Nützliche Makros um in der Quantenmechanik Bras, Kets und das Skalarprodukt
%%zwischen den beiden darzustellen.
%%Benutzung:
%% \bra{x}  ->    < x |
%% \ket{x}  ->    | x >
%% \braket{x}{y} ->   < x | y >



\newcommand\bra[1]{\left\langle #1 \right|}
\newcommand\ket[1]{\left| #1 \right\rangle}
\newcommand\braket[2]{%
 \left\langle \vphantom{#2} #1%
   \middle|%
   \vphantom{#1} #2\right\rangle}%

%-------------------------------------------------------------------------------
%%Aus dem Buch:
%%Titel:  Latex in Naturwissenschaften und Mathematik
%%Autor:  Herbert Voß
%%Verlag: Franzis Verlag, 2006
%%ISBN:   3772374190, 9783772374197
%%
%%Hier werden drei Makros definiert:\mathllap, \mathclap und \mathrlap, welche
%%analog zu den aus Latex bekannten \rlap und \llap arbeiten, d.h. selbst
%%keinerlei horizontalen Platz benötigen, aber dennoch zentriert zum aktuellen
%%Punkt erscheinen.

\newcommand*\mathllap{\mathstrut\mathpalette\mathllapinternal}
\newcommand*\mathllapinternal[2]{\llap{$\mathsurround=0pt#1{#2}$}}
\newcommand*\clap[1]{\hbox to 0pt{\hss#1\hss}}
\newcommand*\mathclap{\mathpalette\mathclapinternal}
\newcommand*\mathclapinternal[2]{\clap{$\mathsurround=0pt#1{#2}$}}
\newcommand*\mathrlap{\mathpalette\mathrlapinternal}
\newcommand*\mathrlapinternal[2]{\rlap{$\mathsurround=0pt#1{#2}$}}

%%Das Gleiche nur mit \def statt \newcommand.
%\def\mathllap{\mathpalette\mathllapinternal}
%\def\mathllapinternal#1#2{%
%  \llap{$\mathsurround=0pt#1{#2}$}% $
%}
%\def\clap#1{\hbox to 0pt{\hss#1\hss}}
%\def\mathclap{\mathpalette\mathclapinternal}
%\def\mathclapinternal#1#2{%
%  \clap{$\mathsurround=0pt#1{#2}$}%
%}
%\def\mathrlap{\mathpalette\mathrlapinternal}
%\def\mathrlapinternal#1#2{%
%  \rlap{$\mathsurround=0pt#1{#2}$}% $
%}

%-------------------------------------------------------------------------------
%%Hier werden zwei neue Makros definiert \overbr und \underbr welche analog zu
%%\overbrace und \underbrace funktionieren jedoch die Gleichung nicht
%%'zerreißen'. Dies wird ermöglicht durch das \mathclap Makro.

\def\overbr#1^#2{\overbrace{#1}^{\mathclap{#2}}}
\def\underbr#1_#2{\underbrace{#1}_{\mathclap{#2}}}


\begin{document}
\section*{Aufgabe 37: Störungsrechnung}

Der Hamilton-Operator eines 3-Zustands-Systems ist durch \(H=H_0+V\) gegeben, wobei

\[H_0 = \hbar\omega \begin{pmatrix}2&0&0\\ 0&5&0\\ 0&0&6\\ \end{pmatrix}, \qquad V = \begin{pmatrix}0&\alpha&\beta\\ \alpha&0&0\\ \beta&0&0\\ \end{pmatrix} \]

Darin ist V als Störung aufzufassen, d.h, \(\alpha,\beta << \hbar\omega\). Hier sind beide Operatoren in der Basis der ungestörten Energieeigenzustände \(|1\rangle^{(0)}, |2\rangle^{(0)},|3\rangle^{(0)} \) zu den Eigenwerten \(E_1 = 2\hbar\omega, E_2 = 5\hbar\omega, E_3 = 6\hbar\omega\) geschrieben.

\begin{enumerate}
\item[(a)] Berechnen Sie den Grundzustand in 1. Ordnung und die Grundzustandsenergie in 2. Ordnung zeitunabhängiger Störungsrechnung.
\item[(b)] Die Störung \(V\) sei jetzt nur für \(0<t<t_1\) eingeschaltet. Das System befinde sich zum Zeitpunkt \(t=0\) im Zustand \(|2\rangle^{(0)} \). Berechnen Sie den Wahrscheinlichkeit, das System zu einem Zeitpunkt \(t_{obs}>t_1\) im Zustand  \(|1\rangle^{(0)} \) vorzufinden in erster Ordnung zeitabhängiger Störungstheorie.

\item[(c)] Sei nun wieder \(V(t) = const = V\). Das ungestörte System habe jedoch einen entarteten Grundzustand: wir setzen \(\langle 2|H_0|2\rangle =2\hbar\omega\). Wie lauten unter Berücksichtigung von V bis zur ersten Ordnung der entarteten Störungsrechnung die Energien und Energieeigenzustände?
\end{enumerate}




\subsection*{LSG a)}

Für den Eigenzustand in der Störungsrechnung gilt:

\[ |n\rangle = |n\rangle ^{(0)}+\sum_{k\neq n} \frac{\langle k|V|n\rangle }{E_n^{(0)}- E_k^{(0)}}|k\rangle ^{(0)} + ... \]

Für den Grundzustand bis erster Ordnung:

\begin{align} 
|1\rangle &= |1\rangle ^{(0)}+\frac{\langle 2|V|1\rangle }{E_1^{(0)}- E_2^{(0)}}|2\rangle ^{(0)} +  \frac{\langle 3|V|1\rangle }{E_1^{(0)}- E_3^{(0)}}|3\rangle ^{(0)} \\
&= |1\rangle ^{(0)}+\frac{\alpha }{\hbar\omega 2- \hbar\omega 5}|2\rangle ^{(0)} +  \frac{\beta }{\hbar\omega 2- \hbar\omega 6}|3\rangle ^{(0)} \\
&= |1\rangle ^{(0)}-\frac{\alpha }{\hbar\omega 3}|2\rangle ^{(0)} -  \frac{\beta }{\hbar\omega 4}|3\rangle ^{(0)} \\
\end{align}

Für die Energie bis zu der zweiter Ordnung gilt:

\begin{align}
E_n &= E_n^{(0)}+ E_n^{(1)}+E_n^{(2)} \\
&=E_n^{(0)} + \langle n|V|n\rangle + \sum_{k\neq n}\frac{|\langle
  k|V|n\rangle|^2}{E_n^{(0)}- E_k^{(0)}}
\end{align}

Da die Grundzustandsenergie per Definition der kleinste Energiezustand ist, d.h in dem Beispiel \(E_1=\hbar\omega\cdot 2\):


\[E_1 = \hbar\omega 2 + 0 +  \frac{|\langle 2|V|1\rangle|^2}{\hbar\omega 2
  -\hbar\omega 5  } + \frac{|\langle 3|V|1\rangle|^2}{\hbar\omega 2
  -\hbar\omega 6  } \]



\subsection*{LSG b)}

Für die Berechnung der Wahrscheinlichkeiten ist es leichter/zweckmäßig das Sytem im Wechselwirkungsbild zu betrachten, da dort die Statische Komponente \(H_0\) keine rolle spielt:

\[i\hbar \frac{\partial}{\partial}|\alpha,t_0;t\rangle_I = V_I|\alpha,t_0;t\rangle_I\]

Dies lässt sich weiter umformen zu:

\[ i\hbar \frac{\partial}{\partial} U_I(t,t_0) = V_I U(t,t_0)  \]

mit \(U_I\) als einen unitären Zeitentwicklungsoperator \(|\alpha,t_0;t\rangle_I = U_I(t,t_0) |\alpha,t_0;t\rangle_I \) der den \(t_0\)-Zustand zu einen beliebigen \(t\) Zeitzustand 'transformiert'. Integration der obigen gleichung über die Zeit ergibt eine rekursive formel:

\[U_I^{(n)}(t,t_0) = 1-\frac{i}{\hbar}\int_{t_0}^t V_IU_I^{(n-1)}(t,t_0)dt\]

Der unitäre Zeitentwicklungsoperator \(U_I\) soll den Zustand nicht verändert wenn keine Zeitdifferenz vorhanden ist, sprich \(U_I(t_0,t_0)=1\). Zeitabhängige Störung 0.Ordnung ist demnach:

\[U_I^{(0)}(t,t_0)=1\]


und 1. Ordnung ergibt sich aus dem Einsetzen in die Rekursive Formel:

\[U_I^{(1)}(t,t_0) = 1-\frac{i}{\hbar}\int_{t_0}^t V_IU_I^{(0)}(t,t_0)dt = 1-\frac{i}{\hbar}\int_{t_0}^t V_I dt\]

Nun werden die Zeitabhängigen Matrixelemente von \(U_I\) benötigt. Ein Zustand im WW-Bild lässt sich offenbar in Abhängigkeit der Zeitabhängigen Matrixelemente \(c_n(t)\) entwickeln:

\begin{align}
|i,t_0,t\rangle_I &= U_I(t,t_0)|i\rangle \\
&=\mathbb 1\cdot U_I(t,t_0)|i\rangle \\
&=\sum_n |n\rangle \langle n| U_I(t,t_0)|i\rangle \\
&= \sum_n c_n(t) |n\rangle
\end{align}


Diese Koeffizienten \(c_n(t)\equiv \langle n| U_I(t,t_0)|i\rangle \) bestimmen den (Übergangs)-Zustand des Systems und deren Betragsquadrat \(|c_n(t)|^2\) die Übergangs-Warhscheinlichkeit. In 1.Ordnung Störungsrechnung werden zunächst die Matrixelemente von \(U_I^{(1)}(t,t_0)\) berechnet:

\[ c_n^{(1)}(t) =  \langle n| U_I^{(1)}(t,t_0)|i\rangle = \langle n|i\rangle -  \frac{i}{\hbar}\int_{t_0}^t \langle n|V_I|i\rangle dt \]

Der Störungsoperator \(V_I\) ist von der Zeit abhängig, um das Integral zu berechnen wird hier nun das Schrödinger Bild gewählt:

\begin{align}
c_n^{(1)}(t) &= \delta_{ni} -  \frac{i}{\hbar}\int_{t_0}^{t'} \langle n|e^{\frac{i}{\hbar}E_nt'} V_S e^{-\frac{i}{\hbar}E_it'} |i\rangle dt' \\
&= \delta_{ni} -  \frac{i}{\hbar}\langle n|V_S  |i\rangle \int_{t_0}^{t'} dt' e^{\frac{i}{\hbar}(E_n-E_i)t'} \\
&= \delta_{ni} -  \frac{i}{\hbar}\langle n|V_S  |i\rangle \int_{t_0}^{t'} dt e^{i\omega_{ni}t'}
\end{align}

Mit \(\omega_{ni} =\frac{E_n-E_i}{\hbar} \). 

In unserem Fall befindet sich das System im Zustand \(|2\rangle^{(0)} \equiv |i\rangle  \). Es soll die Warscheinlichkeit im Zustand \(|1\rangle^{(0)} \equiv |n\rangle  \) bestimmt werden. Konkret heißt das:

\[ c_1^{(1)}(t) =  \langle 1| U_I^{(1)}(t,t_0)|2\rangle = 0 - \frac{i}{\hbar}\langle 1|V_S  |2\rangle \int_{t_0}^{t'} dt' e^{i\omega_{12}t'} = - \frac{i}{\hbar}\alpha  \int_{t_0}^{t'} dt' e^{i\omega(2-5)t'}\]

\[ c_1^{(1)}(t) =  - \frac{i}{\hbar}\alpha   \left[\frac{e^{-i3\omega t'}}{-i3\omega}\right]^{t}_{0} = - \frac{i}{\hbar}\alpha (\frac{e^{-i3\omega t}}{-i3\omega}-\frac{1}{-i3\omega}) = \frac{\alpha}{3\hbar\omega} (e^{-i3\omega t}-1) \]

Die Übergangswahrscheinlichkeit ist das Betragsquadrat des Matrixelements:

\begin{align} 
p_{2 \to 1} &= |c_1^{(1)}(t)|^2 = \frac{\alpha^2}{9\hbar^2\omega^2} (e^{i3\omega t}-1)(e^{-i3\omega t}-1) \\
&= \frac{\alpha^2}{9\hbar^2\omega^2} (1-e^{i3\omega t}-e^{-i3\omega t}+1) \\
&= \frac{\alpha^2}{9\hbar^2\omega^2} (2-(e^{i3\omega t}+e^{-i3\omega t})) \\
&= \frac{2\alpha^2}{9\hbar^2\omega^2} (1-cos(3\omega t))
\end{align}

 
\subsection*{LSG c)}

Der \(H_0\) Operator hat nun 2 Zustände zu einem Energiewert \(\hbar\omega\cdot 2\) \(\Rightarrow \) entartet.

\[H_0 = \hbar\omega \begin{pmatrix}2&0&0\\ 0&2&0\\ 0&0&6\\ \end{pmatrix}, \qquad V = \begin{pmatrix}0&a&b\\ a&0&0\\ b&0&0\\ \end{pmatrix} \]

und \(V(t) = const = V\), Aus unerfindlichen Gründen geht \(\alpha\) in a und \(\beta\) in b über. Der wahre Grund ist, dass \(\alpha,\beta\) sich besser als Indizies für die Herleitung im Entarteten Unterraum eignen. Nein, der noch wahrere Grund ist, dass ich die Indizies so gewählt habe und erst zum Schluss bemerkte, dass es mit den Konstanten in der Matrix kollidiert.


Die Energie \(E_n = E_2 = 2\hbar\omega\) ist zweifach entartet, dazu gehören die zwei Zustände im nicht entarteten Fall \(|1\rangle^{(0)} \)  \(|2\rangle^{(0)} \). Als Eigenwert Gleichung so Ausgedrückt:

\[H_0 |1\rangle^{(0)} = E_2 |1\rangle^{(0)} \]
\[H_0 |2\rangle^{(0)} = E_2 |2\rangle^{(0)} \]

Oder wenn man den Entarteten Unterraum separat betrachtet gilt für \(\alpha=1,2\) allgemein:

\[H_0 |\alpha\rangle^{(0)} = E_\alpha |\alpha\rangle^{(0)} \]

Nach dem Überlagerungsprinzip, wonach die Summe der Zustandsfunktionen ebenso eine Lösung für die Eigenwertgleichung erfüllt:

\[|n_\alpha\rangle^{(0)} = \sum_\alpha c_\alpha |\alpha\rangle ^{(0)}\]

\[H_0 |n_\alpha \rangle^{(0)} = E_\alpha |n_\alpha\rangle^{(0)} = E_\alpha \mathbb 1 |n_\alpha\rangle^{(0)} =  E_\alpha \sum_\alpha |\alpha\rangle^{(0)} \cdot^{(0)}\langle \alpha |n_\alpha\rangle^{(0)} = E_\alpha \sum_\alpha c_\alpha|\alpha\rangle ^{(0)} \]
Wobei die Koeffizienten \(c_\alpha =  ^{(0)} \langle \alpha |n_\alpha\rangle^{(0)}\)

Nun benötigen wir aus der Störungsrechnung die Gleichung mit der ersten Potenz \(\lambda^1\)

\[(H_0-E_n^{(0)})|n_\alpha\rangle ^{(1)} = (E_n^{(1)}-V)|n_\alpha\rangle^{(0)}\]

Multipliziert diese Gleichung mit \(^{(0)}\langle \alpha|\) sieht man:

\begin{align}
^{(0)}\langle \alpha|(H_0-E_n^{(0)})|n_\alpha\rangle ^{(1)} &= ^{(0)}\langle \alpha|(E_n^{(1)}-V)|n_\alpha\rangle^{(0)}\\
^{(0)}\langle \alpha|\underbrace{(E_0^{(0)}-E_n^{(0)})}_{=0}|n_\alpha\rangle ^{(1)} &=^{(0)}\langle \alpha| (E_n^{(1)}-V)|n_\alpha\rangle^{(0)}
\end{align}

mit \(|n_\beta \rangle^{(0)} = \sum_\beta c_\beta |\beta\rangle ^{(0)} \) und \(\langle \alpha|n_\beta\rangle = c_\beta\delta_{\alpha\beta}\)

\[\Rightarrow  ^{(0)} \langle \alpha|V-E_n^{(1)}|n_\beta\rangle^{(0)} = 0 \]
\[  ^{(0)} \langle \alpha|V|n_\alpha\rangle^{(0)} -E_n^{(1)} \langle \alpha|n_\beta\rangle^{(0)}  = 0 \]
\[  ^{(0)} \langle \alpha|V|n_\alpha\rangle^{(0)} -E_n^{(1)} c_\alpha  \delta_{\alpha\beta}  = 0 \]


Ersetze nun \(|n_\beta\rangle^{(0)}\) mit der Summe der Entarteten untervektoren als \(|n_\beta\rangle^{(0)} = \sum_\beta |\beta\rangle \). Der Index \(\beta\) läuft genau wie \(\alpha\) über alle entarteten Zustände

\[\sum_\beta V_{\alpha\beta}-E_n^{(1)}c_\alpha  \delta_{\alpha\beta} = 0\]
Das ist offensichtlich eine Sekulärgleichung im entarteten Unterraum mit einem Ausschnitt aus der Störungsmatrix (in unserem Fall):

\[V_{\alpha\beta} = \begin{pmatrix}0&a\\ a&0 \end{pmatrix} \]

Oder als Eigenwertgleichung:


\[V|\alpha\rangle ^{(0)} = E_n^{(1)}c_\alpha|\alpha\rangle ^{(0)}  \]

\[V|\alpha\rangle ^{(0)} = E_\alpha^{(1)} |\alpha\rangle ^{(0)}  \]

Lösen der Determinante:

\[\begin{vmatrix} 0-\lambda&a\\ a& 0-\lambda  \end{vmatrix} = \lambda^2-a^2 = 0 \]

\[\Rightarrow \lambda =\pm a\]

ist das Ergebniss:  \(E_\alpha^{(1)} = \pm a \).  Das bedeutet, dass die Lösung dieser Eigenwertgleichung im entarteten Unterraum der Störungsmatrix, uns die Energie-Eigenwerte in 1 Ordnung der Störung liefert. Die gesamte Energie (bis zur 1 Ordnung) sieht dann so aus:

\[E_n^{(ges)} = E_n^{(0)}+E_n^{(1)}\]

Im entarteten Unterraum dann so:

\[E_\alpha^{(ges)} = E_\alpha^{(0)}+E_\alpha^{(1)} =E_\alpha^{(0)}\pm a \]

Die Gesamtenergie ist dank der konstanten Störung nicht mehr entartet. Die Störung hebt offensichtlich (i.a. teilweise oder ganz) die Störung auf.

Für die Eigenvektoren im entarteten Unterraum ergibt sich:

\[|+a\rangle :\]

\[\begin{pmatrix}0&a\\a&0\end{pmatrix} \begin{pmatrix}x\\y\end{pmatrix}=a\begin{pmatrix}x\\y\end{pmatrix} \rightarrow x=y\]

\[\Rightarrow |+a\rangle = \frac{1}{\sqrt{2}} \begin{pmatrix}1\\1\end{pmatrix} = \frac{1}{\sqrt{2}} (|1\rangle^{(0)}+ |2\rangle^{(0)}) \]


\[|-a\rangle :\]

\[\begin{pmatrix}0&a\\a&0\end{pmatrix} \begin{pmatrix}x\\y\end{pmatrix}=-a\begin{pmatrix}x\\y\end{pmatrix} \rightarrow x=-y\]

\[\Rightarrow |-a\rangle = \frac{1}{\sqrt{2}} \begin{pmatrix}1\\-1\end{pmatrix} = \frac{1}{\sqrt{2}} (|1\rangle^{(0)}- |2\rangle^{(0)})  \]





\end{document}
