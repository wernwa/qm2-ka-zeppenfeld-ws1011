\documentclass[10pt,a4paper,oneside,fleqn]{article}
\usepackage{geometry}
\geometry{a4paper,left=20mm,right=20mm,top=1cm,bottom=2cm}
\usepackage[utf8]{inputenc}
%\usepackage{ngerman}
\usepackage{amsmath}                % brauche ich um dir Formel zu umrahmen.
\usepackage{amsfonts}                % brauche ich für die Mengensymbole
\usepackage{graphicx}
\setlength{\parindent}{0px}
\setlength{\mathindent}{10mm}
\usepackage{bbold}                    %brauche ich für die doppel Zahlen Darstellung (Einheitsmatrix z.B)
\usepackage{dsfont}          %F�r den Einheitsoperator \mathds 1


\usepackage{color}
\usepackage{titlesec} %sudo apt-get install texlive-latex-extra

\definecolor{darkblue}{rgb}{0.1,0.1,0.55}
\definecolor{verydarkblue}{rgb}{0.1,0.1,0.35}
\definecolor{darkred}{rgb}{0.55,0.2,0.2}

%hyperref Link color
\usepackage[colorlinks=true,
        linkcolor=darkblue,
        citecolor=darkblue,
        filecolor=darkblue,
        pagecolor=darkblue,
        urlcolor=darkblue,
        bookmarks=true,
        bookmarksopen=true,
        bookmarksopenlevel=3,
        plainpages=false,
        pdfpagelabels=true]{hyperref}

\titleformat{\chapter}[display]{\color{darkred}\normalfont\huge\bfseries}{\chaptertitlename\
\thechapter}{20pt}{\Huge}

\titleformat{\section}{\color{darkblue}\normalfont\Large\bfseries}{\thesection}{1em}{}
\titleformat{\subsection}{\color{verydarkblue}\normalfont\large\bfseries}{\thesubsection}{1em}{}

% Notiz Box
\usepackage{fancybox}
\newcommand{\notiz}[1]{\vspace{5mm}\ovalbox{\begin{minipage}{1\textwidth}#1\end{minipage}}\vspace{5mm}}

\usepackage{cancel}
\setcounter{secnumdepth}{3}
\setcounter{tocdepth}{3}





%-------------------------------------------------------------------------------
%Diff-Makro:
%Das Diff-Makro stellt einen Differentialoperator da.
%
%Benutzung:
% \diff  ->  d
% \diff f  ->  df
% \diff^2 f  ->  d^2 f
% \diff_x  ->  d/dx
% \diff^2_x  ->  d^2/dx^2
% \diff f_x  ->  df/dx
% \diff^2 f_x  ->  d^2f/dx^2
% \diff^2{f(x^5)}_x  ->  d^2(f(x^5))/dx^2
%
%Ersetzt man \diff durch \pdiff, so wird der partieller
%Differentialoperator dargestellt.
%
\makeatletter
\def\diff@n^#1{\@ifnextchar{_}{\diff@n@d^#1}{\diff@n@fun^#1}}
\def\diff@n@d^#1_#2{\frac{\textrm{d}^#1}{\textrm{d}#2^#1}}
\def\diff@n@fun^#1#2{\@ifnextchar{_}{\diff@n@fun@d^#1#2}{\textrm{d}^#1#2}}
\def\diff@n@fun@d^#1#2_#3{\frac{\textrm{d}^#1 #2}{\textrm{d}#3^#1}}
\def\diff@one@d_#1{\frac{\textrm{d}}{\textrm{d}#1}}
\def\diff@one@fun#1{\@ifnextchar{_}{\diff@one@fun@d #1}{\textrm{d}#1}}
\def\diff@one@fun@d#1_#2{\frac{\textrm{d}#1}{\textrm{d}#2}}
\newcommand*{\diff}{\@ifnextchar{^}{\diff@n}
  {\@ifnextchar{_}{\diff@one@d}{\diff@one@fun}}}
%
%Partieller Diff-Operator.
\def\pdiff@n^#1{\@ifnextchar{_}{\pdiff@n@d^#1}{\pdiff@n@fun^#1}}
\def\pdiff@n@d^#1_#2{\frac{\partial^#1}{\partial#2^#1}}
\def\pdiff@n@fun^#1#2{\@ifnextchar{_}{\pdiff@n@fun@d^#1#2}{\partial^#1#2}}
\def\pdiff@n@fun@d^#1#2_#3{\frac{\partial^#1 #2}{\partial#3^#1}}
\def\pdiff@one@d_#1{\frac{\partial}{\partial #1}}
\def\pdiff@one@fun#1{\@ifnextchar{_}{\pdiff@one@fun@d #1}{\partial#1}}
\def\pdiff@one@fun@d#1_#2{\frac{\partial#1}{\partial#2}}
\newcommand*{\pdiff}{\@ifnextchar{^}{\pdiff@n}
  {\@ifnextchar{_}{\pdiff@one@d}{\pdiff@one@fun}}}
\makeatother
%
%Das gleich nur mit etwas andere Syntax. Die Potenz der Differentiation wird erst
%zum Schluss angegeben. Somit lautet die Syntax:
%
% \diff_x^2  ->  d^2/dx^2
% \diff f_x^2  ->  d^2f/dx^2
% \diff{f(x^5)}_x^2  ->  d^2(f(x^5))/dx^2
% Ansonsten wie Oben.
%
%Ersetzt man \diff durch \pdiff, so wird der partieller
%Differentialoperator dargestellt.
%
%\makeatletter
%\def\diff@#1{\@ifnextchar{_}{\diff@fun#1}{\textrm{d} #1}}
%\def\diff@one_#1{\@ifnextchar{^}{\diff@n{#1}}%
%  {\frac{\textrm d}{\textrm{d} #1}}}
%\def\diff@fun#1_#2{\@ifnextchar{^}{\diff@fun@n#1_#2}%
%  {\frac{\textrm d #1}{\textrm{d} #2}}}
%\def\diff@n#1^#2{\frac{\textrm d^#2}{\textrm{d}#1^#2}}
%\def\diff@fun@n#1_#2^#3{\frac{\textrm d^#3 #1}%
%  {\textrm{d}#2^#3}}
%\def\diff{\@ifnextchar{_}{\diff@one}{\diff@}}
%\newcommand*{\diff}{\@ifnextchar{_}{\diff@one}{\diff@}}
%
%Partieller Diff-Operator.
%\def\pdiff@#1{\@ifnextchar{_}{\pdiff@fun#1}{\partial #1}}
%\def\pdiff@one_#1{\@ifnextchar{^}{\pdiff@n{#1}}%
%  {\frac{\partial}{\partial #1}}}
%\def\pdiff@fun#1_#2{\@ifnextchar{^}{\pdiff@fun@n#1_#2}%
%  {\frac{\partial #1}{\partial #2}}}
%\def\pdiff@n#1^#2{\frac{\partial^#2}{\partial #1^#2}}
%\def\pdiff@fun@n#1_#2^#3{\frac{\partial^#3 #1}%
%  {\partial #2^#3}}
%\newcommand*{\pdiff}{\@ifnextchar{_}{\pdiff@one}{\pdiff@}}
%\makeatother

%-------------------------------------------------------------------------------
%%Nützliche Makros um in der Quantenmechanik Bras, Kets und das Skalarprodukt
%%zwischen den beiden darzustellen.
%%Benutzung:
%% \bra{x}  ->    < x |
%% \ket{x}  ->    | x >
%% \braket{x}{y} ->   < x | y >



\newcommand\bra[1]{\left\langle #1 \right|}
\newcommand\ket[1]{\left| #1 \right\rangle}
\newcommand\braket[2]{%
 \left\langle \vphantom{#2} #1%
   \middle|%
   \vphantom{#1} #2\right\rangle}%

%-------------------------------------------------------------------------------
%%Aus dem Buch:
%%Titel:  Latex in Naturwissenschaften und Mathematik
%%Autor:  Herbert Voß
%%Verlag: Franzis Verlag, 2006
%%ISBN:   3772374190, 9783772374197
%%
%%Hier werden drei Makros definiert:\mathllap, \mathclap und \mathrlap, welche
%%analog zu den aus Latex bekannten \rlap und \llap arbeiten, d.h. selbst
%%keinerlei horizontalen Platz benötigen, aber dennoch zentriert zum aktuellen
%%Punkt erscheinen.

\newcommand*\mathllap{\mathstrut\mathpalette\mathllapinternal}
\newcommand*\mathllapinternal[2]{\llap{$\mathsurround=0pt#1{#2}$}}
\newcommand*\clap[1]{\hbox to 0pt{\hss#1\hss}}
\newcommand*\mathclap{\mathpalette\mathclapinternal}
\newcommand*\mathclapinternal[2]{\clap{$\mathsurround=0pt#1{#2}$}}
\newcommand*\mathrlap{\mathpalette\mathrlapinternal}
\newcommand*\mathrlapinternal[2]{\rlap{$\mathsurround=0pt#1{#2}$}}

%%Das Gleiche nur mit \def statt \newcommand.
%\def\mathllap{\mathpalette\mathllapinternal}
%\def\mathllapinternal#1#2{%
%  \llap{$\mathsurround=0pt#1{#2}$}% $
%}
%\def\clap#1{\hbox to 0pt{\hss#1\hss}}
%\def\mathclap{\mathpalette\mathclapinternal}
%\def\mathclapinternal#1#2{%
%  \clap{$\mathsurround=0pt#1{#2}$}%
%}
%\def\mathrlap{\mathpalette\mathrlapinternal}
%\def\mathrlapinternal#1#2{%
%  \rlap{$\mathsurround=0pt#1{#2}$}% $
%}

%-------------------------------------------------------------------------------
%%Hier werden zwei neue Makros definiert \overbr und \underbr welche analog zu
%%\overbrace und \underbrace funktionieren jedoch die Gleichung nicht
%%'zerreißen'. Dies wird ermöglicht durch das \mathclap Makro.

\def\overbr#1^#2{\overbrace{#1}^{\mathclap{#2}}}
\def\underbr#1_#2{\underbrace{#1}_{\mathclap{#2}}}

%couchdb db=physik
%couchdb id=qm2uba13_drehimpulse_und_oszillatoren
%couchdb tags=qm2ub
%couchdb pdflink=http://wernwa-physik-ka.googlecode.com/svn/qm2ub/ub04/a13.pdf

\begin{document}
\section*{Aufgabe 13: Drehimpulse und Oszillatoren}

Wir untersuchen ein System von zwei unabhängigen harmonischen Oszillatoren (im folgenden durch +,- benannt) mit Erzeugungs- und Vernichtungsoperatoren \(a^\dagger_\pm,a_\pm\) die die üblichen Vertauschungsrelationen erfüllen. Damit definieren wir die Operatoren
\[J_\pm = \hbar a^\dagger_\pm a_\mp, \qquad 
J_z=\frac{\hbar}{2}( a^\dagger_+ a_+-a^\dagger_- a_-), \qquad 
N=N_++N_-=a^\dagger_+ a_+ + a^\dagger_- a_-
\]

Zeigen Sie, dass die \(J_\pm,J_z\) eine Drehimpulsalgebra erfüllen, also

\[ [J_z,J_\pm] = \pm \hbar J_\pm, \qquad
[\vec J^2,J_z]=0 \qquad
[J_+,J_-]=2\hbar J_z
\]

Zeigen Sie weiterhin, dass

\[ \vec J^2 = \hbar \frac{N}{2}(\frac{N}{2}+1), \qquad (vec J^2 = J^2_x+J^2_y+J^2_z)\]

Damit müsste es also eine Zusammenhang zwischen der Besetzungszahldarstellung \(|n_+,n_-\rangle\) und der Drehimpulsdarstellung \(|j,m\rangle\) von Zuständen dieses Systems geben. Wie lautet dieser Zusammenhang und wie lassen sich damit Drehimpulse ganz allgemein deuten?


\subsection*{LSG}


\begin{itemize}
\item (1) \([a,a^\dagger]=1\)  \(a^\dagger_\pm a_\pm = N_\pm\)
\item (2) \([N,a^\dagger]=a^\dagger\)  \(\rightarrow [N_+,a^\dagger_+ a_-]=[N_+,a^\dagger_+]a_-=a^\dagger_+a_-\)
 \([N,a]=-a\)  \(\rightarrow [N_+,a^\dagger_- a_+]=a^\dagger_-[N_+, a_+]=-a^\dagger_-a_+\)
\(\Rightarrow [N_+,a^\dagger_\pm a_\mp] = \pm a^\dagger_\pm a_\mp\)

\([N,a]=-a\)  \(\rightarrow [N_-,a^\dagger_+ a_-]=a^\dagger_+[N_-,a_-]=-a^\dagger_+a_-\)
 \([N,a^\dagger]=a^\dagger\)  \(\rightarrow [N_-,a^\dagger_-a_+]=[N_-,a^\dagger_-]a_+=a^\dagger_-a_+\)
\(\Rightarrow [N_-,a^\dagger_\pm a_\mp] = \mp a^\dagger_\pm a_\mp\)
\(\Rightarrow -[N_-,a^\dagger_\pm a_\mp] = - \mp a^\dagger_\pm a_\mp= \pm a^\dagger_\pm a_\mp \)


\item (2)   \(a^\dagger_\pm a_\pm = N_\pm\)
\item (3) \([N_+,a^\dagger_+]=a^\dagger_+\)
\item (4) 

\end{itemize}



\begin{align}
 [J_z,J_\pm]&=  [\frac{\hbar}{2}( a^\dagger_+ a_+-a^\dagger_- a_-), \hbar a^\dagger_\pm a_\mp] \\
&= \frac{\hbar^2}{2} \left( [a^\dagger_+ a_+,a^\dagger_\pm a_\mp]-[a^\dagger_- a_-,a^\dagger_\pm a_\mp]  \right) \\
^{(1)}&= \frac{\hbar^2}{2} \left( [N_+,a^\dagger_\pm a_\mp]-[N_-,a^\dagger_\pm a_\mp]  \right) \\
^{(2)}&= \frac{\hbar^2}{2} \left( \pm a^\dagger_\pm a_\mp +  \pm a^\dagger_\pm a_\mp  \right) \\
&= \frac{\hbar^2}{2} \left( 2 \pm a^\dagger_\pm a_\mp \right) \\
&= \hbar \left( \hbar \pm a^\dagger_\pm a_\mp \right) \\
&= \pm \hbar J_\pm
\end{align}


\begin{itemize}
\item (1) \(J_x=\frac{1}{2}(J_++J_-)\),\(J_y=\frac{1}{2i}(J_+-J_-)\)  \(J_z=\frac{\hbar}{2}( a^\dagger_+ a_+-a^\dagger_- a_-)\)

 \begin{align} \vec J^2 &= J_x+J_y+J^2_z \\
&= \frac{1}{4}(J_++J_-)^2-\frac{1}{4}(J_+-J_-)^2+J^2_z \\
&= \frac{1}{4}(J^2_++J^2_-+J_+J_-+J_-J_+-J^2_+-J^2_-+J_+J_-+J_-J_+) + J^2_z \\
&= \frac{1}{4}(J_+J_-+J_-J_++J_+J_-+J_-J_+) + J^2_z \\
&= \frac{1}{4}(2J_+J_-+2J_-J_+) + J^2_z \\
&= \frac{1}{2}J_+J_-+  \frac{1}{2}J_-J_+ + J^2_z \\
\end{align}
\item (2) \([A,BC]=[A,B]C+B[A,C]\)
\item (3) \([J_z,J_\pm] = \pm \hbar J_\pm\)
\end{itemize}

\begin{align}
[\vec J^2,J_z]\stackrel{\mathrm{(1)}}=&[\frac{1}{2}J_+J_-+  \frac{1}{2}J_-J_+ + J^2_z,J_z] \\
&=[\frac{1}{2}J_+J_-,J_z] +  [\frac{1}{2}J_-J_+,J_z] + \underbrace{[J^2_z,J_z]}_{=0} \\
&=\frac{1}{2} ([J_+J_-,J_z] + [J_-J_+,J_z]) \\
&=\frac{1}{2} (-[J_z,J_+J_-] - [J_z,J_-J_+]) \\
^{(2)}&=\frac{1}{2} (-[J_z,J_+]J_--J_+[J_z,J_-] - [J_z,J_-]J_+- J_-[J_z,J_+]) \\
^{(3)}&=\frac{1}{2} (-\hbar J_+J_-+\hbar J_+J_- + \hbar J_-J_+- \hbar J_-J_+) \\
&=\frac{\hbar}{2} (-J_+J_-+ J_+J_- +J_-J_+- J_-J_+) \\
&=0 
\end{align}

\begin{itemize}
\item (1) \(J_\pm = \hbar a^\dagger_\pm a_\mp\)
\item (2) \([A,BC]=[A,B]C+B[A,C]\)
\item (3) \([a,a^\dagger]=1\)
\end{itemize}

\begin{align}
[J_+,J_-]&=[\hbar a^\dagger_+ a_-,\hbar a^\dagger_- a_+] \\
&=\hbar^2([a^\dagger_+ a_-,a^\dagger_- a_+]) \\
^{(2)}&=\hbar^2([a^\dagger_+ a_-,a^\dagger_-] a_+ +a^\dagger_- [a^\dagger_+ a_-, a_+]) \\
&=\hbar^2([a_-,a^\dagger_-]a^\dagger_+a_+ +a^\dagger_-a_-[a^\dagger_+, a_+]) \\
^{(3)}&=\hbar^2(a^\dagger_+a_+ - a^\dagger_-a_-) \\
&=2\hbar(\frac{\hbar}{2}(a^\dagger_+a_+ - a^\dagger_-a_-)) \\
&=2\hbar J_z
\end{align}


\begin{itemize}
\item (1) \([a,a^\dagger]=1=aa^\dagger-a^\dagger a \leftrightarrow aa^\dagger=1+a^\dagger a\)
\end{itemize}


\begin{align} \vec J^2 &= J_x+J_y+J^2_z = \frac{1}{2}J_+J_-+  \frac{1}{2}J_-J_+ + J^2_z \\
&= \frac{1}{2}(J_+J_-+ J_-J_+) + J^2_z\\
&= \frac{\hbar^2}{2}(a^\dagger_+a_-a^\dagger_-a_+ + a^\dagger_-a_+a^\dagger_+a_-)+ \frac{\hbar^2}{4}( a^\dagger_+ a_+-a^\dagger_- a_-)^2 \\
&= \frac{\hbar^2}{2}(a^\dagger_+a_+a_-a^\dagger_- + a^\dagger_-a_-a_+a^\dagger_+)+ \frac{\hbar^2}{4}( a^\dagger_+ a_+-a^\dagger_- a_-)^2 \\
&= \frac{\hbar^2}{2}(a^\dagger_+a_+(1+a^\dagger_-a_-) + a^\dagger_-a_-(1+a^\dagger_+a_+))+ \frac{\hbar^2}{4}( a^\dagger_+ a_+-a^\dagger_- a_-)^2 \\
&= \frac{\hbar^2}{2}(N_+(1+N_-) + N_-(1+N_+))+ \frac{\hbar^2}{4}( N_+-N_-)^2 \\
&= \frac{\hbar^2}{4}(2N_+(1+N_-) + 2N_-(1+N_+) + N^2_+-N^2_--N_+N_--N_-N_+) \\
&= \frac{\hbar^2}{4}(2N_++2N_+N_- + 2N_-+2N_-N_+ + N^2_+-N^2_--N_+N_--N_-N_+) \\
&= \frac{\hbar^2}{4}(2N_++N_+N_- + 2N_-+N_-N_+ + N^2_+-N^2_-) \\
&= \frac{\hbar^2}{4}(2N_++  2N_- + N^2_+-N^2_- + N_+N_- +N_-N_+ ) \\
&= \frac{\hbar^2}{4}(2N_++  2N_- + (N_++N_-)^2) \\
&= \frac{\hbar^2}{4}(2N + (N)^2) \\
&= \hbar^2(\frac{N^2}{2} + \frac{N}{2} ) \\
\vec J^2 &= \hbar^2 \frac{N}{2}(\frac{N}{2}+1)
\end{align}



Zusammenhang mit \(|jm\rangle\)

\(\vec J^2 = \hbar^2 \frac{N}{2}(\frac{N}{2}+1) \stackrel{\mathrm{!}}=\hbar^2 j(j+1)\)
\[ \Rightarrow j=\frac{N_++N_-}{2} \]

\(J_z=\frac{\hbar}{2}(N_+-N_-)  \stackrel{\mathrm{!}}= \hbar m \)

\[ \Rightarrow m=\frac{N_+-N_-}{2} \]

\end{document}
