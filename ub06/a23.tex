\documentclass[10pt,a4paper,oneside,fleqn]{article}
\usepackage{geometry}
\geometry{a4paper,left=20mm,right=20mm,top=1cm,bottom=2cm}
\usepackage[utf8]{inputenc}
%\usepackage{ngerman}
\usepackage{amsmath}                % brauche ich um dir Formel zu umrahmen.
\usepackage{amsfonts}                % brauche ich für die Mengensymbole
\usepackage{graphicx}
\setlength{\parindent}{0px}
\setlength{\mathindent}{10mm}
\usepackage{bbold}                    %brauche ich für die doppel Zahlen Darstellung (Einheitsmatrix z.B)
\usepackage{dsfont}          %F�r den Einheitsoperator \mathds 1


\usepackage{color}
\usepackage{titlesec} %sudo apt-get install texlive-latex-extra

\definecolor{darkblue}{rgb}{0.1,0.1,0.55}
\definecolor{verydarkblue}{rgb}{0.1,0.1,0.35}
\definecolor{darkred}{rgb}{0.55,0.2,0.2}

%hyperref Link color
\usepackage[colorlinks=true,
        linkcolor=darkblue,
        citecolor=darkblue,
        filecolor=darkblue,
        pagecolor=darkblue,
        urlcolor=darkblue,
        bookmarks=true,
        bookmarksopen=true,
        bookmarksopenlevel=3,
        plainpages=false,
        pdfpagelabels=true]{hyperref}

\titleformat{\chapter}[display]{\color{darkred}\normalfont\huge\bfseries}{\chaptertitlename\
\thechapter}{20pt}{\Huge}

\titleformat{\section}{\color{darkblue}\normalfont\Large\bfseries}{\thesection}{1em}{}
\titleformat{\subsection}{\color{verydarkblue}\normalfont\large\bfseries}{\thesubsection}{1em}{}

% Notiz Box
\usepackage{fancybox}
\newcommand{\notiz}[1]{\vspace{5mm}\ovalbox{\begin{minipage}{1\textwidth}#1\end{minipage}}\vspace{5mm}}

\usepackage{cancel}
\setcounter{secnumdepth}{3}
\setcounter{tocdepth}{3}





%-------------------------------------------------------------------------------
%Diff-Makro:
%Das Diff-Makro stellt einen Differentialoperator da.
%
%Benutzung:
% \diff  ->  d
% \diff f  ->  df
% \diff^2 f  ->  d^2 f
% \diff_x  ->  d/dx
% \diff^2_x  ->  d^2/dx^2
% \diff f_x  ->  df/dx
% \diff^2 f_x  ->  d^2f/dx^2
% \diff^2{f(x^5)}_x  ->  d^2(f(x^5))/dx^2
%
%Ersetzt man \diff durch \pdiff, so wird der partieller
%Differentialoperator dargestellt.
%
\makeatletter
\def\diff@n^#1{\@ifnextchar{_}{\diff@n@d^#1}{\diff@n@fun^#1}}
\def\diff@n@d^#1_#2{\frac{\textrm{d}^#1}{\textrm{d}#2^#1}}
\def\diff@n@fun^#1#2{\@ifnextchar{_}{\diff@n@fun@d^#1#2}{\textrm{d}^#1#2}}
\def\diff@n@fun@d^#1#2_#3{\frac{\textrm{d}^#1 #2}{\textrm{d}#3^#1}}
\def\diff@one@d_#1{\frac{\textrm{d}}{\textrm{d}#1}}
\def\diff@one@fun#1{\@ifnextchar{_}{\diff@one@fun@d #1}{\textrm{d}#1}}
\def\diff@one@fun@d#1_#2{\frac{\textrm{d}#1}{\textrm{d}#2}}
\newcommand*{\diff}{\@ifnextchar{^}{\diff@n}
  {\@ifnextchar{_}{\diff@one@d}{\diff@one@fun}}}
%
%Partieller Diff-Operator.
\def\pdiff@n^#1{\@ifnextchar{_}{\pdiff@n@d^#1}{\pdiff@n@fun^#1}}
\def\pdiff@n@d^#1_#2{\frac{\partial^#1}{\partial#2^#1}}
\def\pdiff@n@fun^#1#2{\@ifnextchar{_}{\pdiff@n@fun@d^#1#2}{\partial^#1#2}}
\def\pdiff@n@fun@d^#1#2_#3{\frac{\partial^#1 #2}{\partial#3^#1}}
\def\pdiff@one@d_#1{\frac{\partial}{\partial #1}}
\def\pdiff@one@fun#1{\@ifnextchar{_}{\pdiff@one@fun@d #1}{\partial#1}}
\def\pdiff@one@fun@d#1_#2{\frac{\partial#1}{\partial#2}}
\newcommand*{\pdiff}{\@ifnextchar{^}{\pdiff@n}
  {\@ifnextchar{_}{\pdiff@one@d}{\pdiff@one@fun}}}
\makeatother
%
%Das gleich nur mit etwas andere Syntax. Die Potenz der Differentiation wird erst
%zum Schluss angegeben. Somit lautet die Syntax:
%
% \diff_x^2  ->  d^2/dx^2
% \diff f_x^2  ->  d^2f/dx^2
% \diff{f(x^5)}_x^2  ->  d^2(f(x^5))/dx^2
% Ansonsten wie Oben.
%
%Ersetzt man \diff durch \pdiff, so wird der partieller
%Differentialoperator dargestellt.
%
%\makeatletter
%\def\diff@#1{\@ifnextchar{_}{\diff@fun#1}{\textrm{d} #1}}
%\def\diff@one_#1{\@ifnextchar{^}{\diff@n{#1}}%
%  {\frac{\textrm d}{\textrm{d} #1}}}
%\def\diff@fun#1_#2{\@ifnextchar{^}{\diff@fun@n#1_#2}%
%  {\frac{\textrm d #1}{\textrm{d} #2}}}
%\def\diff@n#1^#2{\frac{\textrm d^#2}{\textrm{d}#1^#2}}
%\def\diff@fun@n#1_#2^#3{\frac{\textrm d^#3 #1}%
%  {\textrm{d}#2^#3}}
%\def\diff{\@ifnextchar{_}{\diff@one}{\diff@}}
%\newcommand*{\diff}{\@ifnextchar{_}{\diff@one}{\diff@}}
%
%Partieller Diff-Operator.
%\def\pdiff@#1{\@ifnextchar{_}{\pdiff@fun#1}{\partial #1}}
%\def\pdiff@one_#1{\@ifnextchar{^}{\pdiff@n{#1}}%
%  {\frac{\partial}{\partial #1}}}
%\def\pdiff@fun#1_#2{\@ifnextchar{^}{\pdiff@fun@n#1_#2}%
%  {\frac{\partial #1}{\partial #2}}}
%\def\pdiff@n#1^#2{\frac{\partial^#2}{\partial #1^#2}}
%\def\pdiff@fun@n#1_#2^#3{\frac{\partial^#3 #1}%
%  {\partial #2^#3}}
%\newcommand*{\pdiff}{\@ifnextchar{_}{\pdiff@one}{\pdiff@}}
%\makeatother

%-------------------------------------------------------------------------------
%%Nützliche Makros um in der Quantenmechanik Bras, Kets und das Skalarprodukt
%%zwischen den beiden darzustellen.
%%Benutzung:
%% \bra{x}  ->    < x |
%% \ket{x}  ->    | x >
%% \braket{x}{y} ->   < x | y >



\newcommand\bra[1]{\left\langle #1 \right|}
\newcommand\ket[1]{\left| #1 \right\rangle}
\newcommand\braket[2]{%
 \left\langle \vphantom{#2} #1%
   \middle|%
   \vphantom{#1} #2\right\rangle}%

%-------------------------------------------------------------------------------
%%Aus dem Buch:
%%Titel:  Latex in Naturwissenschaften und Mathematik
%%Autor:  Herbert Voß
%%Verlag: Franzis Verlag, 2006
%%ISBN:   3772374190, 9783772374197
%%
%%Hier werden drei Makros definiert:\mathllap, \mathclap und \mathrlap, welche
%%analog zu den aus Latex bekannten \rlap und \llap arbeiten, d.h. selbst
%%keinerlei horizontalen Platz benötigen, aber dennoch zentriert zum aktuellen
%%Punkt erscheinen.

\newcommand*\mathllap{\mathstrut\mathpalette\mathllapinternal}
\newcommand*\mathllapinternal[2]{\llap{$\mathsurround=0pt#1{#2}$}}
\newcommand*\clap[1]{\hbox to 0pt{\hss#1\hss}}
\newcommand*\mathclap{\mathpalette\mathclapinternal}
\newcommand*\mathclapinternal[2]{\clap{$\mathsurround=0pt#1{#2}$}}
\newcommand*\mathrlap{\mathpalette\mathrlapinternal}
\newcommand*\mathrlapinternal[2]{\rlap{$\mathsurround=0pt#1{#2}$}}

%%Das Gleiche nur mit \def statt \newcommand.
%\def\mathllap{\mathpalette\mathllapinternal}
%\def\mathllapinternal#1#2{%
%  \llap{$\mathsurround=0pt#1{#2}$}% $
%}
%\def\clap#1{\hbox to 0pt{\hss#1\hss}}
%\def\mathclap{\mathpalette\mathclapinternal}
%\def\mathclapinternal#1#2{%
%  \clap{$\mathsurround=0pt#1{#2}$}%
%}
%\def\mathrlap{\mathpalette\mathrlapinternal}
%\def\mathrlapinternal#1#2{%
%  \rlap{$\mathsurround=0pt#1{#2}$}% $
%}

%-------------------------------------------------------------------------------
%%Hier werden zwei neue Makros definiert \overbr und \underbr welche analog zu
%%\overbrace und \underbrace funktionieren jedoch die Gleichung nicht
%%'zerreißen'. Dies wird ermöglicht durch das \mathclap Makro.

\def\overbr#1^#2{\overbrace{#1}^{\mathclap{#2}}}
\def\underbr#1_#2{\underbrace{#1}_{\mathclap{#2}}}

%couchdb db=physik
%couchdb id=qm2uba23_Harmonischer_Oszillator_mit_Stoerung
%couchdb tags=qm2ub
%couchdb pdflink=http://wernwa-physik-ka.googlecode.com/svn/qm2ub/ub06/a23.pdf

\begin{document}
\section*{Aufgabe 23: Harmonischer Oszillator mit Störung}

Ein eindimensionaler harmonischer Oszillator mit Frequenz \(\omega\) bekommt einen Störterm

\[V=\frac{1}{2}\epsilon m \omega^2x^2\]

Bestimmen Sie den gestörten Grundzustand in erster und dessen Energieeigenwert in zweiter Ordnung der zeitunabhängigen Störungsrechnung. Sie können das Problem auch exakt lösen. Wie vergleicht sich Ihr Ergebnis für das gestörte Energieniveau mit dem exakten Ergebnis für \(\epsilon \rightarrow 0\)?

\subsection*{LSG}

\[ H_0 = \frac{p^2}{2m} + \frac{1}{2}m\omega^2 x^2 \]

\begin{align}
H &= H_0 + \lambda V \\
& = \frac{p^2}{2m} + \frac{1}{2}m\omega^2 x^2 +  \frac{1}{2}\epsilon m \omega^2x^2  \\
& = \frac{p^2}{2m} + \frac{1}{2}m\omega^2(1  + \epsilon ) x^2  \\
& = \frac{p^2}{2m} + \frac{1}{2}m\Omega^2 x^2  \\
\end{align}


mit \(\Omega^2 = \omega^2(1  + \epsilon)\)

\[ H = \hbar \Omega(n+\frac{1}{2})\]

\(V=\frac{1}{2}\epsilon m \omega^2x^2\) mit \(x=\sqrt{\frac{\hbar}{2\omega m}}(a^\dagger+a)\) 

\begin{align}
V&=\frac{1}{2}\epsilon m \omega^2\frac{\hbar}{2\omega m}(a^\dagger+a)^2 \\
&=\frac{1}{4}\epsilon \omega\hbar(a^\dagger+a)^2 \\
&=\frac{1}{4}\epsilon \omega\hbar(a^{\dagger 2} + a^2+a^\dagger a+ aa^\dagger) \text{mit\quad} aa^\dagger=1+a^\dagger a\\
&=\frac{1}{4}\epsilon \omega\hbar(a^{\dagger 2} + a^2+2a^\dagger a+ 1)
\end{align}


Berechnung des Eigenzustandes in erster Ordnung:

\(|n\rangle =|n^{(0)}\rangle + (\lambda)\sum_{k\neq n}|k^{(0)}\rangle \frac{\langle k^{(0)}|V|n^{(0)}\rangle }{E_n^{(0)}-E_k^{(0)}}\)


\begin{align}
\langle k^{(0)}|V|n^{(0)}\rangle &=  \frac{1}{4}\epsilon \omega\hbar \langle k^{(0)}|(a^{\dagger 2} + a^2+2a^\dagger a+ 1)|n^{(0)}\rangle\\
&=\frac{1}{4}\epsilon \omega\hbar( \langle k^{(0)}|a^{\dagger 2}|n^{(0)}\rangle +\langle k^{(0)}| a^2|n^{(0)}\rangle+ \\
&+2\langle k^{(0)}|a^\dagger a|n^{(0)}\rangle+\langle k^{(0)}|1|n^{(0)}\rangle)\\
&=\frac{1}{4}\epsilon \omega\hbar[\sqrt{n(n-1)}\delta_{k,n-2}+\sqrt{(n+1)(n+2)}\delta_{k,n+2}+\\
&+2\sqrt{n\cdot n}\delta_{k,n}+\delta_{k,n}]\\
&=\frac{1}{4}\epsilon \omega\hbar[\sqrt{n(n-1)}\delta_{k,n-2}+\sqrt{(n+1)(n+2)}\delta_{k,n+2}+\\
&+(2n+1)\delta_{k,n}]
\end{align}


Berechnung der Gestörten Energie:

(nichtgestörter Term) 0-Ordnung \(E^{(0)}_n = \hbar \omega(n+\frac{1}{2})\)

Störungsterm 1-Ordnung: \(E^{(1)}_n = \langle \phi_n|V|\phi_n\rangle = \frac{1}{2}\epsilon \omega\hbar (n + \frac{1}{2}) \) 


\begin{align}
\rightarrow \langle \phi_n | V | \phi_n \rangle &= \frac{1}{4}\epsilon \omega\hbar \langle \phi_n |(a^{\dagger 2} + a^2+2a^\dagger a+ 1) | \phi_n \rangle \\
&= \frac{1}{4}\epsilon \omega\hbar (\underbrace{\langle \phi_n |a^{\dagger 2}|\phi_n \rangle}_{=0} + \underbrace{\langle \phi_n|a^2|\phi_n \rangle}_{=0}+2\underbrace{\langle \phi_n|a^\dagger a|\phi_n \rangle}_{=n}+ \underbrace{\langle \phi_n|1| \phi_n \rangle}_{=1})\\
&= \frac{1}{4}\epsilon \omega\hbar (2n + 1)\\
&= \frac{1}{2}\epsilon \omega\hbar (n + \frac{1}{2})\\
\end{align}

Störungsterm 2-Ordnung:
\[E^{(2)}_n = \sum_{n\neq m} \frac{|\langle \phi_m| V |\phi_n\rangle|^2}{E^{(0)}_n-E^{(0)}_m}\]

Terme die übrig bleiben:

mit \(a|n\rangle = \sqrt{n}|n-1\rangle\), \(a^\dagger|n\rangle = \sqrt{n+1}|n+1\rangle\)

\begin{align}
\langle \phi_{n+2}|V|\phi_n\rangle &=\frac{1}{4}\epsilon \omega\hbar \langle \phi_{n+2}|a^{\dagger 2}|\phi_n\rangle\\
&= \frac{1}{4}\epsilon \omega\hbar \sqrt{n+1} \langle \phi_{n+2}|a^{\dagger}|\phi_{n+1}\rangle\\
&=  \frac{1}{4}\epsilon \omega\hbar \sqrt{n+1} \sqrt{n+2} \langle \phi_{n+2}|\phi_{n+2}\rangle\\
&=  \frac{1}{4}\epsilon \omega\hbar \sqrt{(n+1)(n+2)} \\
\end{align}

\begin{align}
\langle \phi_{n-2}|V|\phi_n\rangle &=\frac{1}{4}\epsilon \omega\hbar \langle \phi_{n-2}|a^2|\phi_n\rangle\\
&= \frac{1}{4}\epsilon \omega\hbar \sqrt{n} \langle \phi_{n-2}|a|\phi_{n-1}\rangle\\
&=  \frac{1}{4}\epsilon \omega\hbar \sqrt{n} \sqrt{n-1} \langle \phi_{n-2}|\phi_{n-2}\rangle\\
&=  \frac{1}{4}\epsilon \omega\hbar \sqrt{n(n-1)} \\
\end{align}

Beim Einsetzen beachten, dass die Indizies \(n,m\) im Nenner und Zähler genau gegenüber liegen.

\begin{align}
\rightarrow E^{(2)}_n &= \frac{|\langle \phi_{n+2}|V|\phi_n\rangle|^2}{E^{(0)}_n-E^{(0)}_{n+2}} + \frac{|\langle \phi_{n-2}|V|\phi_n\rangle|^2}{E^{(0)}_n-E^{(0)}_{n-2}}\\
&=\frac{\frac{1}{16}\epsilon^2 \omega^2\hbar^2 (n+1)(n+2)}{\hbar \omega(n+\frac{1}{2})-\hbar \omega(n+2+\frac{1}{2})} + \frac{\frac{1}{16}\epsilon^2 \omega^2\hbar^2 n(n-1)}{\hbar \omega(n+\frac{1}{2})-\hbar \omega(n-2+\frac{1}{2})}\\
&=\frac{1}{16}\epsilon^2 \omega\hbar \left[\frac{(n+1)(n+2)}{n+\frac{1}{2}-n-2-\frac{1}{2}} + \frac{n(n-1)}{n+\frac{1}{2}-n+2-\frac{1}{2}}\right]\\
&=\frac{1}{16}\epsilon^2 \omega\hbar \left[-\frac{(n+1)(n+2)}{2} + \frac{n(n-1)}{2}\right]\\
&=\frac{1}{16\cdot 2}\epsilon^2 \omega\hbar [-(n+1)(n+2) + n(n-1)]\\
&=\frac{1}{16\cdot 2}\epsilon^2 \omega\hbar [-n^2-2n-n-2 + n^2-n]\\
&=-\frac{1}{8}\epsilon^2 \omega\hbar [n+\frac{1}{2}]\\
\end{align}

Der Energiewert bis in die Störung 2-Ordnung lautet somit:

\begin{align}
\Rightarrow \tilde E_n &= E^{(0)}_n+E^{(1)}_n+E^{(2)}_n\\
&= \hbar \omega(n+\frac{1}{2}) + \frac{1}{2}\epsilon \omega\hbar (n + \frac{1}{2})- \frac{1}{8}\epsilon^2 \omega\hbar [n+\frac{1}{2}]\\
&=\hbar \omega(n+\frac{1}{2})(1+\frac{\epsilon}{2}-\frac{\epsilon^2}{4})
\end{align}

Exakte Rechnung, vermute mal nur möglich wenn man von hinten nach forne 'denkt'/'rechnet'. Falls man die Energie \(\tilde E_n\) mit \(E^{(0)}_n = \hbar\omega(n+\frac{1}{2})\) vergleicht, so muss \(\tilde \omega\) zunächst dem Term \((1+\frac{\epsilon}{2}--\frac{\epsilon^2}{4})\) gleich sein. Mit einem Supermanns-Röntgen-Blick sieht man sofort (oder man is so verwegen und hat sämtliche Taylor-Entwicklungen auf seinen Arm tätoviert) dass diese Klammer eine Wurzelentwicklung seien könnte, etwa derart:

\[ \sqrt{1+\epsilon} = (1+\frac{\epsilon}{2}-\frac{\epsilon^2}{4}+\mathcal O(\epsilon^3)) \]

Also setzt man den Hamiltonoperator mit der 'exakten' \(\tilde\omega=\omega\sqrt{1+\epsilon}\) Frequenz an:

\[ H = \frac{p^2}{2m}+\frac{1}{2}m\tilde\omega^2x^2=\frac{p^2}{2m}+\frac{1}{2}m\omega(1+\epsilon)x^2 \]
\[\rightarrow E_n = \hbar \omega(n+\frac{1}{2})(1+\frac{\epsilon}{2}-\frac{\epsilon^2}{4} +\mathcal O(\epsilon^3) )\]

Exaktes Ergebnis entspicht dem Ergebnis der Störungsrechnung (zumindest in den führenden Potenzen). Für \(\epsilon \rightarrow 0\) kann \(E_n\) als exakt angesehen werden.







\end{document}
